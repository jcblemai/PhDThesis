\begin{fullwidth}
   \maketitle
   \blankpage
\end{fullwidth}
% r.7 dedication
%\cleardoublepage
%~\vfill
%\begin{doublespace}
%\noindent\fontsize{18}{22}\selectfont\itshape
%\nohyphenation
%Dedicated to those who appreciate \LaTeX{} 
%and the work of \mbox{Edward R.~Tufte} 
%and \mbox{Donald E.~Knuth}.
%\end{doublespace}
%\vfill
%\vfill
%\end{fullwidth}
% r.9 introduction
%\cleardoublepage
%\begin{fullwidth}

\newenvironment{dedication}
  {\clearpage           % w- want a new page
   \thispagestyle{empty}% no header and footer
   \vspace*{\stretch{2}}% some space at the top 
   %\itshape             % the text is in italics
  
\leftskip=10cm
\raggedleft
\parindent=0pt
\begin{fullwidth}
  }
  {\par % end the paragraph
   \vspace{\stretch{3}} % space at bottom is three times that at the top
   \clearpage           % finish off the page
\end{fullwidth}  }
 \begin{dedication} %\leftskip=10cm 
\raggedleft\textit{
Vedrò con mio diletto~~~~~~~~~~~~~~~~~~~~\\ 
L'alma dell'alma mia~~~~~~~~~~~~~~~~~~~~\\ %, dell'alma mia  \\
Il core del mio cor~~~~~~~~~~~~~~~~~~~~\\
Pien di contento~~~~~~~~~~~~~~~~~~~~\\ %, pien di contento  \\
Vedrò con mio diletto~~~~~~~~~~~~~~~~~~~~\\
L'alma dell'alma mia~~~~~~~~~~~~~~~~~~~~\\ %, dell'alma mia \\
Il cor di questo cor~~~~~~~~~~~~~~~~~~~~\\
Pien di contento~~~~~~~~~~~~~~~~~~~~\\ %, pien di contento \\
E se dal caro oggetto~~~~~~~~~~~~~~~~~~~~\\
Lungi convien che sia~~~~~~~~~~~~~~~~~~~~\\ %, convien che sia \\
Sospirerò penando~~~~~~~~~~~~~~~~~~~~\\
Ogni momento~~~~~~~~~~~~~~~~~~~~\\
Vedrò con mio diletto~~~~~~~~~~~~~~~~~~~~\\
L'alma dell'alma mia~~~~~~~~~~~~~~~~~~~~\\ %, dell'alma mia \\
Il core del mio cor~~~~~~~~~~~~~~~~~~~~\\
Pien di contento~~~~~~~~~~~~~~~~~~~~\\ %, pien di contento \\
Vedrò con mio diletto~~~~~~~~~~~~~~~~~~~~\\
L'alma dell'alma mia~~~~~~~~~~~~~~~~~~~~\\ %, dell'alma mia \\
Il cor di questo cor~~~~~~~~~~~~~~~~~~~~ \\
Pien di contento~~~~~~~~~~~~~~~~~~~~ \\ %, pien di contento \\
-- Anastasio~~~~~~~~~~~~~~~~~~~~ \\
%\hspace{11cm}
}
% \textit{On vise le Porsche Panamera et dire aux proches que ça ira.} \\
%\hspace{14cm}-- \textsc{Maes}
\end{dedication}

\cleardoublepage
 \vspace{-.5cm}
\chapter*{Acknowledgements} \addcontentsline{toc}{chapter}{Acknowledgements}\markboth{Acknowledgements}{}
 \vspace{-.5cm}
%Spending 4.5 years to study the dynamics of cholera (and \textsc{covid}-19) was a surprising decision; but a great one so much my Ph.D. journey, while challenging, has been incredibly rewarding -- in no small part thanks to amazing colleagues.
The present thesis is the result of the past 4.5 years spent studying the dynamics of cholera and \textsc{covid}-19. While challenging, this journey has been incredibly rewarding -- in no small part thanks to amazing colleagues. I am grateful to Prof. Andrea Rinaldo for warmly welcoming me to science and academia, and for allowing me to explore a subject that I grew to love. Thank you for encouraging me to pursue my research interests while benefiting from your unconditional support. % It made for a great inspiration and I'll take with me the rules of the ECHO lab.
 \marginnote{
  \begin{center}\textsc{Standing on the shoulder of giants}\end{center}
 During this thesis, I have been fortunate to participate to the GTFCC annual meeting, IDDconf, and SISMID. These conferences were an incredible learning experience, and I would like to thank the infectious disease modeling community for being so welcoming, inclusive, and patient. I learned and still learn every day from many of you. 
 
 This thesis is built on many datasets, from rainfall to reported cholera cases. I am very grateful to the workers all along the chain from collection to curation who have made my job possible.

 Thanks to the reviewers and co-authors from whom I have learned both scientific concepts and valuable writing tips.
 
 Tools shape the way you think about problems. This thesis was not only made possible but also sculpted by the languages and libraries that were used. I want to acknowledge all the open-source maintainers and contributors who tirelessly make, document, and maintain powerful tools, enabling everybody to use cutting-edge methods and to fill bug-reports.}
My most sincere thanks go to Damiano Pasetto for deeply caring about every aspect of my Ph.D. experience and for either pushing or carrying me through these years. It has been wonderful to evolve under your supervision.
Scientifically, this thesis has benefited enormously from Javier Perez-Saez. Thank you for your insights on modeling, inference, and life in general.
 Thank you Mario Zanon for making learning complicated optimal control algorithms a pleasure.
Many thanks to Jacques Fellay and Andrew Azman for your scientific insights and for introducing me to new lines of work.
  
 Upon my arrival at the ECHO lab in 2017, I've been welcomed by wonderful friends and colleagues. Thanks Silvia for the best tea cookies ever; Giezi for caring so much about the lab and organizing our pasta Thursdays; and Luca because sports and meta-science are meant to go together. I wish to thanks Mitra for sharing delicious Iranian delicacies and a unique perspective on life; Paolo for the support at all time and spontaneous lunch discussions; Filippo for teaching me experimental science and theoretical parenting. Finally Cristiano: I am truly grateful to have you as an office mate, colleague, and friend. It did change my last year at EPFL.
 
  From my intense and abbreviated stay in Baltimore, I would like to thank Prof. Justin Lessler for welcoming me and for trusting me on the scenario pipeline project. Thank you Elizabeth, because wine and sports mix so well with science. Thanks, Hannah, Kyu, Shaun, Steve, and Quifang, for the inspirational science and the warm welcome in Baltimore. We loved our time there. Thanks to the COVID Scenario Pipeline team, working at night is much more pleasant with you all; especially to Josh for long hours pair debugging sessions.
 
 Merci à mes ami·e·s et à ma famille pour votre soutien et tous les moments de joie; les cafés en EL, les longues discussions à toute heure et toutes ces belles réunions.

Merci à mes parents de m'avoir ouvert au monde et de m'avoir tant encouragé et soutenu au cours de cette thèse. À Céline, Eugène et Olivier, merci évidemment pour tout; sans votre folie et votre aide, cela n'aurait pas été pareil.

Finalement, merci Marion ! Merci d'avoir traversé avec moi toutes ces épreuves, merci d'avoir donné un sens à tout ça, merci d'avoir rendu ces quatre ans aussi doux. Merci pour tout le reste évidemment. Merci au merveilleux petit bébé qui bouge pour m'encourager et que l'on a si hâte de rencontrer !
 

 \chapter*{Summary} \addcontentsline{toc}{chapter}{Summary}\markboth{Summary}{}%research problem and objective
  \marginnote{\paragraph{Keywords} cholera, \textsc{covid}-19, epidemiology, public health, ecohydrology, infectious disease dynamics, SIR, mathematical modeling, statistical inference, optimal control.}
\vspace{-.5cm}
Emerging and existing infectious diseases pose a constant threat to individuals and communities across the world. In many cases, the burden of these diseases is preventable through public health interventions. However, taking the right decisions and designing effective policies is an intricate task: epidemics are complex phenomena resulting from the interaction between the environment, pathogens, individuals, and societies. Modeling offers a principled way to reason about infectious disease dynamics from scarce and biased information and to guide decision-makers towards effective policies. 

This thesis tackles selected topics in cholera and \textsc{covid}-19 modeling towards informed public-health decisions. These two contrasting diseases were associated by a twist of fate, but also through the lens of a common modeling approach:   compartmental, SIR-based, models are conditioned on the available evidence using computer-age statistical inference frameworks. A set of five models is developed, each tackling a different facet of the spread and control of these two infectious diseases. Each model aims at answering questions related to either the understanding of the mechanisms behind disease transmission, the projection of the future dynamics under different scenarios, or the assessment of the effectiveness of past interventions. Moreover, a novel application of epidemiological models to the formal design of control policies is proposed. Optimal control provides a rigorous framework to identify the most effective control measures under a set of operational constraints, providing a benchmark on what it is possible to achieve with the available resources.

%Your key results
The results presented in this thesis range from scientific insight on the relationship between cholera and rainfall in Juba, South Sudan to the COVID Scenario Pipeline which produces reports used to inform the response to the \textsc{covid}-19 pandemic of different governmental entities. Furthermore, the effectiveness of the non-pharmaceutical interventions against \textsc{covid}-19 in Switzerland is evaluated; and so is the probability of eliminating cholera from Haiti under different scenarios of mass vaccination campaigns. Finally, the development of an optimal control framework towards the effective spatial allocation of vaccines against SARS-CoV-2 in Italy closes this conversation of models.

%Your conclusion
The present thesis demonstrates how infectious disease modeling enables informed decision-making by projecting the uncertainties under the light of the available evidence. It also highlights the effort needed to tailor the models and inference methods to the specificities of the transmission setting and the research question considered. From insights on transmission pathways to weekly reports aimed at decision-makers, it explores different applications of infectious disease modeling. Methods developed along the way enrich the toolbox available to modelers, to guide policy decisions further towards a reduction of the burden of infectious diseases on communities.

%\enlargthispage[2\baselineskip]
 \vspace{-.5cm}
 \chapter*{Résumé} \addcontentsline{toc}{chapter}{Résumé}\markboth{Résumé}{}
   \marginnote{\paragraph{Mot-clés} choléra, \textsc{covid}-19, épidemiologie, santé publique, écohydrologie, dynamique des maladies infectieuses, SIR, modélisation mathématique, inférence statistique, commande optimale.}
\vspace{-.5cm}
Existantes ou émergentes, les maladies infectieuses constituent une menace constante sur les individus et les communautés du monde entier. Dans de nombreux cas, le fardeau de ces maladies peut être évité grâce à des interventions de santé publique. Cependant, prendre les bonnes décisions et concevoir des politiques efficaces sont des tâches difficiles : les épidémies sont des phénomènes complexes résultant de l'interaction entre l'environnement, les agents pathogènes, les individus et les sociétés. La modélisation offre un moyen pour raisonner sur la dynamique de ces maladies à partir d'informations rares et biaisées, et ainsi guider les décideurs vers des politiques efficaces. 

Cette thèse aborde la modélisation du choléra et de la \textsc{covid}-19 pour la prise de décisions de santé publique éclairées. Ces deux maladies très différentes ont été associées ici par le hasard des événements, mais aussi par le biais d'une approche analytique commune: des modèles compartimentaux, basés sur le SIR, sont conditionnés par les données disponibles en utilisant des méthodes d'inférence statistique modernes. Un ensemble de cinq modèles y est développé, chacun abordant des facettes différentes de la propagation et du contrôle de ces deux maladies. Chaque modèle cherche à répondre à des questions liées soit à la compréhension des mécanismes de transmission, soit à la projection de la dynamique future selon différents scénarios, ou encore à l'évaluation de l'efficacité d'interventions passées. En outre, une nouvelle application des modèles épidémiologiques y est proposée: la conception formelle de politiques d'allocation de ressource de contrôle. La commande optimale offre un cadre rigoureux pour identifier les mesures de contrôle les plus efficaces tout en respectant les contraintes opérationnelles, ce qui permet de déterminer ce qu'il est possible d'atteindre avec les ressources disponibles.

Les résultats présentés dans cette thèse vont de la compréhension scientifique de la relation entre le choléra et les précipitations à Juba, au Sud-Soudan, à la COVID Scenario Pipeline qui produit des rapports utilisés par différentes entités gouvernementales pour répondre à la pandémie de \textsc{covid}-19 . En outre, l'efficacité des interventions non-pharmaceutiques contre la \textsc{covid}-19 en Suisse est évaluée, de même que la probabilité d'éliminer le choléra d'Haïti selon différents scénarios de campagnes de vaccination de masse. Enfin, le développement d'un outil de commande optimale qui produit une allocation spatiale efficace des vaccins contre le SRAS-CoV-2 en Italie clôt cette conversation de modèles.

La présente thèse montre comment la modélisation des maladies infectieuses permet une prise de décision éclairée en projetant les incertitudes à la lumière des données disponibles. Elle souligne également l'effort nécessaire pour adapter les modèles et les méthodes d'inférence aux spécificités du contexte de transmission et de la question de recherche considérée. De la compréhension des voies de transmission aux rapports hebdomadaires destinés aux décideurs, elle explore les différentes applications de la modélisation des maladies infectieuses. Les méthodes développées en cours de route enrichissent la boîte à outils mise à disposition de la communauté scientifique, afin de mieux orienter la prise de décisions avec pour but une réduction durable du fardeau des maladies infectieuses sur les communautés.

 %\pdfbookmark[section]{\contentsname}{toc}
%\begin{fullwidth}\tableofcontents\listoffigures\listoftables\end{fullwidth}