\begin{fullwidth}
   \maketitle
   \blankpage
\end{fullwidth}
% r.7 dedication
%\cleardoublepage
%~\vfill
%\begin{doublespace}
%\noindent\fontsize{18}{22}\selectfont\itshape
%\nohyphenation
%Dedicated to those who appreciate \LaTeX{} 
%and the work of \mbox{Edward R.~Tufte} 
%and \mbox{Donald E.~Knuth}.
%\end{doublespace}
%\vfill
%\vfill
%\end{fullwidth}
% r.9 introduction
%\cleardoublepage
%\begin{fullwidth}

\newenvironment{dedication}
  {\clearpage           % w- want a new page
   \thispagestyle{empty}% no header and footer
   \vspace*{\stretch{2}}% some space at the top 
   %\itshape             % the text is in italics
  
\leftskip=10cm
\raggedleft
\parindent=0pt
\begin{fullwidth}
  }
  {\par % end the paragraph
   \vspace{\stretch{3}} % space at bottom is three times that at the top
   \clearpage           % finish off the page
\end{fullwidth}  }
 \begin{dedication} %\leftskip=10cm 
\raggedleft\textit{
Vedrò con mio diletto~~~~~~~~~~~~~~~~~~~~\\ 
L'alma dell'alma mia~~~~~~~~~~~~~~~~~~~~\\ %, dell'alma mia  \\
Il core del mio cor~~~~~~~~~~~~~~~~~~~~\\
Pien di contento~~~~~~~~~~~~~~~~~~~~\\ %, pien di contento  \\
Vedrò con mio diletto~~~~~~~~~~~~~~~~~~~~\\
L'alma dell'alma mia~~~~~~~~~~~~~~~~~~~~\\ %, dell'alma mia \\
Il cor di questo cor~~~~~~~~~~~~~~~~~~~~\\
Pien di contento~~~~~~~~~~~~~~~~~~~~\\ %, pien di contento \\
E se dal caro oggetto~~~~~~~~~~~~~~~~~~~~\\
Lungi convien che sia~~~~~~~~~~~~~~~~~~~~\\ %, convien che sia \\
Sospirerò penando~~~~~~~~~~~~~~~~~~~~\\
Ogni momento~~~~~~~~~~~~~~~~~~~~\\
Vedrò con mio diletto~~~~~~~~~~~~~~~~~~~~\\
L'alma dell'alma mia~~~~~~~~~~~~~~~~~~~~\\ %, dell'alma mia \\
Il core del mio cor~~~~~~~~~~~~~~~~~~~~\\
Pien di contento~~~~~~~~~~~~~~~~~~~~\\ %, pien di contento \\
Vedrò con mio diletto~~~~~~~~~~~~~~~~~~~~\\
L'alma dell'alma mia~~~~~~~~~~~~~~~~~~~~\\ %, dell'alma mia \\
Il cor di questo cor~~~~~~~~~~~~~~~~~~~~ \\
Pien di contento~~~~~~~~~~~~~~~~~~~~ \\ %, pien di contento \\
-- Anastasio~~~~~~~~~~~~~~~~~~~~ \\
%\hspace{11cm}
}
% \textit{On vise le Porsche Panamera et dire aux proches que ça ira.} \\
%\hspace{14cm}-- \textsc{Maes}
\end{dedication}

\cleardoublepage
 %\vspace{-.5cm}
\chapter*{Acknowledgements} \addcontentsline{toc}{chapter}{Acknowledgements}\markboth{Acknowledgements}{}
 %\vspace{-.5cm}
Spending 4 years to study the dynamics of cholera (and \textsc{covid}-19) was a surprisingly good decision, as my Ph.D. has been a challenging but incredibly rewarding journey, in no small part thanks to amazing colleagues. For this, I am grateful to Prof. Andrea Rinaldo for warmly welcoming me to science and academia, and for allowing me to explore a subject that I grew to love. Thank you for encouraging me to pursue my research interests while benefiting from your unconditional support. % It made for a great inspiration and I'll take with me the rules of the ECHO lab.
 \marginnote{
  \begin{center}\textsc{Standing on the shoulder of giants}\end{center}
 During this thesis, I have been fortunate to participate to the GTFCC annual meeting, IDDconf, and SISMID. These conferences were an incredible learning experience, and I would like to thank the infectious disease modeling community for being so welcoming, inclusive, and patient. I learned and still learn every day from many of you. 
 
 This thesis is built on many datasets, from rainfall to reported cholera cases. I am very grateful to the workers all along the chain from collection to curation who have made my job possible.

 Thanks to the reviewers, co-reviewers, and co-authors from whom I have learned general scientific concepts and valuable scientific writing tips.
 
 Tools shape the way you think about problems. This thesis was not only made possible, but also sculpted by the languages and libraries that were used. I want to acknowledge all the open-source maintainers and contributors who tirelessly make, document, and maintain powerful tools, enabling everybody to use cutting-edge methods and to fill bug-reports.}
 My most sincere thanks go to Damiano Pasetto for deeply caring about every aspect my Ph.D. experience and for pushing or carrying me through these years. It has been wonderful to evolve under your supervision.
Scientifically, this thesis has benefited enormously from Javier Perez-Saez. Thank you for your insights on inference and life in general.
 Thank you Mario Zanon for making learning complicated optimal control algorithms a pleasure.
Many thanks to Jacques Fellay and Andrew Azman for your scientific insights and for introducing me to new institutions and collaborators.
  
 Upon my arrival at the ECHO lab in 2017, I've been welcomed by wonderful friends and colleagues who have made this journey fun. Thanks Silvia for the best tea cookie ever; Giezi for caring so much about the lab and organizing our pasta thursdays; and Luca because sports and meta-science are meant to go together. I wish to thanks Mitra for sharing delicious Iranian delicacies and an unique perspective on life; Paolo for the support at all time and spontaneous lunch discussions and so much more; Filippo for teaching me experimental science and finally Cristiano: I am truly grateful to have you as an office mate, and it did changed my last year at EPFL.
 
  From my intense and abbreviated stay in Baltimore, I would like to thank Prof. Justin Lessler for welcoming me and for trusting me on the scenario pipeline project. Thanks you Elizabeth, because wine and sports mix well with science. Thanks, Hannah, Kyu, Shaun, Steve, and Quifang for the warm welcome in Baltimore. Thanks to the COVID Scenario Pipeline team, working at night is much more pleasant with you all; especially to Josh for long hours pair debugging sessions.
 
 Merci à mes ami·e·s, et à ma famille pour votre soutien et tous les moments de joie; les cafés en EL et les longues discussions à toute heure.

Merci à mes parents pour m'avoir ouvert au monde et m'avoir encouragé et soutenu au cours de cette thèse.
À Céline, Eugène et Oliver, merci évidemment pour tout; sans votre folie cela n'aurait pas été pareil.

Finalement, merci Marion ! Merci d'avoir traversé avec moi toutes ces épreuves, d'avoir donné un sens à tout ça.%, et de l'avoir rendu  serein et doux. 
 Merci pour tout le reste évidemment.
 
 \chapter*{Summary} \addcontentsline{toc}{chapter}{Summary}\markboth{Summary}{}%research problem and objective
  \marginnote{\paragraph{Keywords} cholera, \textsc{covid}-19, epidemiology, public health, ecohydrology, infectious disease dynamics, SIR, mathematical modeling, statistical inference, optimal control.}
\vspace{-.5cm}
Emerging and existing infectious diseases pose a constant threat to individuals and communities across the world. In many cases, the burden of these diseases is preventable through public health interventions. However, taking the right decisions and designing effective policies is an intricate task: infectious disease epidemics are complex phenomena resulting from the interaction between pathogens, individuals, the environment, and societies. Modeling offers a principled way to reason about infectious disease dynamics from scarce and biased information and to guide decision-makers towards effective policies. 

This thesis tackles selected topics in cholera and \textsc{covid}-19 modeling towards informed public-health decisions. These two contrasting diseases were associated by a twist of fate, but also through the lens of a common modeling approach.  Compartmental, SIR-based, models are conditioned on the available evidence using computer-age statistical inference frameworks. A set of five models is developed, each tackling a different facet of the spread and control of these two infectious diseases. Each model aims at answering questions related to either the understanding of the mechanisms behind disease transmission, the projection of the future dynamics under different scenarios, or the assessment of the effectiveness of past interventions. Moreover, a novel application of epidemiological models to the formal design of control policies is proposed. Optimal control provides a rigorous framework to identify the most effective control measures under a set of operational constraints, providing a benchmark on what is possible to achieve with the available resources.

%Your key results
The results presented in this thesis range from scientific insight on the relationship between cholera and rainfall in Juba, South Sudan to the COVID Scenario Pipeline which produces reports used to inform the response to the \textsc{covid}-19 pandemic of different governmental entities. Furthermore, the effectiveness of the non-pharmaceutical interventions against \textsc{covid}-19 in Switzerland is evaluated; and so is the probability of eliminating cholera from Haiti under different scenarios of mass vaccination campaigns. Finally, the development of an optimal control framework towards the effective spatial allocation of vaccines against SARS-CoV-2 in Italy closes this conversation of models.

%Your conclusion
The present thesis demonstrates how infectious disease modeling enables informed decision-making by projecting the uncertainties under the light of the available evidence. It also highlights the effort needed to tailor the models and inference methods to the specificities of the transmission setting and the research question considered. From insights on transmission pathways to weekly reports aimed at decision-makers, it explores different applications of infectious disease modeling. Methods developed along the way may contribute to the toolbox of modelers, to guide policy decisions further towards a reduction of the burden of infectious diseases on communities.

%\enlargthispage[2\baselineskip]
 \vspace{-.5cm}
 \chapter*{Résumé} \addcontentsline{toc}{chapter}{Résumé}\markboth{Résumé}{}
   \marginnote{\paragraph{Mot-clés} cholera, \textsc{covid}-19, epidemiologie, santé publique, ecohydrologie, dynamique des maladies infectieuses, SIR, modelisation mathématique, inférence statistique, control optimal.}
\vspace{-.5cm}
 Les maladies infectieuses émergentes et existantes constituent une menace vive pour les individus et les communautés du monde entier. Dans de nombreux cas, la charge de ces maladies peut être évitée grâce à des interventions de santé publique. Cependant, prendre les bonnes décisions et concevoir des politiques efficaces est une tâche complexe : les épidémies de maladies infectieuses sont des phénomènes complexes résultant de l'interaction entre les agents pathogènes, les individus, l'environnement et les sociétés. En outre, seules des informations rares et partiales sont disponibles. La modélisation offre un moyen raisonné de raisonner sur la dynamique des maladies infectieuses et de guider les décideurs vers des politiques efficaces. 
 \marginnote{Il s'agit d'une traduction automatique en attendant la validation du résumé en anglais, qui fait donc foi pour ce brouillon.}

Cette thèse aborde des sujets choisis dans la modélisation du choléra et du \textsc{covid}-19 en vue de prendre des décisions éclairées en matière de santé publique. Ces deux maladies contrastées ont été associées par un coup du sort, mais aussi par le biais d'une approche de modélisation commune.  Des modèles compartimentaux, basés sur le SIR, sont conditionnés par les preuves disponibles en utilisant le cadre d'inférence statistique de l'ère informatique. Un ensemble de cinq modèles est développé, chacun abordant une facette différente de la propagation et du contrôle des maladies infectieuses. Chaque modèle vise à répondre à des questions liées à la compréhension des mécanismes de transmission des maladies, à la projection des dynamiques futures selon différents scénarios ou à l'évaluation de l'efficacité des interventions passées. En outre, une nouvelle application des modèles épidémiologiques à la conception formelle des politiques de contrôle est proposée. Le contrôle optimal fournit un cadre rigoureux pour identifier les mesures de contrôle les plus efficaces sous un ensemble de contraintes opérationnelles, fournissant un point de référence sur ce qu'il est possible de réaliser avec les ressources disponibles.

Les résultats présentés dans cette thèse vont de la compréhension scientifique de la relation entre le choléra et les précipitations à Juba, au Sud-Soudan, à la COVID Scenario Pipeline qui produit des rapports utilisés pour informer la réponse à la pandémie de \textsc{covid}-19 de différentes entités gouvernementales. En outre, l'efficacité des interventions non pharmaceutiques contre le \textsc{covid}-19 en Suisse est évaluée, de même que la probabilité d'éliminer le choléra en Haïti selon différents scénarios de campagne de vaccination de masse. Enfin, le développement d'un cadre de contrôle optimal pour l'allocation spatiale efficace du vaccin contre le SARS-CoV-2 en Italie clôt cette conversation entre modèles.

La présente thèse démontre comment la modélisation des maladies infectieuses permet une prise de décision éclairée en projetant les incertitudes à la lumière des preuves disponibles. Elle souligne également l'effort nécessaire pour adapter les modèles et les méthodes d'inférence aux spécificités du contexte de transmission et de la question de recherche considérée. De la compréhension des voies de transmission aux rapports hebdomadaires destinés aux décideurs, il explore les différentes applications de la modélisation des maladies infectieuses. Les méthodes développées en cours de route peuvent contribuer à la boîte à outils des modélisateurs, afin d'orienter les décisions politiques vers une réduction de la charge des maladies infectieuses sur les communautés.


 %\pdfbookmark[section]{\contentsname}{toc}
%\begin{fullwidth}\tableofcontents\listoffigures\listoftables\end{fullwidth}