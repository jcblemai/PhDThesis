\begin{fullwidth}
\maketitle
\blankpage
\end{fullwidth}
% r.7 dedication
%\cleardoublepage
%~\vfill
%\begin{doublespace}
%\noindent\fontsize{18}{22}\selectfont\itshape
%\nohyphenation
%Dedicated to those who appreciate \LaTeX{} 
%and the work of \mbox{Edward R.~Tufte} 
%and \mbox{Donald E.~Knuth}.
%\end{doublespace}
%\vfill
%\vfill
%\end{fullwidth}
% r.9 introduction
%\cleardoublepage
%\begin{fullwidth}

\newenvironment{dedication}
  {\clearpage           % w- want a new page
   \thispagestyle{empty}% no header and footer
   \vspace*{\stretch{2}}% some space at the top 
   %\itshape             % the text is in italics
  
\leftskip=10cm
\raggedleft
\parindent=0pt
\begin{fullwidth}
  }
  {\par % end the paragraph
   \vspace{\stretch{3}} % space at bottom is three times that at the top
   \clearpage           % finish off the page
\end{fullwidth}  }
 \begin{dedication} %\leftskip=10cm 
\raggedleft\textit{
Vedrò con mio diletto~~~~~~~~~~~~~~~~~~~~\\ 
L'alma dell'alma mia~~~~~~~~~~~~~~~~~~~~\\ %, dell'alma mia  \\
Il core del mio cor~~~~~~~~~~~~~~~~~~~~\\
Pien di contento~~~~~~~~~~~~~~~~~~~~\\ %, pien di contento  \\
Vedrò con mio diletto~~~~~~~~~~~~~~~~~~~~\\
L'alma dell'alma mia~~~~~~~~~~~~~~~~~~~~\\ %, dell'alma mia \\
Il cor di questo cor~~~~~~~~~~~~~~~~~~~~\\
Pien di contento~~~~~~~~~~~~~~~~~~~~\\ %, pien di contento \\
E se dal caro oggetto~~~~~~~~~~~~~~~~~~~~\\
Lungi convien che sia~~~~~~~~~~~~~~~~~~~~\\ %, convien che sia \\
Sospirerò penando~~~~~~~~~~~~~~~~~~~~\\
Ogni momento~~~~~~~~~~~~~~~~~~~~\\
Vedrò con mio diletto~~~~~~~~~~~~~~~~~~~~\\
L'alma dell'alma mia~~~~~~~~~~~~~~~~~~~~\\ %, dell'alma mia \\
Il core del mio cor~~~~~~~~~~~~~~~~~~~~\\
Pien di contento~~~~~~~~~~~~~~~~~~~~\\ %, pien di contento \\
Vedrò con mio diletto~~~~~~~~~~~~~~~~~~~~\\
L'alma dell'alma mia~~~~~~~~~~~~~~~~~~~~\\ %, dell'alma mia \\
Il cor di questo cor~~~~~~~~~~~~~~~~~~~~ \\
Pien di contento~~~~~~~~~~~~~~~~~~~~ \\ %, pien di contento \\
-- Anastasio~~~~~~~~~~~~~~~~~~~~ \\
%\hspace{11cm}
}
% \textit{On vise le Porsche Panamera et dire aux proches que ça ira.} \\
%\hspace{14cm}-- \textsc{Maes}
\end{dedication}

\cleardoublepage

\chapter*{Acknowledgements} \addcontentsline{toc}{chapter}{Acknowledgements}\markboth{Acknowledgements}{}
Spending 4 years to study the dynamics of cholera (and \textsc{covid}-19) was in retrospect a surprisingly good decision, as my PhD has been a challenging but incredibly rewarding journey. I am grateful to Prof. Andrea Rinaldo for warmly welcoming me in science and academia, and for giving me the opportunity of exploring a new subject that I grew to love. Thank you for giving me the freedom to pursue my interests while benefiting from your unconditional support. It made for a great inspiration and I'll take with me the rule of the ECHO lab.
 \marginnote{
  \begin{center}\textsc{Standing on the shoulder of giants}\end{center}
 During this thesis, I have been fortunate to participate to the GTFCC annual meeting, to IDDConf and to SISMID. These conference were incredible experience, and I would like to thank the infectious disease modeling community for being so welcoming, inclusive and patient. I learned and still learn from many of you. 
 This thesis is built on many datasets, from rainfall to reported cholera cases. I am very grateful to the workers all along the chain from collection to curation who have made my job possible.
 Thanks to reviewers, co-reviewers and co-authors from whom I have learned general concepts and valuable scientific writing tips.
 Tools shape the way you think about problems. This thesis was made possible, but also sculpted by the languages and libraries that were used. I want to acknowledge all the open-source maintainers and contributors who tirelessly make, document and maintain powerful tools, rendering difficult easy and enabling everybody to use.}
 My sincere thanks goes to Damiano Pasetto for deeply caring about my PhD experience, at time carrying me through this thesis. I own you this thesis %I fondly remember our late evening spend the office, and for your critical eye, and . 
Scientifically, this thesis has benefited enormously from Javier Perez-Saez. Thank you for your insights on inference and life in general, I have learnt so much.
 Thanks you Mario Zanon for making the learning a new complicated algorithms is a pleasuret hem to you%, , and thankfully complicated math were followed by discussion and there nothing 
 
 Upon my arrival at the ECHO lab in 2017, I've been welcomed by wonderful people who have made this journey fun. Thanks Silvia for the best tea cookie ever, Giezi for caring so much about the lab and organizing our pastas and Luca, because sports and meta-science talks are meant to go together. From my later years at ECHO I with to thanks Mitra for many discoveries and the unique perspective on life on work; Paolo for and Cristiano: I am truly grateful to have you as an office mate
 
 I want 
 
 and thanks Jacques
 From my intense and abbreviated stay in Baltimore, I would like to thank Prof. Justin Lessler for welcoming me and for trusting me after meeting me two times ... Elizabeth, Hannah, Kyu, Shaun have been welcoming. Thanks to the \textsc{covid} Scenario pipeline team, especially to Josh.  Yeah, some people still uses emacs.
 
 --- 
 
 Merci à mes ami·e·s

À famille, papa, maman, pour m'avoir ouvert au monde et changer le regard sur les choses.

À Céline, Eugène et Oliver, merci pour toujours près de mon coeur, 

Finalement, merci Marion ! Merci de m'avoir soutenue tout au long de ce doctorat, d'y avoir donner un sens et d'avoir rendu  serein et doux. 
 et d'avoir traverser avec moi toutes ces ép.  
 
 \chapter*{Summary} \addcontentsline{toc}{chapter}{Summary}\markboth{Summary}{}%research problem and objective
  \marginnote{\paragraph{Keywords} cholera, \textsc{covid}-19, epidemiology, public health, ecohydrology, infectious disease dynamics, SIR, mathematical modeling, statistical inference, optimal control.}
Emerging and existing infectious diseases pose a lively threat on individuals and communities across the world. In many cases, the burden of these diseases is preventable through public-health interventions. However, taking the right decisions and designing effective policies is an intricate task: infectious disease epidemics are complex phenomena resulting from the interaction between pathogens, individuals, environment and societies. Moreover, only scarce and biased information is available. Modeling offers a principled way to reason infectious disease dynamics and to guide decision-makers towards effective policies. 

This thesis tackles selected topics in cholera and \textsc{covid}-19 modeling towards informed public-health decisions. These two contrasting diseases were associated by a twist of fate, but also through the lens of a common modeling approach based on compartmental, SIR-based models that are conditioned on the available evidence using computer-age statistical inference framework. A set of five models is proposed, each tackling a different facet of the spread and control of infectious diseases. 
%Your key results
The results presented in this thesis range from scientific insight on the relationship between cholera and rainfall in Juba, South Sudan to the COVID Scenario Pipeline which produces reports used to inform the response to the \textsc{covid}-19 pandemic of different governmental entities. Along this road the effectiveness of the non-pharmaceutical interventions against \textsc{covid}-19 in Switzerland is evaluated, along with the probability of eliminating cholera from Haiti under different scenarios of mass vaccination campaign. Finally the development of an optimal control framework, where algorithms designs the most effective spatial allocation of vaccine against  SARS-CoV-2 in Italy
%Your conclusion
This thesis shows how infectious disease modeling allow to guide decisions by projecting uncertainties under the light of the available evidence, and highlight the need to tailor model structure and methods to the transmissions setting and the research question. By providing report to decision maker, it bridges the gap between science and public health; highlighting how infectious disease model guide policy makers towards a reduction of the burden on communities.

Successive questions related to the understanding of the mechanisms behind disease transmission, to the projection of future incidence under different scenarios of control measures, and the assessment the effectiveness of control measures are answered using models. A novel application of disease dynamics model is also proposed: the formal design of control policies. Optimal control provides a rigorous framework to identify the most effective control measures under a set of operational constraints, providing a benchmark on what is possible to achieve with available resources


%\enlargthispage[2\baselineskip]
 
 \chapter*{Résumé} \addcontentsline{toc}{chapter}{Résumé}\markboth{Résumé}{}
 
 %\pdfbookmark[section]{\contentsname}{toc}
%\begin{fullwidth}\tableofcontents\listoffigures\listoftables\end{fullwidth}