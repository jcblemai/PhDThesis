\begin{fullwidth}
\maketitle
\blankpage
\end{fullwidth}
% r.7 dedication
%\cleardoublepage
%~\vfill
%\begin{doublespace}
%\noindent\fontsize{18}{22}\selectfont\itshape
%\nohyphenation
%Dedicated to those who appreciate \LaTeX{} 
%and the work of \mbox{Edward R.~Tufte} 
%and \mbox{Donald E.~Knuth}.
%\end{doublespace}
%\vfill
%\vfill
%\end{fullwidth}
% r.9 introduction
%\cleardoublepage
%\begin{fullwidth}

\newenvironment{dedication}
  {\clearpage           % w- want a new page
   \thispagestyle{empty}% no header and footer
   \vspace*{\stretch{2}}% some space at the top 
   %\itshape             % the text is in italics
  
\leftskip=10cm
\raggedleft
\parindent=0pt
\begin{fullwidth}
  }
  {\par % end the paragraph
   \vspace{\stretch{3}} % space at bottom is three times that at the top
   \clearpage           % finish off the page
\end{fullwidth}  }
 \begin{dedication} %\leftskip=10cm 
\raggedleft\textit{
Vedrò con mio diletto~~~~~~~~~~~~~~~~~~~~\\ 
L'alma dell'alma mia~~~~~~~~~~~~~~~~~~~~\\ %, dell'alma mia  \\
Il core del mio cor~~~~~~~~~~~~~~~~~~~~\\
Pien di contento~~~~~~~~~~~~~~~~~~~~\\ %, pien di contento  \\
Vedrò con mio diletto~~~~~~~~~~~~~~~~~~~~\\
L'alma dell'alma mia~~~~~~~~~~~~~~~~~~~~\\ %, dell'alma mia \\
Il cor di questo cor~~~~~~~~~~~~~~~~~~~~\\
Pien di contento~~~~~~~~~~~~~~~~~~~~\\ %, pien di contento \\
E se dal caro oggetto~~~~~~~~~~~~~~~~~~~~\\
Lungi convien che sia~~~~~~~~~~~~~~~~~~~~\\ %, convien che sia \\
Sospirerò penando~~~~~~~~~~~~~~~~~~~~\\
Ogni momento~~~~~~~~~~~~~~~~~~~~\\
Vedrò con mio diletto~~~~~~~~~~~~~~~~~~~~\\
L'alma dell'alma mia~~~~~~~~~~~~~~~~~~~~\\ %, dell'alma mia \\
Il core del mio cor~~~~~~~~~~~~~~~~~~~~\\
Pien di contento~~~~~~~~~~~~~~~~~~~~\\ %, pien di contento \\
Vedrò con mio diletto~~~~~~~~~~~~~~~~~~~~\\
L'alma dell'alma mia~~~~~~~~~~~~~~~~~~~~\\ %, dell'alma mia \\
Il cor di questo cor~~~~~~~~~~~~~~~~~~~~ \\
Pien di contento~~~~~~~~~~~~~~~~~~~~ \\ %, pien di contento \\
-- Anastasio~~~~~~~~~~~~~~~~~~~~ \\
%\hspace{11cm}
}
% \textit{On vise le Porsche Panamera et dire aux proches que ça ira.} \\
%\hspace{14cm}-- \textsc{Maes}
\end{dedication}

\cleardoublepage

\chapter*{Acknowledgements} \addcontentsline{toc}{chapter}{Acknowledgements}\markboth{Acknowledgements}{}
The past 4 years spent studying infectious disease modeling were an incredible  learning experience. I wish to thanks Andrea Rinaldo first for welcoming me in science and for the opportunity of exploring a new subject. But also for showing me the importance of a personal contribution.  and for the freedom and trust support; the rules of the ECHO's lab applies through
 
 My sincere thanks to Damiano Pasetto for deeply caring about  making me stay employed and sane, for making this enjoyable, for all the good time spend together, and for your critical eye, and carrying me through this thesis I own you this thesis.
 
 Scientifically, this thesis has benefited enormously from discussion and projects with Javier Perez-Saez. Thank you for your insights on life  I have learnt so much and I am grateful
 Learning a new technolgy was a pleasure, and thanks Jacques
 
 Mario Jacques
 
 Upon my arrival at the ECHO lab in 2017, I've been welcomed by wonderful people who have made this journey fun. Thanks Silvia for the best tea cookie , Giezi and Luca, because sports and meta-science talks are meant to be done near the lake. And towards my later years Mitra, Cristiano and Paolo.
 
 From my intense and abbreviated stay in Baltimore, I would like to thank Justin Lessler for welcoming me and for trusting me after meeting me two times ... Elizabeth, Hannah, Kyu, Shaun have been welcoming. Thanks to the \textsc{covid} Scenario pipeline team, especially to Joshua Kamiski. Yeah, some people still uses emacs.
 
\marginnote{I have been fortunate to participate to GTFCC, to IDDConf 2019 and to SISMID 2019, thanks for the infectious disease modeling and connected public health communities have been so welcoming, greatly explaining. I learned and still learn from many. Standing on the shoulder of giants, 
 
 This thesis makes use of many datasets from precipitation to reported cholera cases. I am very grateful to the workers all along the chain from collection to curation who have made my job possible.
 
 Finally, thanks to reviewers, co-reviewers and co-authors from whom I have learned general concepts and valuable scientific writing tips.
 
 Tools shape the way you think about problems. This thesis was made possible, but also sculpted by the languages and libraries I used. I want to acknowledge all the open-source maintainers and contributors who tirelessly make, document and maintain powerful tools, making the difficult easy and enabling everybody to use.
 }

À famille, papa, maman, pour m'avoir ouvert au monde et changer le regard sur les choses.

À Céline, Eugène et Oliver, toujours près de mon coeur.

Finalement, Marion ! Merci de m'avoir soutenue tout au long de cette incroyable voyage. Mais aussi d'avoir rendu cela serein et doux. 
 
 \chapter*{Summary} \addcontentsline{toc}{chapter}{Summary}\markboth{Summary}{}
 research problem and objectives

Public-health interventions strive to spare individuals and communities from the burden in infectious disease poses on humanity. Designing effective policies is an intricate task. Infectious disease epidemics are complex phenomena that results from the interaction between pathogens, individuals, environment and societies, and only scarce and biased information is available.  Modeling offers a principled way to deal with the available evidence to reason infectious disease dynamics, which in turn allow for informed control decisions.



compartmental models, diversity of models.

The present thesis tackles selected infectious disease modeling topics with the aim of helping decision-makers to design effective policies. 

%Your methods
A set of five modeling st of infectious disease transmission is proposed, each covering a different facet of the spread and control of disease.
Two mInitially focused on cholera, one of humanity's earliest disease, still present in many part of the world,  Model are used as representation of reality to 

The emergence of covid-19 in 2019 diverted the initial research plan focused on cholera, and we present modeling work , bridging the gap between science.


The final application of these model is within an optimal control framework, where algorithms designs the most effective SARS-CoV-2 vaccine allocation strategy




%Your key results
Results highlight 


%Your conclusion
This thesis Infectious disease modeling is a necessary, and the diversity 
 
\paragraph{Keywords} cholera, \textsc{covid}-19, infectious disease modeling,
epidemiology, optimal control, public health, inference
 
 
 \chapter*{Résumé} \addcontentsline{toc}{chapter}{Résumé}\markboth{Résumé}{}
 
 %\pdfbookmark[section]{\contentsname}{toc}
%\begin{fullwidth}\tableofcontents\listoffigures\listoftables\end{fullwidth}