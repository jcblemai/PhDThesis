\chapter*{Conclusion and Perspectives}
\addcontentsline{toc}{chapter}{Conclusion and Perspectives}
\markboth{Conclusion and Perspectives}{}
Each of the five chapters of this thesis present an infectious disease model tailored to tackle a public-health policy related question, whether related to infectious disease transmission pathways or to the effects of control policies. Within the four stage framework defined in the \textsc{Introduction} (fig. \ref{fig:modeling}), different facets of infectious disease modeling are exploited to approach the complex phenomenas that are cholera and \textsc{covid}-19 epidemics. Let's go through this conversation of models one more time.

First, as in classical hypothesis testing, a model might be considered as a simplified representation of reality. Among different hypothesis, the aim is to find the ``true'' model underlying the observed dynamics. This perspective is taken in \textsc{Chapter 2}, where the explanatory power of different pathways for rainfall-mediated cholera transmission are compared. Results stress the importance of rainfall as a covariate for cholera transmission while highlighting the complexity of the mechanistic pathways considered. While context-dependent, it is nevertheless interesting to observe how different transmission routes proposed in the litterature fare on intra-seasonal rainfall events in Juba, South Sudan.

In \textsc{Chapters 3} and 4, it is seeked predictive accurary with respect to interventions and transmission of interest; models that most adequately reproduce an unknown, highly complex reality in order to perform experiments are designed\footnote{Indeed, accurate mechanistic modeling is still required for the processes of interest and these two perspectives are more related than opposed}. As a part of multi-modeling study, a spatial stochastic model of cholera transmission in Haiti is proposed in \textsc{Chapter~3}. Building on ECHO's experience on cholera modeling, it is used to assess the probability of cholera elimination from Haiti under different scenarios of mass vaccination campaigns. A restrospective analysis reveals that while the proposed model fits past dynamics, it under-estimates the probability of elimination of cholera in Haiti. This result recalls that modeling is not a silver bullet and stresses the importance of careful model design and that projections uncertainties communication includes modeling assumptions in addition to modeled uncertainty.

Nonwithstanding the \textsc{covid}-19 pandemic, cholera would have been the sole focus of the present thesis. This interuption leave rhe next question unanswered  that derives from these results. There are no confirmed cholera cases since early 2019 in Haiti, which is surprising with regard to the presented modeling work. What is the cause of the extinction of cholera from Haiti ? To what extent did climate, herd immunity and the WaSH interventions carried out in the past year played a role ? Hopefuly the answer will provide additional insights on the path towards elimination of the cholera by 2030, as sought after by WHO and GTFCC. Another open-problem is an open allocation of the cholera vaccine stockpile where demand by countries, and at risk population, vastly exceed supply and production\cite{Pezzoli:GlobalOralCholera:2019}. 

Most of the work undertaken for the response to \textsc{covid}-19 pandemic took place within or around the COVID Scenario Pipeline, a configurable framework to projects epidemic trajectories and healthcare impacts under different suites of interventions. The pipeline is used to support several partners including the state of California and the national US response with report to inform and guide decisions. It is still being actively developed to address the ever-changing need of decision-makers. Since the description given in \textsc{Chapter~4}, a year of historical data brought the need for high-dimensional inference algorithms, and the challenges posed by SARS-CoV-2 have brough it the need for flexible disease transmission and health outcome modeling to capture the dynamics cause by competing strains, immune escape and the different vaccination campaigns. The pipeline is an operational forecasting platform, a goal of the initial research plan, providing a unified framework to project and forecast dynamics from disease emergence to endemicity\footnote{Updated outputs of the pipeline are visible on the \textsc{covid}-19 Scenario Modeling Hub and the \textsc{covid}-19 Forecast Hub (\url{covid19scenariomodelinghub.org} and \url{covid19forecasthub.org}).}.

The assessement of past policies effectiveness is important to project dynamics and to inform decisions. In Switzerland, the pipeline has been used to inform CHUV, Canton de Vaud main hospital. The dialogue with policy makers trigered the exchange of a dataset of lenght of stays in hospital to improve scenario planning report accuracy. In turn, this data has enabled a research study on the estimation of SARS-CoV-2 reproduction number, $R_0$, in Switzerland. Using stochastic models and iterated filtering, a procedure different from most other $R_0$ estimation methods, it uncovers the effectiveness of the non-pharmaceutical interventions in different cantons. Among the numerous takeways from this early \textsc{covid}-19 work presented in \textsc{Chapter 5}, the timing of the decrease in transmission preceeding NPIs implementation is especially interesting. The feedback loop was closed as these estimates were used as assumptions in subsequent modeling reports for Canton de Vaud.


Finally, provided an accurate model of transmission dynamics and interventions effectiveness, an objective and descriptions of operational constraints, optimal control is the ultimate stage of infectious disease modeling towards informed decisions: policies that minimize the burden of a disease are programatically designed. To date both the demanding prerequistes and the difficulty of controling epidemiological models at scales that are useful were hindered the progress on this topic. The optimal control framework is presented in \textsc{Chapter 6} is  Using automatic differenciation and non-linear programing, the proposed solver design the most effective vaccination strategy for a given objective, under operational constraint. A proof of concept is done on vaccination against \textsc{covid}-19 in Italy. While limitations in the model, such as the abscence of age-stratification, limit the scope of the presented results, it is as the first country scale optimization of a comparmental model  a signicant contribution towards making these novel algorithms a tool agaisnt infections disease.

% These models were developed along the same cycle: model \textit{design}$\rightarrow$ model \textit{fit}$\rightarrow$model \textit{adaptation}$\rightarrow$model \textit{insight} to answer scientific and public-health questions. Computer-age statistical inference at the service of infectious disease modeling help to reason about the complex dynamics of epidemics. 


Overall, this thesis is a testament of the relevance of compartmental models as a tools to inform public health decisions, and to reason about infectious disease transmission, providing scientific insights on the underlying processes and the effectiveness of past and future public-health policies. % computer-age

  These two chapters were interesting as after careful model design, results are surprising. More often than not due to an error in the model specification, in this case the investigation enable the discovery of important processes. The studied system is complex and its many interactions and models are an invitation to explore in-depth our belief and assumptions. But sometime model are surprising due to reality being supprising. Or depressing such as the first \textsc{covid}-19 transmission models. 
   In this case careful communication and. Models presented in this thesis have either guided directly decision makers



Yet, the presented research works prove that there is no one-size-fit-all approach to infectious disease modeling, and each research question and transmissions settings requires numerous adjustment to capture and project the dynamics of interests. An unified framework is not possible, and the diversity in approaches is important to tackle.  And even on the same The diversity of sensibilities and background in practitioners bring complementary viewpoint.
  This takeaway might seems discouraging if it is not highlighted that it is not necessary to start from scratch: each model builds on the previous works borrowing conceptuals breakthrough which after some work become tools (taken here in a broad sense encompassing conceptual approaches to software package), developed spefically to be re-used accross settings.
  
%The importance of  These compartmental models share the same backbone and many characteristics, but different features of disease transmission are emphasized to tailor these approach to the research question at stake. 

  Often outbreaks and pandemics have sparked advances on different aspects of infectious disease modeling. The research and public-health communities design new methods and tools leap forward. These advances remains available for endemic diseases and futur pandemic. It improves the preparedness against the emergence of new pathogens, and the response to the \textsc{covid}-19 pandemic has benefitted extensively from conceptual and concrete tools developed from past pandemcis.\footnote{The $R_0$ package and thouroughly used for the \textsc{covid}-19 pandemics was developed after the H1N1 pandemics\fullcite{Obadia:R0PackageToolbox:2012}, comes to mind. Among other topics there are \eg real-time modelisation for Ebola 2014--2015, one-health approach for cholera, multi-modeling studies for influenza, and Zika.}. 
  \marginnote{Apart from \textsc{Chapter 2}, code and data to reproduce these projects are available. However, quality documentation and instructruction are lacking.}
  In addition to the research studies results and conclusions by themselves, it is possible to envision as tools three chapters of the thesis. Arguably, the COVID ScenarioPipeline is already one: it has been used across different settings by different entities. It has proven robust, and recent developement improved its flexibility to allow for other diseases to be modeled. Then in The method to estimate $R_0$, while not novel, is different from most of the current method based either on deconvolutions (backward-forward viewpoint) or parametrized model. While it is involved and necessiate good data and representative assumptions, it works well and provide an alternative look on transmission.
Finally, the optimal control framework allows to explore a model and a disease, from another angle. It is interesting, and slightly dangerous to let the the computer handle the choices to control a complex high-dimenssional  phenomena, just giving it an objective and some constraints. The obtained solution make use of every feature of the model, and one must be careful on the interpretability, these algorithm allow for the design of effective interventions and the best allocation of control ressources. 

As infectious diseases pose a constant threat on children and adults accross the globe, tools and prepardenss to constrction effective interventions are needed. Towards this, improvement in the scientific understandfing of diseases transmissiion, but also the exploration of the facets of the complex system tthat link environement, individuals and societies. While modeling may mislead or render overconfident even in good hands, parcimony and careful inspection makes modeling a key instrument to fight diseases. A point that has been evident during the covid-19 pandemic. In fact, aside from scienfic curiosity lies the neccessity to control infectious disease. Despite tremondous advances, the data about infectious diseases will stay noisy, missing and biased in the foreseable futur. More so because oubreaks thrive where conflicts, natural disaster and instability lies. Mathematical and computational tools to reason about diseases, to infer unknown quantities are a keystone in the path towards elimination, which calls for improvements in every facet of disease control. Tools developed as part of theis thes to better arm in the fight agaisnt futur and exisiting disease. The \textsc{covid}-19 pandemic has exposed what is possible in term of mobilisation and collaboration from partners around the world. The sudden availablity of mobility datasets, something long desired for cholera, is an example of the paradigm shift brought by a pandemic that affected every country in the world. Hopefully, it will set a precedent to involve more partners on great collaborations to tackles other diseases that remains unfair to children and adults worldwide, such as cholera.
And despite limited efficacy, oral cholera vaccines remain a powerful tools, promising in many countries with endemic cholera. 
The course of this thesis has brough reports to decision makers and influenced decisions, and developed tools are a contribution the the arsenal of methods to do se. Towards the elimination of cholera.

