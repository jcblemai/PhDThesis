\chapter*{Conclusion and Perspectives}
\addcontentsline{toc}{chapter}{Conclusion and Perspectives}
\markboth{Conclusion and Perspectives}{}
Each of the five chapters presented in this thesis introduces an infectious disease model tailored to tackle a public-health policy related question, whether related to infectious disease transmission pathways or the effects of past or planned interventions. Each study takes a different path within the interactive process to model infectious diseases dynamics\footnote{see \textsc{Introduction}, fig. \ref{fig:modeling} and \fullcite{Heesterbeek:ModelingInfectiousDisease:2015}}. Naturally, the angle of attack to approach the complex phenomena that are cholera and \textsc{covid}-19 transmissions varies depending on the research question and many factors that set the stage for the study. Let's go through this conversation of models one more time.
%, among which \eg the data available, the potential collaboration with decision-makers. 

First, as in classical hypothesis testing, a model might be considered as a simplified representation of reality. Among different hypothesis, the aim is to find the ``true'' model underlying the observed dynamics. This perspective is taken in \textsc{Chapter~2}, where the explanatory power of different pathways for rainfall-mediated cholera transmission are compared\cite{Rinaldo:Reassessment20102011:2012, Eisenberg:ExaminingRainfallCholera:2013}. Results stress the importance of rainfall as a covariate for cholera transmission while highlighting the complexity of the mechanistic pathways considered. While context-dependent, it is nevertheless interesting to observe how the different transmission routes proposed in the literature fare on intra-seasonal rainfall events in Juba, South Sudan.

Policy related questions in \textsc{Chapters 3} and 4 change the viewpoint on models: predictive accuracy with respect to interventions and transmission of interest is desired. Models are designed to most adequately reproduce an unknown, highly complex reality in order to perform experiments\footnote{Indeed, accurate mechanistic modeling is still required for the processes of interest and these two perspectives related more than opposed.}. As a part of multi-modeling study\cite{Lee:AchievingCoordinatedNational:2020}, a spatial stochastic model of cholera transmission in Haiti is proposed in \textsc{Chapter~3}. Building on ECHO's experience on cholera modeling\cite{Rinaldo:RiverNetworksEcological:2020a}, it is used to assess the probability of cholera elimination from Haiti under different scenarios of mass vaccination campaigns. A retrospective analysis reveals that while the proposed model fits past dynamics, it under-estimates the probability of elimination of cholera in Haiti. This discrepancy  is a reminder that modeling is no silver bullet, further stressing the importance of careful model design aligned with the study objective. 

Non-withstanding the \textsc{covid}-19 pandemic, cholera would have been the sole focus of the present thesis. This regretful interruption left the following research questions on cholera unanswered. There has been no confirmed cholera cases since early 2019 in Haiti, which is surprising with regard to the aforementioned projections. What is the cause of the elimination of cholera from Haiti ? What role did climate, herd immunity and the WaSH interventions carried out in the past year played ? Hopefully the answer will provide additional insights on the path towards elimination of cholera by 2030, as sought after by WHO and GTFCC. The oral cholera vaccine stockpile put in place at the disposal of countries plays a key role in this strategy\cite[-2\baselineskip]{GlobalTaskForceonCholeraControl:EndingCholeraGlobal:2017}. But the population  at risk vastly exceed the vaccine supply and production\cite{Pezzoli:GlobalOralCholera:2019}, and effective targeting of the existing doses between countries remains an important open problem\cite{Lessler:MappingBurdenCholera:2018}.

The \textsc{covid-19} pandemic mobilized infectious disease epidemiologist across the world. Most of the work undertaken for the response to the \textsc{covid}-19 pandemic took place within or around the COVID Scenario Pipeline, a configurable framework to project epidemic trajectories and healthcare impacts under different suites of interventions. The pipeline is used to support several partners including the state of California and the national US response. It is still being actively developed to address the ever-changing need of decision-makers. Since the description given in \textsc{Chapter~4}, a year of historical data brought the need for high-dimensional inference algorithms, and the challenges posed by SARS-CoV-2 necessitated more flexible disease transmission and health outcome modeling. These evolutions allowed to capture the dynamics of competing strains, immune escape and the different vaccination campaigns\cite{Borchering:ModelingFutureCOVID19:2021}. The pipeline is an operational forecasting platform that provides a unified framework to project and forecast dynamics from emergence to endemicity\footnote{Updated outputs of the pipeline are visible on the \textsc{covid}-19 Scenario Modeling Hub and the \textsc{covid}-19 Forecast Hub (\url{covid19scenariomodelinghub.org} and \url{covid19forecasthub.org}).}.

Assessing the effectiveness of past policies is paramount to project disease dynamics and to inform future decisions. In Switzerland, the pipeline has been used to inform CHUV, the main hospital of Canton de Vaud. The dialogue with policy makers triggered the exchange of a dataset of length of stays in hospital to improve scenario planning report accuracy. In turn, this data has enabled a research study on the estimation of SARS-CoV-2 reproduction number, $R_0$, in Switzerland. Using stochastic models and iterated filtering -- an alternative procedure to the predominant $R_0$ estimation methods -- it uncovers the effectiveness of non-pharmaceutical interventions against \textsc{covid-19} in Switzerland. Among the numerous takeaways from this early \textsc{covid}-19 work presented in \textsc{Chapter~5}, the timing of the decrease in transmission preceding NPIs implementation is especially interesting. The feedback loop was closed when these estimates were used as assumptions in subsequent modeling reports for Canton de Vaud.

Finally, provided an accurate model of transmission dynamics and interventions effectiveness, optimal control is the ultimate stage of infectious disease modeling towards informed decisions: policies that minimize the burden of a disease are programmatically designed. To date both the demanding prerequisites and the difficulty of controlling epidemiological models at scales that are relevant to decision-makers were blocking to apply optimal control on practical problems. In \textsc{Chapter~6}, a large-scale optimal control framework coupled to an existing model of \textsc{covid}-19 transmission\cite{Gatto:SpreadDynamicsCOVID19:2020,Bertuzzo:GeographyCOVID19Spread:2020} is presented.  Using automatic differentiation and non-linear programing, the proposed solver design the most effective vaccination strategy for a given objective, feasible under operational constraints. This proof of concept is performed on spatial vaccines allocation against \textsc{covid}-19 in Italy. While limitations of the model, such as the absence of age-stratification, restrict the scope of the results, it is the first country scale optimization of a compartmental disease dynamic model % and hopefully a significant contribution towards making these novel algorithms tools against infections disease.


Taken as a whole, the present thesis demonstrates the relevance of infectious disease modeling as a tool to inform public-health decisions through five studies, each providing scientific insights on the underlying transmission processes or the effectiveness of past and future control policies. Yet, it also shows that there is no one-size-fit-all approach to infectious disease modeling: each research question and transmission settings requires numerous adjustments to capture and project the disease dynamics. The diversity of approaches and viewpoints\footnote{enhanced with the diversity of sensibilities and background of the practitioners and teams.} undertaken is important to tackle these problems. This takeaway might seems discouraging, but this enterprise never starts from scratch: each model builds on the previous works, borrowing conceptual breakthrough and methods. With these new tools (in a broad sense encompassing  mathematical approaches and software packages), the scientific and public-health communities are left better armed to tackle harder problems on endemic diseases and to face novel threats. Often outbreaks and pandemics have sparked advances on many aspects of infectious disease modeling: new methods are designed and tools leap forward. These advances remains available for endemic diseases and future pandemic. It improves the preparedness against the emergence of new pathogens, and the response to the \textsc{covid}-19 pandemic has benefitted extensively from conceptual and concrete tools developed from past pandemics\footnote{The $R_0$ package and thoroughly used for the \textsc{covid}-19 pandemics was developed after the H1N1 pandemics\fullcite{Obadia:R0PackageToolbox:2012}, comes to mind. Among other topics there are \eg real-time forecasting for Ebola 2014--2015, one-health approach for cholera, multi-modeling studies for influenza, and Zika.}. 
  \marginnote{Apart from \textsc{Chapter~2}, code and data to reproduce these projects are available. However, quality documentation and instruction are lacking.}
  
  In addition to the research studies results by themselves, it is possible to envision as tools three chapters of the thesis. Arguably, the COVID Scenario Pipeline is already one: it has been used by different entities across many settings, and has proven robust and flexible enough through the pandemic. The method used to estimate $R_0$ in Switzerland, while not novel, is different from most of current methods based either on deconvolution (backward or forward viewpoints) or functional form git model. While it is harder to use and necessitate some important assumptions, it provides a global look on time-varying transmission, which has proved useful for \textsc{covid}-19. % Its powerful uncertainties representation may be more even relevant in contexts where data is scarce. 
Finally, the optimal control framework allows one to explore a model and a disease, from another angle. The computer handle the choices to control a complex high-dimensional  phenomena, given an objective and constraints. The obtained solution make use of every feature of the model, and these algorithms allow for the design of effective interventions and the best allocation of control resources.


The sustaining transmission of cholera and \textsc{covid}-19 is the product of complex interactions between environment, individuals, pathogens and societies. Despite tremendous advances, the data about infectious diseases will stay noisy, missing and biased in the foreseeable future. Modeling helps to design effective policies, allowing to properly weight different course of action and to conveying the associated outcomes and uncertainties. As such, it has been a key instrument to in the fight against infectious diseases, a point made more evident by the \textsc{covid}-19 pandemic. Using compartmental models and computer-age statistical inference methods, the present thesis explored way to inform decision to control infectious disease epidemics. The scientific understanding of transmission pathways and intervention effectiveness has been unraveled in specific contexts. The continuous (and ongoing) issue of planning reports about \textsc{covid}-19 to decision-makers, has influenced public-health policies of different governments. Hopefully, the tools developed along this road, either at their infancy for optimal control or mature like the COVID Scenario Pipeline, will hopefully strengthen the arsenal of methods to guide decisions in deal with infectious disease.  In the future, novel data streams and rapidly evolving computer-age statistical learning methods promise, if coupled with practitioner care, more accurate models able to answer more complicated questions. 

How to protect

While the solutions to fight infectious diseases exists, resources are notoriously lacking, even more so because outbreaks thrive where conflicts, natural disaster and instability lies. Indeed, there is so much modeling can do, and the elephant in the room is the lack of resources to fight the infectious diseases pose a constant threat on children and adults across the globe. The \textsc{covid}-19 pandemic has exposed what is possible in term of mobilization and collaboration from partners around the world. Hopefully, it will set a precedent and other diseases that that plague impoverished and stigmatized communities around the world will also benefit from involvement and resources necessary to eliminate\footnote[][-5\baselineskip]{Even there, the sudden availability of mobility datasets, something long desired for cholera, is an example of the paradigm shift brought by a pandemic that affected every country in the world, and hopefully lessons learned will facilitate collaboration}. Until then, computational epidemiology tools to reason about diseases, to infer unknown quantities are a keystone in the path towards elimination, which calls for improvements in every facet of disease control.
% computer-age
%The systems under study are complexes, and models are an invitation to explore our intuitions and assumptions about the dynamic of disease transmission. Outcomes from models are often unexpected even for the practitioners who carefully designed every stage of the exercise. These surprising results warrant an investigation. More often than not, it is caused by an error, such as the omission of an important process; in this case the model is simply adapted. But sometime results are surprising due to reality being surprising, here lies the beauty of modeling. The first pipeline \textsc{covid}-19 projections on February 27, 2020 comes to mind as a depressing example, but so are the first optimistic expectations results in December 2019. 

%Some models presented in this thesis have guided directly decision maker while other contributed to knowledge on the processes underlying the transmission of \textsc{covid}-19 and cholera.





