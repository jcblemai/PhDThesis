¨\chapter*{Conclusion and Perspectives}
\addcontentsline{toc}{chapter}{Conclusion and Perspectives}
\markboth{Conclusion and Perspectives}{}
Each of the five chapters of this thesis introduces an infectious disease model tailored to tackle a public-health policy related question, whether related to the identification of transmission pathways or to the effects of past or planned interventions. Naturally, the angle of approach to the complex phenomena that are cholera and \textsc{covid}-19 transmission varies depending on the research question and the specificities of the study setting. Hence each modeling work takes a different path within the interactive framework to model infectious disease dynamics\footnote{see \textsc{Introduction}, fig. \ref{fig:modeling} and \fullcite{Heesterbeek:ModelingInfectiousDisease:2015}.}.  Let's go through this conversation of models one more time.


First, as in classical hypothesis testing, a model might be considered as a simplified representation of reality. Among different hypotheses, the aim is to find the ``true'' model underlying the observed dynamics. This perspective is taken in \textsc{Chapter~2}, where the explanatory power of different pathways for rainfall-mediated cholera transmission is compared\cite{Rinaldo:Reassessment20102011:2012, Eisenberg:ExaminingRainfallCholera:2013}. Results stress the importance of rainfall as a covariate for cholera transmission while highlighting the complexity of the mechanistic pathways considered. While context-dependent, it is nevertheless interesting to observe how the different transmission routes proposed in the literature fare on intra-seasonal rainfall events in Juba, South Sudan.

The perspective on models is shifted in the policy related questions proposed in \textsc{Chapters 3} and 4: here the predictive accuracy with respect to the interventions and transmission dynamics of interest is desired. Models are designed to most adequately reproduce an unknown, highly complex reality in order to perform experiments\footnote{Indeed, accurate mechanistic modeling is still required for the processes of interest, and these two perspectives are related more than opposed.}. Building on ECHO's experience on cholera modeling\cite{Rinaldo:RiverNetworksEcological:2020a} a spatial stochastic model of cholera transmission in Haiti is proposed in \textsc{Chapter~3}. It is used as part of a larger multi-modeling study\cite{Lee:AchievingCoordinatedNational:2020}, to study the probability of cholera elimination from Haiti under different scenarios of mass vaccination campaigns. A retrospective analysis reveals that while the proposed model fits past dynamics, it under-estimates the probability of elimination of cholera in Haiti. This discrepancy is a reminder that modeling is no silver bullet, further stressing the importance of careful model design aligned with the study objective. 

Non-withstanding the \textsc{covid}-19 pandemic, cholera would have been the sole focus of the present thesis. This regretful interruption left unanswered some research questions on this ancient disease. There have been no confirmed cholera cases since early 2019 in Haiti, which is surprising with regard to the aforementioned projections. What is the cause of the elimination of cholera from Haiti? What role did climate, herd immunity, and the WaSH interventions carried out in the past years played\cite{Rebaudet:CaseareaTargetedRapid:2019} in this elimination? The answer would provide additional insights on the path towards the elimination of cholera by 2030, as sought after by WHO and GTFCC. Toward this goal, the oral cholera vaccine stockpile put in place at the disposal of countries is brought to play a strategic role\cite{GlobalTaskForceonCholeraControl:EndingCholeraGlobal:2017}. However the population at risk vastly exceeds the vaccine supply and production\cite{Pezzoli:GlobalOralCholera:2019}, and effective targeting of the existing doses remains an important problem\cite{Lessler:MappingBurdenCholera:2018} on the road towards elimination.

The rapid spread of SARS-CoV-2 mobilized infectious disease epidemiologists across the world. Most of the work undertaken for the response to the \textsc{covid}-19 pandemic took place within or around the COVID Scenario Pipeline, a configurable framework to project epidemic trajectories and healthcare impacts under different suites of interventions. The pipeline is used to support several partners including the state of California and the national US response. It is still being actively developed to address the ever-changing needs of decision-makers. Since the description given in \textsc{Chapter~4}, a year of historical data brought the need for high-dimensional inference algorithms, and the challenges posed by SARS-CoV-2 necessitated more flexible disease transmission and health outcome modeling. These evolutions allowed the capture of the dynamics of competing strains, immune escape, and the different vaccination campaigns\cite{Borchering:ModelingFutureCOVID19:2021}. Hence the pipeline is now an operational forecasting platform that provides a unified framework to project and forecast disease dynamics from emergence to endemicity\footnote{Updated outputs of the pipeline are visible on the \textsc{covid}-19 Scenario Modeling Hub and the \textsc{covid}-19 Forecast Hub (\url{covid19scenariomodelinghub.org} and \url{covid19forecasthub.org}).}.

Assessing the effectiveness of past policies is paramount to project disease dynamics and to inform future decisions. In Switzerland, the pipeline has been used to inform CHUV, the main hospital of Canton de Vaud. The dialogue with policy-makers triggered the exchange of a dataset of the length of stays in hospitals to improve scenario planning report accuracy. In turn, this data has enabled a research study on the estimation of SARS-CoV-2 reproduction number, $R_0$, in Switzerland. Using stochastic models and iterated filtering -- an alternative procedure to the predominant $R_0$ estimation methods\cite{Gostic:PracticalConsiderationsMeasuring:2020a,Pasetto:RangeReproductionNumber:2020} -- it uncovers the effectiveness of non-pharmaceutical interventions against \textsc{covid-19} in Switzerland. Among the numerous takeaways from this early \textsc{covid}-19 work presented in \textsc{Chapter~5}, the timing of the decrease in transmission preceding NPIs implementation is especially interesting. %The feedback loop between was closed when these estimates were used as assumptions in subsequent modeling reports for Canton de Vaud.

Finally, provided an accurate model of transmission dynamics and interventions effectiveness, optimal control is the ultimate stage of infectious disease modeling towards informed decisions: policies that minimize the burden of diseases are programmatically designed. To date, both the demanding prerequisites and the difficulty of controlling epidemiological models at scales that are relevant to decision-makers were blocking to apply optimal control on practical problems. In \textsc{Chapter~6}, a large-scale optimal control framework coupled to an existing model of \textsc{covid}-19 transmission\cite{Gatto:SpreadDynamicsCOVID19:2020,Bertuzzo:GeographyCOVID19Spread:2020} is presented.  Using automatic differentiation and non-linear programming, the proposed solver design the most effective vaccination strategy for a given objective, feasible under operational constraints. This proof of concept is performed on spatial vaccines allocation against \textsc{covid}-19 in Italy. While limitations of the model, such as the absence of age-stratification, restrict the scope of the results, this first attempt at country scale optimization of a compartmental disease dynamic model is promising to exploit this method that provides a benchmark of what is possible to achieve with available resources.% and hopefully a significant contribution towards making these novel algorithms tools against infections disease.


Taken as a whole, the present thesis demonstrates the relevance of modeling as a tool to inform public-health decisions through five studies, each providing scientific insights on the underlying transmission processes or the effectiveness of past and future control policies. Yet, it also shows that there is no one-size-fit-all approach to infectious disease modeling: each research question and transmission setting requires numerous adjustments to capture and project the disease dynamics. %The diversity of approaches and viewpoints\footnote{enhanced with the diversity of sensibilities and background of the practitioners and teams.} undertaken is important to tackle these problems. 
This takeaway might seems discouraging, but past efforts are not lost as each model builds on the previous works, borrowing conceptual breakthroughs and methods. With these new tools, %(in a broad sense, encompassing  mathematical approaches and software packages)
 the scientific and public-health communities are left better armed to face existing and upcoming threats\footnote[][-3\baselineskip]{In the past, outbreaks and pandemics have sparked rapid advances on various topics of infectious disease modeling. The response to the \textsc{covid}-19 pandemic has benefitted extensively from conceptual and concrete tools developed in the past, such as the $R_0$ package that was developed after H1N1 \parencite{Obadia:R0PackageToolbox:2012}, but also \eg real-time forecasting methods against Ebola (2014--2015), statistical approaches for cholera, multi-modeling studies for influenza, ... Indeed the \textsc{covid}-19 pandemic has sparked the development of an unprecedented amount of novel tools.}. Hopefully, the methods developed as part of the present thesis, whether at their infancy for optimal control or mature like the COVID Scenario Pipeline, will strengthen the arsenal of methods to guide decisions against infectious disease transmission.
  
%It is possible to envision as tools three chapters of the thesis. \marginnote{Apart from \textsc{Chapter~2}, code and data to reproduce these projects are available. However, quality documentation and instruction are lacking.} Arguably, the COVID Scenario Pipeline is already one: it has been used by different entities across many settings and has proven robust throughout the pandemic. The method used to estimate $R_0$ in Switzerland, while not novel, is different from most current methods based either on deconvolution (backward or forward viewpoints) or parametrized functional forms. While it is harder to use and necessitates some important assumptions, it provides a global look on time-varying transmission, which might be useful in other contexts.%has proved useful for \textsc{covid}-19. % Its powerful uncertainties representation may be more even relevant in contexts where data is scarce. 
%Finally, the optimal control framework allows one to explore a model and a disease, from another angle. The computer handle the choices to control complex high-dimensional phenomena, given operational constraints. It allows for the design of effective interventions and the best allocation of control resources, making use of every feature of the model. 

The sustaining transmission of cholera and \textsc{covid}-19 is the product of complex interactions between the environment, individuals, pathogens, and societies. Despite tremendous advances, especially for \textsc{covid}-19, the data available about infectious diseases will stay noisy, missing, and biased in the foreseeable future. Modeling helps to design effective policies, allowing to properly weigh different courses of action and to convey the associated outcomes and uncertainties. As such, it has been a key instrument in the fight against infectious diseases, a point made more evident during the \textsc{covid}-19 pandemic. 

Using compartmental models and computer-age statistical inference methods, the present thesis demonstrates the relevance of modeling to inform the control measures against epidemics. The scientific understanding of transmission pathways and intervention effectiveness is unraveled in specific contexts. Moreover, the (ongoing) issue of scenario planning reports to inform \textsc{covid}-19 to decision-makers provide a concrete example of the influence of modeling on public-health policies. 
While the solutions to fight infectious diseases exist, resources are notoriously lacking, even more so because outbreaks piles upon impoverished communities, or in regions plagued by conflicts, natural disasters, and instability. Computational epidemiology allows one to communicate the trade-offs on policies and to ensure the best allocation of limited resources, and so that \eg the effect of every vaccine dose is maximized. %are a keystone in the path towards elimination, which calls for improvements in every facet of disease 

Indeed, there is so much modeling can do, and the elephant in the room is the brutal lack of resources to fight the infectious diseases that pose a constant threat on children and adults. The \textsc{covid}-19 pandemic has exposed what is possible in terms of mobilization and collaboration from partners around the world. Hopefully, it will set a precedent, and other diseases that plague impoverished and stigmatized communities will also benefit from the involvement and resources necessary to lighten this unfair burden\footnote[][-2\baselineskip]{Even solely for modeling, the sudden availability of mobility datasets, something long desired for cholera, is an example of the paradigm shift brought by a pandemic that affected every country in the world, and hopefully lessons learned will facilitate collaboration.}.

% computer-age
%The systems under study are complexes, and models are an invitation to explore our intuitions and assumptions about the dynamic of disease transmission. Outcomes from models are often unexpected even for the practitioners who carefully designed every stage of the exercise. These surprising results warrant an investigation. More often than not, it is caused by an error, such as the omission of an important process; in this case, the model is simply adapted. But sometimes results are surprising due to reality being surprising, here lies the beauty of modeling. The first pipeline \textsc{covid}-19 projections on February 27, 2020, comes to mind as a depressing example, but so are the first optimistic expectations results in December 2019. 

%Some models presented in this thesis have guided directly decision-maker while others contributed to knowledge on the processes underlying the transmission of \textsc{covid}-19 and cholera.

