\chapter*{Conclusion and Perspectives}
\addcontentsline{toc}{chapter}{Conclusion and Perspectives}
\markboth{Conclusion and Perspectives}{}
The 5 models presented in this thesis each answer a question related to either the scientific understanding of transmission or the effects of control policies. While these models shares the same basis, different features of disease transmission are put forward depending on the goal. 
In the first work presented in \textsc{Chapter~2}, modeling is used as a representation of a real process, and the the explanatory power of different pathways for cholera transmission is compared. While the results are not strong enough to be generalized. It is nevertheless interesting to observe how the different transmssion route assumed in the existing litterature reproduce the dynamics of this epidemic in Juba.

\textsc{Chapter~3} is part of a collaboration studying the elimination of cholera in Haïti after an hypothetical mass campaign of oral cholera vaccine. An stochastic, spatial model is proposed, that builds up on the long history of cholera modeling in the ECHO laboratory. A restrospective analysis shows that our model under estimates the probability of elimination in the no vaccination scenario, and did not forsee the elimination of cholera from Haiti. Despite these conclusion, oral cholera vaccines remain a powerful tools, promising in many countries with endemic cholera. 
Nonwithstanding the COVID-19 pandemic, cholera would have been the sole focus of this thesis. This regretful interuption makes the next research question open: is it possible to find out the cause of the (surprising for our model) extinction of cholera from Haiti. And to what extent climate and the WaSH intervetions carried out were important in reaching this goal. Another regret is that the current pandemic has exposed what is possible in term of mobilisation and collaboration from partners around the world. The sudden availablity of mobility datasets, something long desired for cholera, is an example of the paradigm shift brought by this pandemic that affected every country in the world. Hopefully, it will set a precedent to involve more partner on great collaborations to tackles other diseases that remains unfair to children and adults worldwide, such as the elimination of cholera.

Most of the work undertaken for the COVID-19 response took place within the COVIDScenarioPipeline, a tool that projects epidemic trajectories and healthcare impacts under different suites of interventions, to produce report to inform and guide decisions. The pipeline is still actively developed and the reports send to gouvernment around the world. Since the description given in \textsc{Chapter~3}, a year of historical data brought the need for inference and calibration, and the pipeline was further upgraded to model vaccines and variants. The pipeline is a close realisation of the operation forecasting platform suggested in the early research plan.
\marginnote[-4\baselineskip]{The most current public outputs of the pipeline are visible on the COVID-19 Scenario Modeling Hub and the COVID-19 Forecast Hub (\url{covid19scenariomodelinghub.org} and \url{covid19forecasthub.org}).}

While using the pipeline to inform CHUV, Canton de Vaud main hospital, the obtention of a dataset of lenght of stays has enabled a research study on the estimation of the reproduction number in Switzerland. This allowed to uncover the effectiveness of interventions, using stochastic models with filtering, a procedure different from most $R_0$ estimations. Of the many takeways from this work, the timing of transmission decrease, slightly before the lockdown is especially interesting.

And finally, an optimal control framework is presented, applied on the transmission of COVID-19 in Italy. Using automatic differenciation and non-linear programing, the solver design the most effective vaccination strategy for a given objective, under operational constraint. A key prerequiste is a model providing a accurate reproduction of reality, and limitations in the models such as the abscence of age-stratification limit the scope of the presented study. Nevertheless, optimal control applied to epidemiological transmsisson is a novel tool that has the potential to provide particulary important  insight in ressource limited setting, such as the global cholera vaccine stockpile.

The present thesis shows how modeling may help to understand disease transmission, assess and design control policies.
 The presented research is a testament of the importance of modeling. as a tool to reason about disease, and to help interventions to block transmission. As a though process, it is interesting as often the results are surprising. More often than not due to an error in the model specification, in this case the investigation enable the discovery of important processes. The studied system is complex and its many interactions and models are an invitation to explore in-depth our belief and assumptions. But sometime model are surprising due to reality being supprising. Or depressing such as the first COVID-19 transmission models. In this case careful communication and. Models presented in this thesis have either guided directly decision makers
 
 % What I personnaly enjoyed about these golem, as Richard MacElreath call them, is that after careful design, you may be surprised by the results. And after,
% model as tools dire qu'on a fait des progrès sur l'OCP
An unified framework for disease modeling is desirable as the diversity of background and sensibilities in practitioners bring complementary viewpoint. And the choices on which facets of this complex system to focus on. 

Often outbreaks and pandemics (e.g H1N1 2009, Cholera 2010, Ebola 2014--2015, and Zika, COVID-19) have sparked advances in many different aspects of infectious disease modeling. The research and public-health communities design new methods while tools leap forward. These advances remains available for endemic diseases and futur pandemic. It improves the preparedness  against the emergence of new pathogens.\marginnote{The $R_0$ package and thouroughly used for the COVID-19 pandemics was developed after the H1N1 pandemics\fullcite{Obadia:R0PackageToolbox:2012}, comes to mind. Advances in real-time modelisation for Ebola, ...}. It is possible to envision as tools the 3 last chapters of the thesis. Arguably, the COVIDScenarioPipeline is already one, as it has been used in different projects by different entities. It has proven robust, and recent developement improved its flexibility to allow for other diseases to be modeled. The method to estimate $R_0$, while not novel, is different from most of the current method based either on deconvolutions (backward-forward viewpoint) or parametrized model. While it is involved and necessiate good data and representative assumptions, it works well and provide an alternative look on transmission.
Finally, the optimal control framework allows to explore a model and a disease, from another angle. It is interesting, and slightly dangerous to let the the computer handle the choices to control a complex high-dimenssional  phenomena, just giving it an objective and some constraints. The obtained solution make use of every feature of the model, and one must be careful on the interpretability, these algorithm allow for the design of effective interventions and the best allocation of control ressources. 
\marginnote{The code and data of these project is available, neverthess generality and documentation are still needeed}

As infectious diseases pose a constant threat on children and adults accross the globe, tools and prepardenss to constrction effective interventions are needed. Towards this, improvement in the scientific understandfing of diseases transmsiion, but also the exploration of the facets of the complex system tthat link environement, individuals and societies. While modeling may mislead or render overconfident even in good hands, parcimony and careful inspection makes modeling a key instrument to fight diseases. A point that has been evident during the covid-19 pandemic.

In fact, aside from scienfic curiosity lies the neccessity to control infectious disease. Despite tremondous advances, the data about infectious diseases will stay noisy, missing and biased in the foreseable futur. More so because oubreaks thrive where conflicts, natural disaster and instability lies. Mathematical and computational tools to reason about diseases, to infer unknown quantities are a keystone in the path towards elimination, which calls for improvements in every facet of disease control. 

%The course of this thesis has brough reports to decision makers, and developed tools to do so. Towards the elimination of cholera. Moreover, w- have . After COVID GTFCC ? COVID ?