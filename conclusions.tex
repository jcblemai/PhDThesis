\chapter{Conclusion and Perspectives}
Over the course of the past four years, we have developed the 5 models presented in this thesis. More importantly, each of this models answers a different questions related to the  scienétific understanding or the control of the disease it represents. For this, each model focuses on a different facets of the disease transmission. 

The first work presented uses modeling as a representation to reality, and we assess the explanatory power of different pathways for cholera transmission. As the results are  specific to this epidemic in Juba, and are not strong enough to be generalized, it is nevertheless interesting to see that direct approaches makes... The second work, part of a larger collaboration, studies the probabilities of elimination of cholera in Haïti under mass vaccination campaign of oral cholera administration. While the cholera may have finally left the Haitian alone, oral cholera vaccines are a powerful tools for other contexts.
Without the COVID-19 pandemic, cholera would have been the sole focus of this thesis. It regretful that my, and others focuses had to leave cholera research plan.
 COVID has shown us the kind of data, mobility, we are able to get from . Hopefully this wills set a precedent for great collaborations to tackles other diseases that are unfair to children and adults worldwide.  I hope in the futur, the extinctions of cholera will benefit from the same colaborations, and I trust for this gtfcc and.
 COVID-19 work took the shape of a tools for scenarios planning, reports to many different. The project is still alive in many different, Continued development of models: pipeline > vaccines > variants.
From using the pipeline to informs the local hospital, data which sparked diffent question on lenght of stays, and effectiveness of interventions.

Finally, we present an optimal control applied to the transmission of COVID-19 in Italy. In the ideal case, model being perfect reproduction of reality, these tools . Nevertheless with incomplete models, allow us to search and discover. 

% model bring
The COVIDScenariopipeline taught me. For the first time I got results I found surprising, and convercing.  What I personnaly enjoyed about these golem, as Richard MacElreath call them, is that after careful design, you may be surprised by the results. And after,
We study a complex systems with many interactions: invter



We have shown how modeling may help to understand disease transmission, assess and design control policies. 

% model as tools
While an unified framework for disease modeling, once a goal of this thesis, has proven not desirable as the diversity necessary to explore many facets of the complex system. Often outbreaks and pandemics (e.g H1N1 2009, Cholera 2010, Ebola 2014--2015, and Zika, COVID-19) have sparked frenetic developement from the research communities different aspects of modeling improves drastically and available tools makes a leap forward. Most of these tools remains for endemic diseases and futur pandemic
\marginnote{The $R_0$ package and thouroughly used for the COVID-19 pandemics was developed after the H1N1 pandemics\fullcite{Obadia:R0PackageToolbox:2012}, comes to mind. Advances in real-time modelisation for Ebola, ...}. 
While the COVIDScenarioPipeline is used in different proects, and has proven robust ... Some of the other models we propose in this thesis would be interesting to toolify. The way we estimate $R_0$ is not novel, but we were surprised to see how well it worked with the pomp and mif2. It's interesting addition to methods like, and to 
Moreover, optimal control allows to explore a model, and a disease, from another angle. It is easy, interesting, and slightly dangerous to let the the computer handle the choices to control a complex high-dimenssional  phenomena, just giving it an objective and some constraints. While the obtained solution make use of every feature of the model, they allow in a principle.
To these three projects, while publication and code are available, it os

Existing infectious diseases poses a constant threat on humans accross the globe. Emergence of new pathogens puts us at risk of . Tools and integrations for for 

Explore more the facets of the complex system.

Despite the models are dangerous, make overconfident. Parcimony and care turns these thing into useful creatures.

There are two goals that understanding what happens, and being able to control it. Once thought to be soon over, infectious diseases still 

Remains a lot of questions 

% towards elimination
Above the scienfic curiosity, neccessity to control effectively. The data about infectious diseases continues to be noisy, missing and biased. Especially as these thrive arounds conflicts, natural disaster. Computational tools to be able to infer states from such data is always welcome. Extinction of this diseases will be possible with improvement in every facet of ..

The course of this thesis has brough reports to decision makers, and developed tools to do so. Towards the elimination of cholera. Moreover, we have . After COVID