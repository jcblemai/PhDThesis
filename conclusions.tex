\chapter*{Conclusion and Perspectives}
\addcontentsline{toc}{chapter}{Conclusion and Perspectives}
\markboth{Conclusion and Perspectives}{}
Each of the five chapters presented in this thesis introduce an infectious disease model tailored to tackle a public-health policy related question, whether related to infectious disease transmission pathways or to the effects of control policies. Each study take a different path within the interactive process to model infectious diseases, see \textsc{Introduction}, fig. \ref{fig:modeling}. Naturally, the angle of attack to approach the complex phenomenas that are cholera and \textsc{covid}-19 varies with the studied disease and the research question but also with many factors that set the stage for the study, among which \eg the data available, the potential collaboration with decision-makers. Let's go through this conversation of models one more time.

First, as in classical hypothesis testing, a model might be considered as a simplified representation of reality. Among different hypothesis, the aim is to find the ``true'' model underlying the observed dynamics. This perspective is taken in \textsc{Chapter~2}, where the explanatory power of different pathways for rainfall-mediated cholera transmission are compared\cite{Rinaldo:Reassessment20102011:2012, Eisenberg:ExaminingRainfallCholera:2013}. Results stress the importance of rainfall as a covariate for cholera transmission while highlighting the complexity of the mechanistic pathways considered. While context-dependent, it is nevertheless interesting to observe how different transmission routes proposed in the literature fare on intra-seasonal rainfall events in Juba, South Sudan.

Policy relates questions in \textsc{Chapters 3} and 4 change the outlook and predictive accuracy with respect to interventions and transmission of interest is desired; models that most adequately reproduce an unknown, highly complex reality in order to perform experiments are designed\footnote{Indeed, accurate mechanistic modeling is still required for the processes of interest and these two perspectives are more related than opposed}. As a part of multi-modeling study\cite{Lee:AchievingCoordinatedNational:2020}, a spatial stochastic model of cholera transmission in Haiti is proposed in \textsc{Chapter~3}. Building on ECHO's experience on cholera modeling\cite{Rinaldo:Reassessment20102011:2012}, it is used to assess the probability of cholera elimination from Haiti under different scenarios of mass vaccination campaigns. A retrospective analysis reveals that while the proposed model fits past dynamics, it under-estimates the probability of elimination of cholera in Haiti. This result recalls that modeling is not a silver bullet and stresses the importance of careful model design and that projections uncertainties communication includes modeling assumptions in addition to modeled uncertainty.

Non-withstanding the \textsc{covid}-19 pandemic, cholera would have been the sole focus of the present thesis. This interruption leave the logical next research questions unanswered. There are no confirmed cholera cases since early 2019 in Haiti, which is surprising with regard to the presented modeling work. What is the cause of the extinction of cholera from Haiti ? To what extent did climate, herd immunity and the WaSH interventions carried out in the past year played a role ? Hopefully the answer will provide additional insights on the path towards elimination of the cholera by 2030, as sought after by WHO and GTFCC. Along this road a cholera vaccine stockpile has been put in place at the disposal of countries. Still at risk population vastly exceed vaccine supply and production\cite{Pezzoli:GlobalOralCholera:2019}, and effective allocation of the existing doses between countries remains an open-problem.

Most of the work undertaken for the response to \textsc{covid}-19 pandemic took place within or around the COVID Scenario Pipeline, a configurable framework to project epidemic trajectories and healthcare impacts under different suites of interventions. The pipeline is used to support several partners including the state of California and the national US response. It is still being actively developed to address the ever-changing need of decision-makers. Since the description given in \textsc{Chapter~4}, a year of historical data brought the need for high-dimensional inference algorithms, and the challenges posed by SARS-CoV-2 necessitated more flexible disease transmission and health outcome modeling to capture the dynamics of competing strains, immune escape and the different vaccination campaigns. The pipeline is an operational forecasting platform -- a milestone in the initial research plan --, it provides a unified framework to project and forecast dynamics from disease emergence to endemicity\footnote{Updated outputs of the pipeline are visible on the \textsc{covid}-19 Scenario Modeling Hub and the \textsc{covid}-19 Forecast Hub (\url{covid19scenariomodelinghub.org} and \url{covid19forecasthub.org}).}.

Assessing the effectiveness of past policies is paramount to project disease dynamics and to inform future decisions. In Switzerland, the pipeline has been used to inform CHUV, the main hospital of Canton de Vaud. The dialogue with policy makers triggered the exchange of a dataset of length of stays in hospital to improve scenario planning report accuracy. In turn, this data has enabled a research study on the estimation of SARS-CoV-2 reproduction number, $R_0$, in Switzerland. Using stochastic models and iterated filtering -- a alternative procedure to the main $R_0$ estimation methods -- it uncovers the effectiveness of non-pharmaceutical interventions against \textsc{covid-19} in Switzerland. Among the numerous takeaways from this early \textsc{covid}-19 work presented in \textsc{Chapter~5}, the timing of the decrease in transmission preceedding NPIs implementation is especially interesting. The feedback loop was closed when these estimates were used as assumptions in subsequent modeling reports for Canton de Vaud.

Finally, provided an accurate model of transmission dynamics and interventions effectiveness, an objective and descriptions of operational constraints, optimal control is the ultimate stage of infectious disease modeling towards informed decisions: policies that minimize the burden of a disease are programmatically designed. To date both the demanding prerequisites and the difficulty of controlling epidemiological models at scales that are useful were the progress on this topic. In \textsc{Chapter~6}, a large-scale optimal control framework coupled to an existing model of \textsc{covid}-19 transmission is presented.  Using automatic differentiation and non-linear programing, the proposed solver design the most effective vaccination strategy for a given objective, feasible under operational constraints. This proof of concept is performed on spatial  vaccines allocation against \textsc{covid}-19 in Italy. While limitations in the model, such as the absence of age-stratification, limit the scope of the presented results, as the first country scale optimization of a compartmental model it is a significant contribution towards making these novel algorithms a tool against infections disease.

This thesis a whole demonstrate the relevance of infectious disease modeling as a tool to inform public-health decisions through studies where that provides scientific insights on the underlying transmission processes and the effectiveness of past and future public-health policies. % computer-age
The systems under study is complex with many component interacting, and models are an invitation to explore in-depth our intuitions and assumptions about the dynamic of disease. Modeling results are often unexpected even for the practitioners who carefully designed every included process. Surprising results warrant investigation. More often than not there it is caused by an error, such as the omission of an important process.  In this case the model is adapted and . But sometime model are surprising due to reality being surprising, here lies the beauty of modeling. The first pipeline \textsc{covid}-19 projections on February 27, 2020 comes to mind as a depressing example, but so are the first optimistic expectations results in December 2019. Some models presented in this thesis have guided directly decision maker while other contributed to knowledge on the processes underlying the transmission of \textsc{covid}-19 and cholera.

Yet, the presented research works prove that there is no one-size-fit-all approach to infectious disease modeling, and each research question and transmissions settings requires numerous adjustments to capture and project the disease dynamics. A unified framework is not possible, as diverse approaches and viewpoints undertaken are important to tackle the problems\footnote{which are enhanced with the diversity of sensibilities and background of the practitioners and teams.}.  
  This takeaway might seems discouraging if not highlighting that this enterprise never starts from scratch: each model builds on the previous works, borrowing conceptual breakthrough and methods. The scientific and public-health communities are better armed with these tools (in a broad sense encompassing \eg mathematical approaches to software packages) to face upcoming threat and to tackle harder problems on existing diseases. Often outbreaks and pandemics have sparked advances on many aspects of infectious disease modeling: new methods are designed and tools leap forward. These advances remains available for endemic diseases and future pandemic. It improves the preparedness against the emergence of new pathogens, and the response to the \textsc{covid}-19 pandemic has benefitted extensively from conceptual and concrete tools developed from past pandemics\footnote{The $R_0$ package and thoroughly used for the \textsc{covid}-19 pandemics was developed after the H1N1 pandemics\fullcite{Obadia:R0PackageToolbox:2012}, comes to mind. Among other topics there are \eg real-time forecasting for Ebola 2014--2015, one-health approach for cholera, multi-modeling studies for influenza, and Zika.}. 
  \marginnote{Apart from \textsc{Chapter~2}, code and data to reproduce these projects are available. However, quality documentation and instruction are lacking.}
  
  In addition to the research studies results by themselves, it is possible to envision as tools three chapters of the thesis. Arguably, the COVID ScenarioPipeline is already one: it has been deployed by different entities across many settings, and has proven robust. Recent developments improved its flexibility and other disease  to allow for other diseases to be modeled. The method to estimate $R_0$, while not novel, is different from most of the current method based either on deconvolution (backward or forward viewpoints) or parametrized model (where the time series is not entirely free). While it is involved and necessitate some important assumptions, it provides an alternative look on transmission and has proved useful for \textsc{covid}-19. Its powerful uncertainties representation may be more even relevant in contexts where data is scarce. 
Finally, the optimal control framework allows one to explore a model and a disease, from another angle. The computer handle the choices to control a complex high-dimensional  phenomena, given an objective and constraints. The obtained solution make use of every feature of the model, and one must be careful on the interpretability, these algorithm allow for the design of effective interventions and the best allocation of control resources which as always scarce.

The course of this thesis explored epidemics as complex system that link environment, individuals, pathogens and societies,  scientific understanding of diseases transmission has brought reports to decision makers and influenced public-health decisions, and developed tools that will hopefully a contributes the the arsenal of methods to do fight infectious disease. It  to address improvement in t, but also 

Indeed, there is so much modeling can do, and the elephant in the room is the lack of resources to fight infectious diseases pose a constant threat on children and adults across the globe. The \textsc{covid}-19 pandemic has exposed what is possible in term of mobilization and collaboration from partners around the world. Hopefully, it will set a precedent and other diseases that that plague impoverished and stigmatized communities around the world will also benefit from involvement and resources\footnote{Even there, the sudden availability of mobility datasets, something long desired for cholera, is an example of the paradigm shift brought by a pandemic that affected every country in the world, and hopefully lessons learned will facilitate collaboration}. 

As for the foreseeable future resources are lacking and modeling helps allocate them in an informed way, making its a key instrument to fight diseases. This point has been evident during the covid-19 pandemic. Despite tremendous advances, the data about infectious diseases will stay noisy, missing and biased in the foreseeable future. More so because outbreaks thrive where conflicts, natural disaster and instability lies. Mathematical and computational tools to reason about diseases, to infer unknown quantities are a keystone in the path towards elimination, which calls for improvements in every facet of disease control. 







