\chapter*{Conclusion and Perspectives}
\addcontentsline{toc}{chapter}{Conclusion and Perspectives}
\markboth{Conclusion and Perspectives}{}

Each of the five chapters of this thesis present an infectious disease model tailored to tackle a public-health policy related question, whether related to infectious disease transmission pathways or to the effects of control policies. While sticking to the 4 stage framework for modeling infectious disease (see \textsc{Introduction}, fig. \ref{fig:modeling}), different viewpoint on modeling are taken to approach different facets of the complex phenomenas that are cholera and \textsc{covid}-19 epidemics.

First, as in classical hypothesis testing, a model might be viewed as a simplified representation of an actual processes, and among different hypothesis the aim is to find the ``true'' model underlying the observed dynamics. This perspective is taken in \textsc{Chapter 2}, where a comparison of the explanatory power of different pathways for rainfall-mediated cholera transmission is presented. Finding stress the importance of rainfall as a covariate for cholera transmission while highlighting the complexity of the mechanistic pathways considered.  While context-dependent, it is nevertheless interesting to observe how different transmission routes proposed in litterature fare on intra-seasonal rainfall events in Juba, South Sudan.

In \textsc{Chapters 3} and 4, models are designed for their predictive accurary with respect to specific interventions and transmission facets; it is seeked models that most adequately reproduce an unknown, highly complex reality in order to perform experiments\footnote{Indeed, accurate mechanistic modeling is still required for the processes of interest and the two perspectives are related}.  An spatial stochastic model of cholera transmission in Haiti is proposed in \textsc{Chapter~3}. Building on ECHO's long history of cholera modeling, the goal is to assess the probability elimination of cholera from Haiti after different scenarios of mass campaign of oral cholera vaccine distribution. This work is a piece in a larger multi-modeling study, and a restrospective analysis shows that while the proposed model fits past dynamics, it also under-estimates the probability of elimination of cholera from Haiti in the no vaccination scenario. It recalls the importance of careful model design and uncertainties communication when modeling highly uncertain systems. 

Nonwithstanding the \textsc{covid}-19 pandemic, cholera would have been the sole focus of the present thesis. This interuption left open-questions and untackled problems in cholera modeling. The next step would have been to search for the cause the cause of the (surprising with regard to the presented modeling work) extinction of cholera from Haiti, wondering to what extent climate, herd immunity and the WaSH interventions carried out in the past year played a role. However, the \textsc{covid}-19 pandemic has exposed what is possible in term of mobilisation and collaboration from partners around the world. The sudden availablity of mobility datasets, something long desired for cholera, is an example of the paradigm shift brought by a pandemic that affected every country in the world. Hopefully, it will set a precedent to involve more partners on great collaborations to tackles other diseases that remains unfair to children and adults worldwide, such as cholera.

Most of the work undertaken for the \textsc{covid}-19 response took place within or around the COVID Scenario Pipeline, a tool that projects epidemic trajectories and healthcare impacts under different suites of interventions, to produce report to inform and guide decisions. The pipeline, used to support several partners including the state of California and the national US response, is still being actively developed to address the ever-changing need of decision makers. Since the description given in \textsc{Chapter~4}, a year of historical data brought the need for inference and calibration, and the pipeline was further upgraded to model vaccines and variants. The pipeline is a close realisation of the operational forecasting platform suggested in the initial research plan, providing a unified framework to project and forecast dynamics from disease emergence to endemicity\footnote{and updated outputs of the pipeline are visible on the \textsc{covid}-19 Scenario Modeling Hub and the \textsc{covid}-19 Forecast Hub (\url{covid19scenariomodelinghub.org} and \url{covid19forecasthub.org}).}

In Switzerland, the pipeline has been deployed to inform CHUV, Canton de Vaud main hospital. In the dialogue between model adaptation and policy makers, the realization that additional data would improve scenario report trigered the exchange of a dataset of lenght of stays in hospital. In turn, this data has enabled a research study on the estimation of the reproduction number in Switzerland, which allowed  to uncover the effectiveness of interventions, using stochastic models with filtering, a procedure different from most other $R_0$ estimation methods. Of the numerous takeways from this early \textsc{covid}-19, the timing of transmission decrease slightly before the strongest NPIs is especially interesting. The feedback loop was closed when these estimates were used as assumptions in subsequent modeling reports.

Finally, an optimal control framework is presented, applied on the transmission of \textsc{covid}-19 in Italy. Using automatic differenciation and non-linear programing, the solver design the most effective vaccination strategy for a given objective, under operational constraint. A key prerequiste is a model providing a accurate reproduction of reality, and limitations in the models such as the abscence of age-stratification limit the scope of the presented study. Nevertheless, optimal control applied to epidemiological transmsisson is a novel tool that has the potential to provide particulary important  insight in ressource limited setting, such as the global cholera vaccine stockpile. Provided an accurate model, clear objective and descriptions of constraints and possible interventions (something not easy to gather) optimal control would be the pinacle of modeling towards informed decision, designing intervnetion strategies.

This thesis presents a conversation of different models of cholera and \textsc{covid}-19 transmission. These models were developed along the same cycle: model \textit{design}$\rightarrow$ model \textit{fit}$\rightarrow$model \textit{adaptation}$\rightarrow$model \textit{insight} to answer scientific and public-health questions. Computer-age statistical inference at the service of infectious disease modeling help to reason about the complex dynamics of epidemics. 
As a though process, it is interesting as often the results are surprising. More often than not due to an error in the model specification, in this case the investigation enable the discovery of important processes. The studied system is complex and its many interactions and models are an invitation to explore in-depth our belief and assumptions. But sometime model are surprising due to reality being supprising. Or depressing such as the first \textsc{covid}-19 transmission models. In this case careful communication and. Models presented in this thesis have either guided directly decision makers
The importance of  These compartmental models share the same backbone and many characteristics, but different features of disease transmission are emphasized to tailor these approach to the research question at stake. % computer-age

Overall, this thesis is a testament of the relevance of compartmental models as a tools to inform public health decisions, and to reason about infectious disease transmission, providing scientific insights on the underlying processes and the effectiveness of past and future public-health policies. Yet, the presented research works prove that there is no one-size-fit-all approach to infectious disease modeling, and each research question and transmissions settings requires numerous adjustment to capture and project the dynamics of interests. An unified framework is not possible, and the diversity in approaches is important to tackle.  And even on the same The diversity of sensibilities and background in practitioners bring complementary viewpoint.
  This takeaway might seems discouraging if it is not highlighted that it is not necessary to start from scratch: each model builds on the previous works borrowing conceptuals breakthrough which after some work become tools (taken here in a broad sense encompassing conceptual approaches to software package), developed spefically to be re-used accross settings.
  
  Often outbreaks and pandemics have sparked advances on different aspects of infectious disease modeling. The research and public-health communities design new methods and tools leap forward. These advances remains available for endemic diseases and futur pandemic. It improves the preparedness  against the emergence of new pathogens, and the response to the \textsc{covid}-19 pandemic has benefitted extensively from conceptual and concrete tools developed from past pandemcis.\footnote{The $R_0$ package and thouroughly used for the \textsc{covid}-19 pandemics was developed after the H1N1 pandemics\fullcite{Obadia:R0PackageToolbox:2012}, comes to mind. Among other topics there are \eg real-time modelisation for Ebola 2014--2015, one-health approach for cholera, multi-modeling studies for influenza, and Zika.}. 
  In addition to the research studies results and conclusions by themselves, it is possible to envision as tools the 3 last chapters of the thesis. Arguably, the COVID ScenarioPipeline is already one: it has been used across different settings by different entities. It has proven robust, and recent developement improved its flexibility to allow for other diseases to be modeled. Then in The method to estimate $R_0$, while not novel, is different from most of the current method based either on deconvolutions (backward-forward viewpoint) or parametrized model. While it is involved and necessiate good data and representative assumptions, it works well and provide an alternative look on transmission.
Finally, the optimal control framework allows to explore a model and a disease, from another angle. It is interesting, and slightly dangerous to let the the computer handle the choices to control a complex high-dimenssional  phenomena, just giving it an objective and some constraints. The obtained solution make use of every feature of the model, and one must be careful on the interpretability, these algorithm allow for the design of effective interventions and the best allocation of control ressources. 
\marginnote{The code and data of these project is available, neverthess generality and documentation are still needeed}

As infectious diseases pose a constant threat on children and adults accross the globe, tools and prepardenss to constrction effective interventions are needed. Towards this, improvement in the scientific understandfing of diseases transmsiion, but also the exploration of the facets of the complex system tthat link environement, individuals and societies. While modeling may mislead or render overconfident even in good hands, parcimony and careful inspection makes modeling a key instrument to fight diseases. A point that has been evident during the covid-19 pandemic.

In fact, aside from scienfic curiosity lies the neccessity to control infectious disease. Despite tremondous advances, the data about infectious diseases will stay noisy, missing and biased in the foreseable futur. More so because oubreaks thrive where conflicts, natural disaster and instability lies. Mathematical and computational tools to reason about diseases, to infer unknown quantities are a keystone in the path towards elimination, which calls for improvements in every facet of disease control. Tools developed as part of theis thes to better arm in the fight agaisnt futur and exisiting disease.
And despite limited efficacy, oral cholera vaccines remain a powerful tools, promising in many countries with endemic cholera. 
The course of this thesis has brough reports to decision makers, and developed tools to do so. Towards the elimination of cholera. Moreover, w- have .