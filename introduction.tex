% Modelling infectious disease dynamics towards informed public-health interventions, with application on \textsc{covid}-19 and cholera.
\chapter*{Introduction} % broad introductio
\addcontentsline{toc}{chapter}{Introduction}
\markboth{Introduction}{}
%epidemics as phenomena
 %d  unevenly distributed among populations. 
 % improverished communties around the world. %public health issue in many countries, and it's elimination of Global North countries -- . 
 \section{Context}
A series of successes in the fight against infectious diseases -- attributable to \eg hygiene, vaccines, antibiotics, safe water, ... -- brough the hope of a global and durable reduction of the burden towards elimination. Especially in privileged communities, long-term improvements has been achieved for many diseases. While these progresses show that infectious diseases are not a necessary fate, setbacks on the control of existing and emerging pathogens remind us the ongoing threat that these diseases poses on public-health. The current global health picture is marked by inequalities in the distribution of the burdens, which disproportionaly piles up on already impoverished communities, in conflict zone or after natural disasters. To date, communicable diseases cause approx. 15\% of global deaths every year\cite[][Table 1, excl. non-transmissible neonatal and maternal diseases and nutritional diseases; pre-\textsc{covid}-19 estimates]{Roth:GlobalRegionalNational:2018}, and nearly 1/3 of all child deaths are caused by pneumonia and diarrhoea alone\cite[][\ie 2\textsc{M} deaths among under 5, every year.]{WHO:EndingPreventableChild:2013}.
 
 Centuries after the first cholera pandemics and 200 years after the realisation that safe drinking-water, adequate sanitation and hygiene prevent its transmission, cholera is still a threat to millions living in hotspots or at risk areas. The recent emergence of the new coronavirus disease 2019, \textsc{covid}-19, and the strain it put on even world's most advanced healthcare systems recalls the constant risks posed by emerging diseases. 
While elimination might be involved, public-health policies have proven the effectiveness of interventions against infectious diseases, showing that the deaths are preventable. The control of infectious disease presents challenges accross every dimensions of environmental and human health; in order to prevent spillover events, to block transmission routes, and to protect and treat individuals equally. 

Epidemics -- the rapid spread of an infectious diseases in a population -- are a complex phenomenas, the results of interactions between pathogens, enviroment, societies and individuals\cite{Buckee:ThinkingClearlySocial:2021, Heesterbeek:ModelingInfectiousDisease:2015,Rinaldo:RiverNetworksEcological:2020a}. Public-health policies strive to save lives by designing effective mitigitation measures. Among the many aspects of designing such policies, the difficulties of reasoning on uncertainties araising from complex multi-factorial interactions with scarsed and biased information and understanding limit the possibilites. The present thesis explore compartmental models as ways to reason about infectious disease transmission, and as tools to guide decisions. 

Models -- conceptuals representations of systems -- are tools to reason about the world. Historically, conceptuals models of the propagation of infectious diseases, from divine wrath to miasma theory, has motivated more (quarantine) or less (bloodletting) effective approaches to the control and treatment of these pests.
Novel statistical modeling approaches\cite[-3\baselineskip]{Freedman:AssociationCausationRemarks:1999} developed in the 20th century -- and continously improved ever since\cite{Gelman:WhatAreMost:2021} --  provides a formal framework to reasons about complex phenemas. It allows to deal with bias on data collection, to account for epistemic uncertainties, and to encompass uncertainties in the transmission dynamics, the affected populations and the effect of intervention policies in a principled way. The toolbox is further re-enforced by breakthrough advances in mechanistic modeling applied to disease transmission, starting from SIR models\cite{Kermack:ContributionMathematicalTheory:1927, Anderson:PopulationBiologyInfectious:1979}. Later, advances in computing power proved a paradigm shift in dealing with the available evidence, representing and infering features. 

The presents thesis aims to answer a series of research questions accross countries and diseases. Each question is a variations of: why do things spread the way they do ? and how can it be prevented from spreading ? It answers these questions using extensively computer-age modeling and inference methods, in the same xx stages cycle: model design, inference, evaluation, results, communication, listening and an important feedback loop.

FIGURE

Despite this recent formalism, statistical inference remains an art, uncomfortably dependent on the practitioners and their backgrounds\sidenote{Multimodeling studies and collaborative experiments strive to mitigate this issue by bringing together assumption and projections by different groups. See \eg \url{covid19forecasthub.org/community}}. While designing models, choices on what to include and how to express dynamics depends on the underlying knowledge of the processes and the available data. A set of 5 models with different features depending on the context, the possible control measures and how much uncertaintainties on observed quantities is proposed in this thesis.  Some models have stochastic transitions, some deterministic, some spatial whereas others assumes well-mixed population. Despite the foregoing differences, all these models are compartmental with some mechanistic components, and are based on the SIR model. 
Statistical inference conditions the model on the observed data. It allows to infer unobserved quantities from reported ones, such a lockdown intervention from hospitalization and deaths. After an evaluation of the \textit{fit} of the model and the implication on results, if the predictability is satisfying comes the time to answer the research questions. The model many be used to identify transmission routes, as a reproduction of reality. Another way to use a model that reproduce observed dynamics may be used as substitute for experiments, to simulate scenarios and to evaluate the impact of hypothetical polices on a virtual system. It determines the range of expectations one may expect with regard to real world consequences of tested policies. Ccommunication of modeling results, with proper aweareness of limitation, provide the opportunity for feedback.

Another developement presented in this thesis is an application of optimal control to epidemic interventions. Optimal control are novel methods, mostly used in automative engeneering, that enable the automatical design of planning interventions. While technicals adaptations are ncessary, this rigorous framework identify the most effective control measures under a set of operational constraints, discovering invisible features and policies, effective because of complex interactions are discovered. Despite requiring very accurate models, the interest of such tools is as a reaisoning aid uncovering another facet of disease transmission and control of a complex system.

\section{Thesis aim and outline} 
The present thesis has been developed within the Swiss National Foundation project ``Optimal control of intervention strategies for waterborne disease epidemics (\textsc{snf} 200021–172578)’’. It initial goal was to develop a decision support system for the real-time design of optimal intervention strategies against cholera in time and space, which includes an operational forecasting tool with an optimal control solver. This framework has been developed, albeit for \textsc{covid}-19 transmission in Italy, and is presented in \textsc{Chapter 8}. Indeed the \textsc{covid}-19 pandemic has disrupted the research plan of the present thesis. As a consequence, it happens to associate two antipodean diseases. Cholera, one of the most ancient recorded disease\footnote[][10\baselineskip]{History of pre-pandemic cholera is uncertain but numerous accounts of the disease are supposed from as early as 400\textsc{bce}. See \fullcite[p. 95]{Byrne:EncyclopediaPestilencePandemics:2008}.}, has caused 7 pandemics in the modern era. In contrast, \textsc{covid}-19 earliest known onset of symptoms is on December 1, 2019. The pathogen for cholera is a bacteria, \textit{Vibrio Cholerae}, responsible for heavy watery diarrheas, whereas SARS-CoV-2, \textsc{covid}-19’s pathogen, is a virus responsible for respiratory infections. Cholera belongs to the negletected tropical diseases, a class of understudied diseases while the \textsc{covid}-19 pandemic as sparked an unprecented accumulation of evidence\cite{COVID-19OpenAccessProject:LivingEvidenceCOVID19:2020}. Differs also the posited transmission route (fecal-oral vs. respiratory route), the affected communities (``poorest of the poor”, more severe on under 5 vs. global and higher severity on elders), etc... Despite these differences, Cholera and \textsc{covid}-19 shares the absence of an intermediate host, mechanisms that makes their transmission sustainable, challenges toward elimination, and a regretully high tool on humanity, potential for pandemics, and repartition of burden that echoes inequal access to care and further striking stigmatized commuties. More importantly, both cholera and \textsc{covid} are infectious diseases, with the potential of starting epidemics. The scientific methods to answer questions about the spread and control of epidemics are nevertheless similar these diseases and others.


The work presented extensively builds on the ECHO laboratory expertise on the spatially-explicit modeling of cholera transmission\footnote{and waterborne diseases in general; see: \fullcite{Rinaldo:RiverNetworksEcological:2020a, Rinaldo:Reassessment20102011:2012}}. The group has developed over a decade a rainfall-mediated, spatially explicit cholera model that has inspired each of the other models presented in this thesis. The original,  model developed in the group, is presented in \textsc{Chapter 3}, with a short introduction on the ancient disease that is cholera, with a rich and sorrowful history.

\begin{table*}[t]
\label{tab:allmodels}
\centering\small
\begin{tabularx}{\textwidth}{x{14mm}cccx{15mm}cx{15mm}p{30mm}}
\toprule
   \small{\textsc{Chapter}}     & Disease           & Comparments & Processes         & \small{$N_{\text{spatial}}$} & Fit       & Aim            & Reference\\
\midrule
4 & Cholera           & SEIR+B      & Stochastic    & --           & MIF like  & explain         & \tiny{\fullcite{Lemaitre:RainfallDriverEpidemic:2019}}\\
4 & Cholera           & SEIR+B      & Deterministic & --             & MCMC-like & project         & \tiny{\fullcite{Lemaitre:RainfallDriverEpidemic:2019}}\\
5  & Cholera           & SEAIR$^3$+V & Stochastic    & 10        & MIFlike   & scenarios       & \tiny{\fullcite{Lee:AchievingCoordinatedNational:2020}} \\ \addlinespace
6  & \textsc{\textsc{covid}}-19 & SEI$^3$R+VH & Stochastic    & 3’000+    & MCMC-like & project         & \tiny{\fullcite{Lemaitre:ScenarioModelingPipeline:2021}} \\
7  & \textsc{\textsc{covid}}-19  & SEI$^3$R+H  & Stochastic    & --             & MIF-like  & infer           & \tiny{\fullcite{Lemaitre:AssessingImpactNonpharmaceutical:2020}}\\
8  & \textsc{\textsc{covid}}-19  & SEIAR+VH    & Deterministic & 107       & MCMC-like & optimal\newline control & \tiny{\fullcite{Lemaitre:OptimizingSpatiotemporalAllocation:2021}}\\ 
\bottomrule
\end{tabularx}
\caption[Presentation of all models]{Presentation of all  compartmetnal models described in this thesis. In columns compartments, in addition to susceptible S, exposed E, infected (infectious, symptoms) I, infected (infectious, no symptoms) A and recoved R, we indicate by H that there are compartiments to model the healthcare facilities (hospitalisation, ICUs), V means compartments for vaccinationated, and B means an eromental reservoir. The exponent denotes the multiplication of compartments to use the linear-chain trick.}
\end{table*}

Cholera is the focus of two additional chapters. The explanatory power of two influential models of cholera transmission and its link to rainfall are compared in \textsc{Chapter 4}, on an outbreak inJuba, South-Sudan. In \textsc{Chapter 5}, a scenario planning estimation of the probability of eliminating cholera from Haiti with a mass vaccination campaign is presented; this work has been carried within the framework of a multi-modeling study with other groups, and input from the ministry of health of Haiti.
\marginnote[-5\baselineskip]{Most bibliographic items and some additional precision are presented as margin notes for convenience. However, the full bibliography is included at the end of the thesis.}
The remaining chapters focuse on \textsc{covid}-19. In \textsc{Chapter 6} some facets of dealing with uncertainties from the early days of the pandemic are uncovered, with a focus on a pipeline for scenario modeling that has been used to inform gouvernements, in the United- States and other countries. 
An inference study to estimate the impact of the interventions against \textsc{covid}-19 in Switzerland is presented in \textsc{Chapter 7}.
%\marginnote[-5\baselineskip]{Incidentaly, this organisation reflects the chronogical order of the work.}
Finally, a tentative to connect an optimal control framework to an epidemiological model towards the allocation of \textsc{covid}-19 vaccines in Italy is presented in \textsc{Chapter 8}.

