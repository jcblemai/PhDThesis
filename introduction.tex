
\textit{Modeling} provides a formal framework to design and test such approaches. It encompasses uncertainties, transmission dynamics and intervention policies. Depending on its accuracy, a cholera model might be able to reproduce, and even forecast, in time and space the spread of the epidemic.

One of the main goals of epidemiological modeling is to predict the incidence of an ongoing epidemic in order to guide the public health officials in the deployment of life-saving medical supplies and medical staff. Moreover, an accurate model is a viable substitute for an experiment, where stakeholders can test intervention policies and scientific hypothesis on disease dynamics.

A novel use of epidemiological modelling is in the formal design of control policies. \textit{Optimal control} provides a rigorous framework to identify the most effective control measures under a set of operational constraints, in order to minimize an objective function of interest, like disease transmission and death. 

The construction of such a framework is the object of the present thesis. Building on the laboratory of Ecohydrology at EPFL (ECHO lab in the following) expertise on waterborne disease modeling, optimal control of cholera epidemic is a step forward in the ability for scientists to help policy makers in reducing the global cholera burden.

\subsection{Objectives}

The present thesis, developed within the framework of the Swiss National foundation project ``Optimal
control of intervention strategies for waterborne disease epidemics (SNF 200021--172578)'', aims at developing a decision support system to optimize cholera intervention strategies.

The  primary objective of this thesis is to design and implement an optimal control solver to be coupled with a spatially explicit model of cholera transmission, to compute optimal intervention strategies in real time for health care actors during outbreaks.

Towards an operational forecasting and decision framework, many obstacles must be overcome. The relevance of any control recommendation depends on the accuracy of the model forecasts. Thus, there exists a feedback loop between epidemiological modeling and control implementation. Since process-based epidemiological modeling for cholera is still at its infancy, the refinement of cholera models is essential to produce valid control policies.

The ECHO lab has been working on a spatially-explicit cholera model that has been able to match, and even forecast, the epidemiological curve across a number of settings. During the whole thesis, we will continue to improve this model, and to develop new features, relaxing assumptions and limitations of existing approaches. A major milestone is the creation of an operational forecasting platform for epidemic dynamics, to communicate results in a timely manner.

%The thesis aims at answering the following questions:
%\begin{itemize}
%    \item For a given setup, with operational constraints, what is the optimal way of %allocating control resources to minimize the impact of a cholera outbreak ?
%    \item Do general rules for intervention exists ? In this cases, what are these rules ?
%    \item How do we should allocate surveillance measure in order to best detect upcoming %outbreaks ?
%\end{itemize}

At the end of the thesis, the final deliverable will be a platform able to provide cholera forecasts and optimal control policies in quasi real time, to exploit the full potential of state-of-the-art spatially explicit cholera models.

\chapter{Infectious Disease Modeling}

\section{Infectious Diseases Epidemics} %as a phenomena)

\section{Modeling of Epidemics}
% (Bayesian philosophy, data, representation of reality) sochastic/determistic Modeling of epidemics (uncertainties, social factors, heterogeneities)
\subsection{Data}
%data on health

\subsection{Model design}
Compartimental + spatially explicit
\subsection{Inference and calibration} 
 Filtering
\subsection{Vizualisation and communication}
\subsection{Optimal control}



 \section{Control of epidemic, and modeling of control of epidemics}%Public-health decision making tools using models




