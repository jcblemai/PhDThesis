\chapter{Assessing the impact of non-pharmaceutical interventions on SARS-CoV-2 transmission in Switzerland}
\label{ch:covid-switzerland-npi}

Abstract
Following the rapid dissemination of COVID-19 cases in Switzerland, large-scale non-pharmaceutical interventions (NPIs) were implemented by the cantons and the federal government between February 28 and March 20. Estimates of the impact of these interventions on SARS-CoV-2 transmission are critical for decision making in this and future outbreaks. We here aim to assess the impact of these NPIs on disease transmission by estimating changes in the basic reproduction number (R0) at national and cantonal levels in relation to the timing of these NPIs. We estimate the time-varying R0 nationally and in eleven cantons by fitting a stochastic transmission model explicitly simulating within hospital dynamics. We use  individual-level data from more than 1,000 hospitalized patients in Switzerland and public daily reports of hospitalizations and deaths. We estimate the national R0 was 2.8 (95\% CI: 2.1-3.8) at the beginning of the epidemic. Starting from around March 7, we find a strong reduction in time-varying R0 with a 86\% median decrease (95\% quantile range, QR: 79\%-90\%) to a value of 0.40 (95\% QR: 0.3-0.58) in the period of March 29-April 5. At the cantonal level, R0 decreased over the course of the epidemic between 53\% and 92\%. Reductions in time-varying R0 were synchronous with changes in mobility patterns as estimated through smartphone activity, which started before the official implementation of NPIs. We inferred that most of the reduction of transmission is attributable to behavioural changes as opposed to natural immunity, the latter accounting for only about 4\% of the total reduction in effective transmission. As Switzerland considers relaxing some of the restrictions of social mixing, current estimates of time-varying R0 well below one are promising. However, as of April 24th 2020, at least 96\% (95\% QR: 95.7\%-96.4\%) of the Swiss population remains susceptible to SARS-CoV-2. These results warrant a cautious relaxation of social distance practices and close monitoring of changes in both the basic and effective reproduction numbers. 
Introduction
As of May 13, 2020, the ongoing coronavirus disease 2019 (COVID-19) pandemic caused by Severe Acute Respiratory Syndrome Coronavirus 2 (SARS-CoV-2) has resulted in more than 4.1 million cases and 280,000 deaths globally (WHO 2020). Independent estimates of the basic reproduction number R0 for SARS-CoV-2 from the initial phases of the epidemic in China, Europe and the US have generally ranged from 2-3 with doubling times on the order of 2-4 days. In response to rapid increase in reported cases and hospitalizations, most countries have implemented non-pharmaceutical interventions (NPIs) including compulsory face mask use, border and schools closures, quarantine of suspected and confirmed cases, up to population-wide home isolation (HIT COVID Team 2020). An observation study conducted in Hong Kong during this pandemic estimated that social distancing measures and school closures reduced COVID-19 transmission, as characterized by the effective reproductive number, by 44\% (Cowling et al. 2020). Comparable reductions have been observed in many different settings (Gatto et al. 2020; Imperial College COVID-19 Response Team 2020). Making decisions around relaxing NPIs requires both a careful assessment of the level of pre-relaxation transmission (e.g., R0) and quantification of the expected increase in transmission from relaxation of different NPIs.

From the initial reported case on the 24th of February through April 29, Switzerland reported more than 24,400 laboratory-confirmed COVID-19 cases and 1,408 official deaths affecting all 26 cantons (OFSP 2020b). The federal government issued a series of special decrees from February 28th banning gatherings of more than 1000 people culminating on March 20 with recommended home isolation (Figure 1). One month after the first NPI, daily confirmed case incidence has decreased from a peak of more than 1000 to a daily average of under 170 in the week of 20-26th of April (Figure 1). Initial reports have suggested a basic reproduction number (R0) of 3.5 at the start of the epidemic, with a decrease of 85\% by March 20 (Flaxman et al. 2020). However, these estimates, part of a multi-country analysis of NPIs, relied on death incidence and did not account for specificities of the hospitalization processes in Switzerland. Moreover, changes in R0 were imposed to be on the date of NPI implementation, thus not allowing for the exploration of the relative timing between NPIs and changes in transmission. Furthermore, such delays might bias the estimate R0.

NPIs affecting daily activities such as school closures and gathering bans aim at having a direct impact on mobility patterns to reduce potentially infectious social contact. In other terms, the causal pathway from NPIs to transmission reduction is mediated by changes in mobility. Recent releases of mobility data from smartphone software providers open the possibility to study the associations between the implementation of movement-limiting measures, behavioral change and the related changes in R0.  Context-specific data on the degree and speed of compliance to these types of NPIs and associations with the observed decreases in R0 could inform scenario-building should the future reinstatement of measures become necessary. Given that R0 directly represents transmission potential, its quantification enables us to estimate the proportion of reduction in transmission attribuable to behavioral changes. In this sense, tracking of R0 is more suited to study the impact of NPIs than the effective reproduction number (Reff), which is an aggregate measure of transmission capturing aspects of both infectious contacts and of population susceptibility.

Here, we aim at estimating changes in R0 over the course of the epidemic at both the national- and cantonal-levels using detailed data on hospitalizations and deaths from Switzerland between February 24th and April 24th. To begin to understand the estimated changes in R0, we explore its relationship with the timing of NPIs and human mobility estimates derived from cell phone data. 

Figure 1: COVID-19 epidemic curve in Switzerland and timing of non-pharmaceutical interventions.  Vertical dotted lines indicate the issuing of NPIs: 1) ban on gatherings of more than 1000 people, 2) school closure, 3) closure of non-essential activities, 4) ban on gatherings of more than 5 people.  Left: Daily case incidence at time of reporting along with death incidence. Right: Current hospitalizations, Intensive Care Units (ICUs) and cumulative deaths. Plotted data was taken from (Probst [2020] 2020) and may therefore present inconsistencies with official reports from the Federal Office of Public Health due to reporting delays. 

 


Methods
Model and assumptions
We developed a stochastic compartmental model of the COVID-19 epidemic and hospitalization processes in each canton of Switzerland. We structure our model around the classical S, E, I and R compartments (Kermack and McKendrick 1927). Namely, we consider that the population is divided in compartments depending on their status with regard to COVID-19. A susceptible individual (S) might be exposed after contact with infectious (I) individuals. Upon exposure, a formerly susceptible individual goes through an incubation period (E) before becoming infectious (I). The individual then recovers (R) and does not participate in transmission anymore. In addition to these dynamics, in the proposed framework after individuals are infected some proportion develop severe disease and among those, some are hospitalized (¡ compartments) and may advance to needing the intensive care unit (U compartments). Hospitalized can progress to discharge, ICU or death and those in the ICU (U compartments) can either be discharged or die (model diagram in SM Figure 4). The model is implemented as a hidden markov model using the pomp R package (King, Nguyen, and Ionides 2015).

We only use hospitalization and death data (see below), thus we do not need to explicitly model a latent stage where individuals are still asymptomatic but infectious (Ganyani et al. 2020; He et al. 2020; Liu et al. 2020). Instead, we parameterize the model conditioning on a mean generation time of 5.2 days (Ganyani et al.), and an exposed and non-infectious duration of 2.9 days (He et al. 2020), yielding a mean duration of 4.6 days in the infectious compartments. The exhaustive description of model transitions and parameters are presented in the Supplementary Material (SM) Table 3 of the Appendix. We assume that 7.5\% of infections are severe and would require hospitalization  that 50\% of deaths happen outside of hospitals (data from Vaud described in section 1 of the Appendix, and data from Geneva collected from OpenZH), that 16\% of those hospitalized will die (data from Vaud, SM Figure 2), and that the infection fatality ratio (IFR) is 0.75\%, which is in the range of published estimates (Verity et al. 2020; Russell et al. 2020). 

We used individual-level data on hospitalized cases from the canton of Vaud to estimate the distribution of time spent in the hospital and in the intensive care unit (ICU). We estimate times to discharge and death using survival models that account for right-censoring of observations (Section 1 of the Appendix, observed and estimated hospitalization durations in Tables SM 1 and SM 2).The number of compartments for each hospitalization state was based on distribution of timings from the Vaud data.
Data and Inference
We use curated data from OpenZH (openZH 2020) up to April 24. This dataset includes, by canton, the number of currently hospitalized COVID patients, and the cumulative numbers of deaths, cases and hospital discharges. The latter is not available for all cantons. For our cantonal estimate, we focused on cantons which had enough cases and data to obtain meaningful results, keeping eleven of the 26 cantons (Bern, Basel-Landschaft, Basel-Stadt, Fribourg, Geneva, Jura, Neuchâtel, Ticino, Vaud, Valais and Zürich). These canton account for 66\% of the Swiss population. Our national estimate uses curated national aggregate data and thus encompasses all cantons (Probst 2020).

We fit unknown parameters of our models using maximum likelihood inference through iterated filtering (Ionides et al. 2015). We did not attempt to include confirmed case data into the observation model due to heterogeneous testing strategies adopted across cantons and over time. We therefore fit the model to death incidence and changes in current hospitalizations using appropriate likelihood functions (details in section 3 of the Appendix). Hospitalization incidence data at the cantonal level would have provided more information, however we were unable to access these data. Fitting to hospitalization incidence data, which we had access to in the canton of Vaud, yielded similar results than when using changes in current hospitalizations (results not shown). 

Time-varying basic reproduction numbers R0 were estimated following a similar approach to Cazelles et al., 2018 (Cazelles, Champagne, and Dureau 2018), recently applied to COVID-19 transmission in Hubei (Kucharski et al. 2020). The method aims at inferring the time series of R0 that yield model dynamics that are in best agreement with the whole set of available observations (details in section 2 of the Appendix). As such, the value of R0 at a given point in time is informed by the whole data, and therefore does not have the limitations of being either “forward looking” or “backwards looking” as it is the case of commonly used statistical methods applied for this purpose (Wallinga and Teunis 2004; Cori et al. 2013). Once the time series is inferred, we assess the timing and slope of changes in R0 by using linear changepoint models (Lindeløv 2020). The null model corresponds to a linear decrease between two plateaus corresponding to the baseline value at the start of the epidemic and a low value after the implementation of NPIs. To allow for different slopes in the decreasing phase of R0, we also fit models with one and two additional breakpoints (corresponding to two and three different slopes), and the best model is selected using Bayesian model selection based on leave-one-out cross-validation (details in section 6 of the Appendix).  

We contrast the estimated changes in R0 with changes in activity-related mobility data produced by Google (Google LLC 2020). Changes in activity are expressed as relative changes with respect to a baseline computed as the median over a 5-week period from January 3 to February 6, 2020.  Mobility changes were computed for different categories: grocery \& pharmacy, parks, transit stations, retail \& recreation, residential and workplace. Mobility estimates are based on smartphone-based geo-location location data (GPS, WiFi connections, Bluetooth) from users who activated Location History for their Google Account. These data are used to determine changes in the  number of visits to and length of stays in locations categorized into the above-mentioned types. The dataset therefore only covers a sample of the Swiss population using smartphones, the latter representing around 80\% of the total population in 2020 (O’Dea 2020).  We use linear interpolation to fill gaps in the dataset, and we apply a 7-day moving average to smooth out weekly seasonality in activities. We compute the cross-correlation between changes in R0 and the averaged changes in each type of activity up to a lag of 10-days. Changes in R0 were computed based on location-specific baselines taken as the mean value of R0 from the beginning of the simulations, five days before the first reported case in each canton, until March 8th. As for R0 we employ changepoint models to identify dates of change in mobility patterns.

All data and code except for individual hospitalization data from Canton of Vaud have been deposited on Zenodo (doi).


Results
Overall we find that over the study period R0  trends followed a common trajectory nationally and across cantons, starting with a high plateau (R0 > 2) in the early stage of the epidemic followed by a rapid reduction starting beginning of March reaching a low and stable value (R0 < 1) from end of March onwards (Figure 2).

We estimate that at the beginning of the epidemic R0 was 2.8(95\% Confidence Interval [CI]: 2.06-3.83) at the national level, with cantonal-level values ranging from 2.5 to 3.1 (SM Table 5). The onset of reduction was estimated to be between March 4 (Basel Stadt and Vaud) and March 11 (Genève and Valais) at the cantonal level and on the 7th of March at the national level (SM Figure 11). Overall we found strong support for the reduction in R0 starting before school closures on March 13 (probability 0.99 at the national level, SM Figure 11). Once started, we estimate a strong decrease in R0 at the national (reduction of 0.16/day) and cantonal levels (between 0.14/day in Jura to 0.18/day in Basel-Landschaft) (Figure 2). We did not find strong support for changes of slope during the decrease phase neither at the national nor cantonal levels except for Bern Basel Stadt and Vaud for which additional changes in slopes were inferred towards the stabilization of R0 at low values (SM Table 7). We estimate that R0 in Switzerland dropped below 1 on March 19 (95\% CI: March 16-March 22) with individual cantons meeting this threshold between March 16 (Basel-Stadt) and March 20 (Neuchâtel) (SM Figure 10). We estimate the probability that R0 had already fallen below one was low when schools closed on March 13 (national: 0.006, cantons: from 0 in Geneva to 0.23nBasel-Landschaft), and high by the time gatherings of 5 people or more were banned on March 20 (national: 0.92, cantons: from 0.52 in Neuchâtel to 0.99 in Ticino) (SM Figure 9). The estimated plateau value of R0 after the reduction, i.e from March 29 to April 10, was of 0.4 (95\% quantile range [QR]: 0.3-0.6) at the national level, with median values at the cantonal level ranging from 0.2 to 0.7 (SM Table 5). At the national level  R0 was reduced by  86\% (95\% QR: 79\%-90\%), with median reductions ranging from 53\% (Jura) to 92\% (Basel-Stadt) at the cantonal level. A gradual reduction in R0 leading to values below one around the third week of March is consistent with the observed reduction of confirmed case incidence in early April, when we take into consideration the delays due to the incubation period, with median of 5.2 days (Lauer et al. 2020), and between symptom onset and reporting (Bi et al. 2020). Similarly, the inflection in the number of current hospitalizations and ICU usage in early April also supports R0 dropping below one in mid March.  

Figure 2: Estimates of changes in the basic reproduction number R0. Median (dashed line), IQR (dark gray) and the 95\% QR (light gray) of the estimated time series of R0 are shown for each canton. Vertical dotted lines indicate the issuing of NPIs as described in Figure 1. Transparency at the end of the time series indicates increasing uncertainty (style inspired by CMMID).


Activity-related mobility patterns changed markedly in all cantons since the beginning of the epidemic (Figure 3). Mobility related to work, retail and recreation, and transit stations dropped by 50 to 75\% at the national level, with cantonal-level reductions ranging from 30 to 80\% depending on activities. Residential-related mobility increased across cantons between 20\% and 30\%. We find strong support for mobility changes starting simultaneously for all activity types within each canton. We estimate that changes in mobility started between March 6 and 14 for all cantons (SM Figure 12), thus finding strong support for changes starting before school closure on March 13 (national-level mean probability across activities: 0.70, cantonal range: 0.55-0.99).

Based on our changepoint models, we find that reductions in R0 likely started (probability: 0.76) before observed reductions in mobility at the national level and across cantons (Figure 3). Changes in R0 were highly correlated with changes in mobility, the strongest associations being with mobility related to work, transit stations, retail and recreation, and residential (cross-correlations > 0.9 in all cantons and nationally, Figure 4). In the majority of cases, correlation between mobility and R0 was strongest with no lag between the two. However, changes in mobility to workplaces lagged behind changes in R0 in Basel-Stadt and Basel-Landschaft. Correlations between changes in R0 and grocery and pharmacy mobility were less marked (national level: 0.65), with changes in mobility occurring after changes in R0(negative lags in Figure 4). . In most cantons, a strong increase in park mobility after March 25 resulted in a positive correlation with changes in R0 but with negative lags (change in activity after change in R0, Figure 4). We did not find significant linear associations between the level of reduction in mobility and maximum reduction in R0 across cantons, except for a small effect size for reduced park mobility (regression coefficient of 0.15, 95\% CI: 0.02-0.25) (SM Figure 8).

Figure 3: Changes in mobility patterns and R0. Changes in mobility with respect to baseline are shown by activity type in terms of the daily values (transparent lines) and 7-day rolling mean (full lines), against the median estimate of R0 (black line). Vertical dotted lines indicate the issuing of NPIs: 1) ban on gatherings of more than 1000 people, 2) school closure, 3) closure of non-essential activities, 4) ban on gatherings of more than 5 people



Figure 4: Timing between changes in R0 and mobility. Left: probability that the first changepoint in R0 occured before the first changepoint in mobility-related activity. Right: Maximum Cross-correlations between time series of changes in R0 and changes in mobility (bars: 95\% CI). Lags refer to the delay between changes in mobility-related activity and changes in R0 (positive lag k indicates that current changes in mobility have maximal cross-correlation with changes in R0 k days in the past).



We estimated a commonly reported metric, the effective reproduction number (Reff), which is an aggregate measure of transmission capturing aspects of both infectious contacts and of population susceptibility. Across cantons, Reff was extremely close to R0 indicating that a small fraction of the population is expected to have natural immunity to SARS-CoV-2 (as we assumed was true in the short-term after infection). We estimate that as of April 24 3.9\% (95\% QR: 3.6\%-4.3\%) of the population nationally had been infected, with median estimates ranging from 1.9\% (Bern) to 16\% (Ticino) (Figure 5). Modelled estimates of the proportion infected of people infected in the canton of Genève are in agreement with preliminary results from ongoing serological studies which have estimated the seroprevalence to be 9.7\% (95\% CrI: 6.1\%-13.1\%) in the third week of April (Stringhini et al. 2020), compared to modeled estimates of 8.9\% (95\% QR: 7.8\%-10.1\%) after accounting for the time from infection to seroconversion (Wölfel et al. 2020) (SM Figure 7). 

Figure 5: Modelled proportion of people infected with SARS-CoV-2 in Switzerland up to April 24th. Estimates were produced for 12 of 26 cantons for which enough data were available (unmodelled cantons are shown in gray with points indicating the national-level estimated incidence proportion of 3\%). Values are reported in SM Table 6 of the Appendix. 

Discussion
Our results suggest a strong reduction of R0 across Switzerland since the start of the epidemic. The reduction in R0 started around March 7, thus about one week before the implementation of lockdown-type NPIs. Analysis of activity-related mobility data also showed strong support for changes in mobility starting prior to the implementation of most NPIs. Estimated  reductions of viral transmission were strongly correlated and mostly synchronous with observed changes in mobility patterns, although the initiation of changes in transmission preceded measurable changes in activity-related mobility. 

The methods used to infer the time series of R0 do not rely on assumptions on the shape of how it changed in time, nor on the dates at which change started. Alternative methods that rely on fixed dates (such as Imperial College COVID-19 Response Team 2020) might be biased as changes in transmission are not synchronous with policy changes. Distribution based methods such as provided by R package EpiEstim (Wallinga and Teunis 2004; Cori et al. 2013) are flexible but subject to bias when misused (Lipsitch, Joshi, and Cobey 2020). In addition our approach enables the estimation of R0 which is a direct quantification of transmission potential, as opposed to the effective reproduction number Reff, which also accounts for the effect of susceptible depletion as done in the above-mentioned statistical approaches. This enables us to estimate the proportion of reduction in transmission attribuable to behavioral changes, which is therefore more suited to study the impact of NPIs. Aside from these methodological differences, our estimates are in line with other estimates in Switzerland:  Althaus et al. estimate a reduction of 89 \%  (83\% - 94\%) from a baseline of 2.78 (2.51 - 3.11) (Althaus 2020), Scire et al. estimate a reduction of 76 \% (70\% - 82\%) from a baseline of 1.88 (1.80 - 1.98) (Scire et al. 2020) and Imperial College estimates a reduction of 60\% (50\% - 80\%) from a baseline of 3.5 (2.8 - 4.3) (Flaxman et al. 2020). 

Our results provide strong support for a reduction of transmission starting about 1 week prior to school closure, the first national-level NPI targeting daily activities, which was ordered on the 13th of March. Moreover, initiation of transmission reduction was found to precede changes in mobility patterns as detectable from the Google dataset. A possible explanation for this initial decrease in transmission could be linked to the strong increase in public interest in COVID-19 in February as measured by Google searches for COVID-19-related keywords (SM Figure 14). In fact, we estimated that a second sharp rise in Google searches started on March 7 (95\% CrI: March 3-March 9) which overlaps with our estimated start of national level decrease in R0 on March 7 (probability that changepoints coincide: 0.76). The Federal Office for Public Health issued an information campaign on COVID-19 on February 28th, which was updated on March 2 stressing basic hygiene rules (OFSP 2020a). This may have resulted in voluntary social distancing as well as increased hygiene early on without noticeable changes in mobility patterns. This type of proactive changes in behavior would be in line with early changes in mobility patterns, which were estimated to precede school closures and subsequent measures.

Our results suggest that the value of R0 was likely already below one on March 20 when the federal government banned gatherings of more than 5 people and recommended voluntary home isolation for the whole population. This result should however be taken within context as the announcement was anticipated on social networks earlier that week, thus probably already impacting social distancing behavior. We therefore warrant caution in any causal interpretation of our results on the role of this last NPI on driving R0 below one.

Despite the strong association between the changes in mobility and reductions in R0 within each canton, the lack of cross-cantonal associations between the level of reduction in mobility types and the level of reduction in R0 suggests context-specific pathways between COVID-19 transmission and mobility intensity. This warrants caution in attempting to apply general relations between mobility and transmission reduction. Investigating general associations will require more in-depth studies controlling for other factors such as population density, economic activities and social mixing patterns, inter-cantonal mobility patterns, in addition to incorporating potential environmental drivers of transmission such as temperature and relative humidity (Neher et al. 2020; Kissler et al. 2020).

We note several limitations to this work. First, due to the relatively recent introduction of SARS-CoV-2 in Switzerland compared to the length of hospital and ICU stays, the time distribution of  in- and out-of hospital patients is biased towards shorter duration (SM Figure 3), which we addressed by accounting with right-censoring using survival models. In addition, due to the limited data available in some places, we were only able to fit our model for twelve of the twenty-six cantons. Modeling results presented in this work are subject to our hypothesis on yet uncertain parameters of COVID-19 including the infection fatality rate and the proportion of severe infections requiring hospitalization. An important uncertainty is the fraction of asymptomatic infections and their relative contribution to disease transmission. We currently assume that all infected individuals contribute equally to transmission, which means our estimate of the proportion of people infected would under-estimate true cumulative incidence if there is a large fraction of asymptomatic infections with relatively low contribution to transmission. Evidence from South Korea however suggests that only a small fraction (2\%) of confirmed COVID-19 infections are totally asymptomatic, and none of the household members of these asymptomatic carriers were infected (Park et al. 2020). Moreover, model results are in agreement with preliminary results from ongoing serological studies in Switzerland (Stringhini et al. 2020). Moreover our estimates of time-varying basic reproduction numbers assume that the generation interval for COVID-19 in Switzerland remained unchanged, thus potentially ignoring the joint role of R0, the infectious period and contact rates in determining the disease’s intrinsic growth rate (Yan 2008). Assuming that the generation interval increased with the reduction of social contact, our estimates are conservative over-estimates of the “true” value of R0, which is encouraging from a public health perspective. Inferred disease dynamics and estimated time-varying R0also depend on the values of the incubation period, which we have set to the estimates currently available in the literature. In our modeling framework, we estimate the initial conditions along with changes in R0, which could however be influenced by the role of imported cases in driving disease dynamics especially in cantons bordering regions with strong COVID-19 transmission in early February (Eastern France for Basel Stadt and Basel Landschaft and Northern Italy for Ticino). Since we do not model importations, this could yield an over-estimation of the initial value of R0, which warrants caution in interpreting specific values of R0 in these cantons. This potential overestimation would however not affect the strong inferred reduction in R0. Another limitation of our study is that it was not possible to disentangle the individual contribution of each NPI on R0 in this analysis due to the early onset of changes in R0 and in mobility patterns as well as the very close spacing between the different types of NPIs. This information would however be extremely valuable in supporting decisions on NPI strategies against COVID-19. Efforts to constitute a global database of NPIs will open the opportunity to extend this type of analysis to other settings and produce evidence for the effect of different types of NPIs (HIT COVID Team 2020).

As the Swiss government plans to gradually lift restrictions, close monitoring of changes in R0 is critical given that the reductions in transmission appear to be almost entirely driven by changes in behavior, not through herd immunity. Near real-time estimates of R0 may serve as a critical tool for public health and political decision makers in the months to come and efforts should be made to refine models like ours using new data, including those from population-based serologic studies, mobility data and more detailed individual-level data on COVID-19 cases across the spectrum of severity.

Figure 1: COVID-19 epidemic curve in Switzerland and timing of non-pharmaceutical interventions.  Vertical dotted lines indicate the issuing of NPIs: 1) ban on gatherings of more than 1000 people, 2) school closure, 3) closure of non-essential activities, 4) ban on gatherings of more than 5 people:  Left. Daily case and death incidence. Right: Current hospitalizations, Intensive Care Units (ICUs) and cumulative deaths. Plotted data was taken from (openZH 2020) and may therefore present inconsistencies with official reports from the Federal Office of Public Health due to reporting delays.

Figure 2: Estimates of changes in the basic reproduction number R0. Median (dashed line), IQR (dark gray) and the 95\% QR (light gray) of the estimated time series of R0 are shown for each canton. Vertical dotted lines indicate the issuing of NPIs as described in Figure 1. Transparency at the end of the time series indicates increasing uncertainty (style inspired by CMMID).

Figure 3: Changes in mobility patterns and R0. Changes in mobility with respect to baseline are shown by activity type in terms of the daily values (transparent lines) and 7-day rolling mean (full lines), against the median estimate of R0 (black line). Vertical dotted lines indicate the issuing of NPIs as described in Figure 1.

Figure 4: Timing between changes in R0 and mobility. Left: probability that the first changepoint in R0 occured before the first changepoint in mobility-related activity. Right: Maximum Cross-correlations between time series of changes in R0 and changes in mobility (bars: 95\% CI). Lags refer to the delay between changes in mobility-related activity and changes in R0 (positive lag k indicates that current changes in mobility have maximal cross-correlation with changes in R0 k days in the past).

Figure 5: Modelled proportion of people infected with SARS-CoV-2 in Switzerland up to April 24th. Estimates were produced for 12 of 26 cantons for which enough data were available (unmodelled cantons are shown in gray with points indicating the national-level estimated incidence proportion of 3\%). Values are reported in SM Table 6 of the Appendix. 



