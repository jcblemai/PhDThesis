\begin{fullwidth}
\chapter[Assessing the impact of non-pharmaceutical interventions on SARS-CoV-2 transmission in Switzerland]{Assessing the impact of non-pharmaceutical\\ interventions on SARS-CoV-2 transmission\\ in Switzerland}
\chaptermark{Assessing the impact of non-pharmaceutical interventions on covid-19 transmission}
\label{ch:covid-switzerland-npi}

Following the rapid dissemination of \textsc{covid}-19 in Switzerland, large-scale non-pharmaceutical interventions (NPIs) were implemented by the cantons and the federal government between February 28 and March 20, 2020. Estimates of the impact of these interventions on SARS-CoV-2 transmission are critical for decision making in the present and future outbreaks. This chapter aims to assess the impact of these NPIs on disease transmission by estimating changes in the basic reproduction number $R_0$ at national and cantonal levels in relation to the timing of these NPIs. For the whole country and eleven cantons, the time-varying $R_0$ is estimated by fitting a stochastic transmission model explicitly simulating within-hospital dynamics. The dataset includes individual-level data from more than 1000 hospitalized patients in Switzerland and public daily reports of hospitalizations and deaths. The national $R_0$ is estimated to be 2.8 (95\% confidence interval 2.1–3.8) at the beginning of the epidemic. Starting from around March 7, a strong reduction of the time-varying $R_0$ is found, with an 86\% median decrease (95\% quantile range [QR] 79–90\%) to a value of 0.40 (95\% QR 0.3–0.58) in the period of March 29 to April 5. At the cantonal level, $R_0$ decreased by between 53\% and 92\% over the course of the epidemic. Reductions in time-varying $R_0$ were synchronous with changes in mobility patterns as estimated through smartphone activity, which started before the official implementation of NPIs. Most of the reduction of transmission is inferred to be attributable to behavioral changes as opposed to natural immunity, the latter accounting for only about 4\% of the total reduction in effective transmission. As Switzerland considers relaxing some of the restrictions of social mixing, current estimates of time-varying $R_0$ well below one are promising. However, as of April 24, 2020, at least 96\% (95\% QR 95.7–96.4\%) of the Swiss population remains susceptible to SARS-CoV-2. These results warrant a cautious relaxation of social distance practices and close monitoring of changes in both the basic and effective reproduction numbers.

This chapter is based on:
\longfullcite{Lemaitre:AssessingImpactNonpharmaceutical:2020}, and J. Perez-Saez shares co-authorship of the work. It  is referred in the following as the postprint (and its supplementary information as \textsc{si}).
\end{fullwidth}

\section{Introduction}
As of May 13, 2020, the ongoing coronavirus disease 2019 (\textsc{covid}-19) pandemic caused by severe acute respiratory syndrome coronavirus 2 (SARS-CoV-2) has resulted in more than 4.1 million cases and 280,000 deaths globally\cite{WHO:WHOSituationReport:2020}. Independent estimates of the basic reproduction number $R_0$ for SARS-CoV-2 from the initial phases of the epidemic in China, Europe, and the United States have generally ranged from 2–3\cite{Riou:PatternEarlyHumantohuman:2020} with doubling times on the order of 2–4 days. In response to the rapid increase in reported cases and hospitalizations, most countries have implemented non-pharmaceutical interventions (NPIs), including compulsory face mask use, border and school closures, quarantine of suspected and confirmed cases, up to population-wide home isolation\cite{HITCOVIDTeam:HealthInterventionsTracking:2020}. An observation study conducted in Hong Kong during this pandemic estimated that social distancing measures and school closures reduced \textsc{covid}-19 transmission, as characterized by the effective reproductive number, by 44\%\cite{Cowling:ImpactAssessmentNonpharmaceutical:2020}. Comparable reductions have been observed in many different settings\cite{Flaxman:Report13Estimating:2020}. Making decisions around relaxing NPIs requires both a careful assessment of the level of pre-relaxation transmission (e.g., $R_0$) and quantification of the expected increase in transmission from the relaxation of different NPIs. 

From the initial reported case on February 24 to April 29, Switzerland reported more than 24'400 laboratory confirmed COVID-19 cases and 1'408 official deaths affecting all 26 cantons\cite{OFSP:RapportSituationEpidemiologique:2020}. The federal government issued a series of special decrees from February 28 banning gatherings of more than 1'000 people culminating on March 20 recommended home isolation (fig.~\ref{fig:covid-ch-data}). One month after the first NPI, daily confirmed case incidence had decreased from a peak of more than 1000 to a daily average of under 170 in the week of April 20–26 (fig.~\ref{fig:covid-ch-data}). Initial reports have suggested a basic reproduction number ($R_0$) of 3.5 at the start of the epidemic, with a decrease of 85\% by March 20\cite[-3\baselineskip]{Flaxman:Report13Estimating:2020}.
\begin{figure*}\centering
  \includegraphics[width=\textwidth]{fig_covid-switzerland-npi/FIGURE_1.png}
  \caption[\textsc{covid}-19 epidemic curve in Switzerland and timing of interventions]{\textsc{covid}-19 epidemic curve in Switzerland and timing of non-pharmaceutical interventions. Dotted lines indicate the issuing of NPIs: (1) ban on gatherings of more than 1'000 people, (2) school closure, (3) closure of non-essential activities, and (4) ban on gatherings of more than five people. Left: Daily case incidence at time of reporting along with death incidence. Right: Current hospitalizations, intensive care units (ICUs), and cumulative deaths. Data from \textcite{Probst:DaenuprobstCovid19casesswitzerland:2020}, and may therefore present inconsistencies with official reports from the Federal Office of Public Health.% due to reporting delays.
  }
  \label{fig:covid-ch-data}
\end{figure*}
 However, these estimates, part of a multicountry analysis of NPIs, relied on death incidence and did not account for specifics of the hospitalization processes in Switzerland. Moreover, changes in $R_0$ were assumed to happen on the date of a NPI implementation, thus not allowing for the exploration of the relative timing between NPIs and changes in transmission. Furthermore, such delays likely bias the estimate $R_0$. 

NPIs affecting daily activities such as school closures and gathering bans aim at having a direct impact on mobility patterns to reduce potentially infectious social contacts. In other terms, the causal pathway from NPIs to transmission reduction is mediated by changes in mobility. Recent releases of mobility data from smartphone software providers give the possibility to study the associations between the implementation of movement-limiting measures, behavioral change, and the related changes in $R_0$. Context-specific data on the degree and speed of compliance with these types of NPIs and associations with the observed decreases in $R_0$ could inform scenario-building should the future reinstatement of measures become necessary. Given that $R_0$ directly represents transmission potential, its quantification enables us to estimate the proportion of reduction in transmission attributable to behavioral changes. In this sense, tracking of $R_0$ is more suited to study the impact of NPIs than the effective reproduction number $R_{eff}$, which is an aggregate measure of transmission capturing aspects of both infectious contacts and of population susceptibility. 

Here, the goal is to estimate the changes in $R_0$ throughout the epidemic at both the national and cantonal levels using detailed data on hospitalizations and deaths from Switzerland between February 24 and April 24. In order to understand the estimated changes in $R_0$, its relationship with the timing of NPIs and human mobility estimates derived from cell phone data is explored.

\section{Methods}
The setting for the present chapter arose while providing scenario modeling reports, as presented in \textsc{Chapter~4}. CHUV, the hospital coordinating Canton de Vaud's response, provided access to individual hospitalization data. The dataset and pre-processing steps to account for right-censoring are presented in the \textsc{Appendix to Chapter~5}. This dataset allowed for refined estimates of stays in hospital and death processes, which in turn were used as assumptions in a \textsc{covid}-19 model; giving said model sufficient stiffness to estimate $R_0$ in time. Model equations and fixed and inferred parameter values are left in the postprint \textsc{si}.

\subsection{Model and assumptions}
\paragraph{Model Structure} A stochastic compartmental model of the \textsc{covid}-19 epidemic and hospitalization processes is developed for each canton of Switzerland. The model is structured around the classical S, E, I, and R compartments\cite{Kermack:ContributionMathematicalTheory:1927}, and the model diagram is shown in fig.~\ref{fig:covid-ch-diagram}. %Namely, the population is divided into compartments depending on their status with regard to \textsc{covid}-19. 
A susceptible individual $S$ might be exposed after contact with infectious $I$ individuals. Upon exposure, a formerly susceptible individual goes through an incubation period $E$ before becoming infectious $I$. The individual then recovers $R$ and does not participate in transmission anymore. 
%In addition to these dynamics, some proportion of infected individuals develop severe disease and among those, some are hospitalized and may advance to needing the intensive care unit. 
In addition to these dynamics, infected individuals have some probability of developing severe symptoms. Estimates derived from data from the Canton de Vaud, as shown in the \textsc{Appendix}, show a high proportion of deaths outside of hospitals ($\approx 50\%$) hence two pathways are modeled depending if the individual seek or has access to hospital care.  Some severely infected will be treated in hospitals after a delay from symptom onset $I_h$. In this case, hospitalization leads to discharge (recovery) or death, either through normal hospitalization ($H_{s}$ and $H_d$ respectively) or passing through Intensive Care Units (ICUs, compartments $U_{s}$ and $U_d$ respectively). Otherwise, the severly infected may recover or die without passing through hospitalization, going into compartment $I_d$. There is no need to explicitly model a latent stage where individuals are still asymptomatic but infectious\cite[-8\baselineskip]{Ganyani:EstimatingGenerationInterval:2020,He:TemporalDynamicsViral:2020, Liu:ContributionPresymptomaticInfection:2020} as only hospitalisation and death data is used. The model is implemented as a hidden Markov model using the pomp R package\cite{King:StatisticalInferencePartially:2015}. 
 \begin{figure}[!htb]
\begin{center}
\includegraphics{fig_covid-switzerland-npi/fig_supp/diagram.png}
\caption[Schematic diagram of \textsc{covid}-19 transmission and hospitalization processes]{Schematic diagram of \textsc{covid}-19 transmission and hospitalization processes. There are two sinks: Death $D$ and recovered $R$. Each stage with regard to the disease may be implemented with several compartments (subscript numbered boxes) to better represent the time distribution spent in that stage.}
\label{fig:covid-ch-diagram}
\end{center}
\end{figure}

\paragraph{Assumptions} The reader is referred to the postprint \textsc{si} for the exhaustive description of model transitions and parameters. As parameter identifiability is needed to capture the dynamics of $R_0$, most of the parameters were fixed to values from the literature or obtained analyzing the data from Canton de Vaud\footnote[][-3\baselineskip]{As noted by Gostic et al., the accuracy of $R_0$ estimates obtained with the method presented here is sensible to assumptions on model structure and parameters; see \fullcite{Gostic:PracticalConsiderationsMeasuring:2020a}.}.
The transmission model is parameterized assuming a mean generation time of 5.2 days\cite{Ganyani:EstimatingGenerationInterval:2020}, and an exposed and non-infectious duration of 2.9 days\cite{He:TemporalDynamicsViral:2020}, yielding a mean duration of 4.6 days in the infectious compartments.  It is assumed that 7.5\% of infections were severe and would require hospitalization, that 50\% of deaths happened outside of hospitals (from data on Canton de Vaud described in the \textsc{Appendix}, and data from Geneva collected from OpenZH), that 16\% of those hospitalized would die (data from Vaud, see \textsc{Appendix}), and that the infection fatality ratio (IFR) was 0.75\%, which is in the range of published estimates\cite{Verity:EstimatesSeverityCoronavirus:2020, Russell:EstimatingInfectionCase:2020}. Individual-level data on hospitalised cases from the canton of Vaud is used to estimate the distribution of time spent in the hospital and in the intensive care unit (ICU). Times to discharge and death are estimated using survival models that account for right-censoring of observations (see \textsc{Appendix}). The number of compartments for the linear-chain trick (\ie the shape parameter for the erlang distributed residence time) for each observable hospitalization state was obtained by fitting Erlang distributions to the data of Canton de Vaud. To account for right-censoring the model is fitted to the estimated log-normal distributions described in the \textsc{Appendix} instead directly to observed times to events but rather. The rate parameter of the Erlang distributions is calibrated for shape parameters between $1$ and $10$ by minimizing the Kullback-Leibler (KL) divergence between the Erlang and estimated log-normal distributions. The final fit was taken to be the one with the smallest KL-divergence. We found that all hospitalization processes were best represented with exponential distributions (Erlang with a shape parameter of $1$)\footnote{In the first preprinted version without the survival analysis, the shape parameters were estimated to be of 2 for many compartments. It highlights the importance of reasoning about potential biases and of careful data pre-processing.}.

\subsection{Data and Inference} 
\paragraph{Dataset} Curated data from OpenZH\cite[3\baselineskip]{openZH:OpenZHCovid19:2020} up to April 24 is used. This dataset included, by canton, the number of hospitalised \textsc{covid} patients, and the cumulative numbers of deaths, cases and hospital discharges; the latter not available for all cantons. The cantonal estimate is focused on cantons that had enough cases and data to obtain meaningful results, keeping 11 of the 26 cantons (Bern, Basel-Landschaft, Basel-Stadt, Fribourg, Geneva, Jura, Neuchâtel, Ticino, Vaud, Valais and Zurich). These cantons account for 66\% of the Swiss population. The national estimate uses curated national aggregate data and thus encompassed all cantons\cite{Probst:DaenuprobstCovid19casesswitzerland:2020}. Unknown parameters of the model are fitted using maximum likelihood inference through iterated filtering\cite{Ionides:InferenceDynamicLatent:2015}. No attempt to include confirmed case data into the observation model were made because of the heterogeneous testing strategies adopted across cantons and over time. The model is therefore fitted to death incidence and changes in current hospitalisations using appropriate likelihood functions:
\begin{equation}
\begin{split}
 \text{deaths}(t) &\sim \text{Poisson}(\Delta D(t)) \\
\Delta  \text{hosp}(t) &\sim \text{Skellam}(\Delta H(t), \Delta D_H(t) + \Delta R_H(t)), \\
\end{split}
\end{equation}
\noindent where, $\Delta D(t)$, $\Delta H(t)$, $\Delta D_H(t)$, $\Delta R_H(t)$ are respectively the number of new deaths, hospitalized, and deaths and discharged from hospitals at time $t$, and $\Delta \text{hosp}(t)$ is the difference between the number of current hospitalizations at times $t$ and $t-1$, for which a Skellam distribution\footnote{which represents the difference between two independent random variables, each Poisson-distributed. See \fullcite{Skellam:FrequencyDistributionDifference:1946}.} is choosen. The full log-likelihood of the observation model was taken as the sum of the individual log-likelihoods of the $\Delta \text{hosp}(t)$ and of the $\text{deaths}(t)$. 
The model is calibrated separately for each canton on the daily death and hospitalization until April 24. Hospitalization incidence data at the cantonal level would have provided more information, but this data was not accessible. Fitting to hospitalization incidence data, which access was granted in the canton of Vaud, yielded similar results to those when changes in current hospitalizations were used. The majority of model parameters were either derived from Vaud data or the literature (see postprint \textsc{si}), which enables the identifiability of the reproduction number $R_0$.
 
 \paragraph{Reproduction number estimation} Time-varying basic reproduction numbers $R_0$ were estimated following a similar approach to Cazelles et al.\cite{Cazelles:AccountingNonstationarityEpidemiology:2018}, recently applied to \textsc{covid}-19 transmission in Hubei\cite{Kucharski:EarlyDynamicsTransmission:2020}. The method aims at inferring the time series of $R_0$\footnote{As mentioned, $R_0$, not $R_{eff}$ is computed as the presented method estimates the basic reproduction number, \ie the expected number of infections generated by one infected individual in a fully susceptible population. To do so, the susceptible and recovered populations are explicitly modeled while other methods usually compute the effective reproduction number in the population by deconvolution of the observed incidence.} that yields model dynamics that are in best agreement with the whole set of available observations. As such, the value of $R_0$ at a given point in time is informed by the whole data, and therefore does not have the limitations of being either “forward-looking” or “backwards-looking” as it is the case of commonly used statistical methods applied for this purpose\cite{Wallinga:DifferentEpidemicCurves:2004,Cori:NewFrameworkSoftware:2013,Lipsitch:CommentPanLiu:2020}. 
 
The model equations are similar to the previously presented partially-observed Markov Processes models, and are left in the postprint \textsc{si}. The difference lies in the force of infection. Given the state of the system at time \(t\), \(\mathcal{X}_t\), and using the same notations as in \textsc{Chapter~3}, the transition $S \longrightarrow E$ reads:

\begin{gather}
\begin{aligned}
    \mathbb{P}\left[ \Delta N_{SE}(t) = 1 \mid\mathcal{X}_t\right] &=  \underbrace{\beta(t)  \frac{I_1(t) + I_2(t) + I_3(t)}{P}}_{\text{Force of infection}} S(t) \Delta t + o(\Delta t).\\
    \end{aligned}
\end{gather}
Time-varying $R_0(t) = \beta(t)/(3r_I)$ is modelled as a geometric random walk defined by its calibrated variance, where $\beta$ is the transmission parameter and $1/(3r_I)$ is the mean duration spent in the infectious compartments $I_1$ to $I_3$. Once the time series is inferred, the timing and slope of changes in $R_0$ are evaluated by using linear changepoint models\cite{Lindelov:McpPackageRegression:2020}. The null model corresponded to a linear decrease between two plateaus corresponding to the baseline value at the start of the epidemic and a low value after the implementation of NPIs. To allow for different slopes in the decreasing phase of $R_0$, models with one and two additional breakpoints (corresponding to two and three different slopes) are also fitted, and the best model was selected using Bayesian model selection based on leave-one-out cross-validation (details in section 6 of the postprint \textsc{si}). 

The estimated changes in $R_0$  were contrasted with changes in activity-related mobility data produced by Google\cite{GoogleLLC:GoogleCOVID19Community:2020}. Changes in activity were expressed as relative changes with respect to a baseline computed as the median over a 5-week period from January 3 to February 6. Mobility changes were computed for different categories: grocery and pharmacy, parks, transit stations, retail and recreation, residential, and workplace. Mobility estimates were based on smartphone-based geo-location location data (GPS, WiFi connections, Bluetooth) from users who activated Location History for their Google Account. These data were used to determine changes in the number of visits to and length of stays in locations categorized into the above-mentioned types. The dataset therefore only covers a sample of the Swiss population who use smartphones, the latter representing around 80\% of the total population in 2020\cite{ODea:SmartphoneUsersSwitzerland:2020}. Gaps in the dataset were filled by linear interpolation and a 7-day moving average was applied to smooth out weekly seasonality in activities. The cross-correlation between changes in $R_0$ and the averaged changes in each type of activity was computed with lags up to 10-days. Changes in $R_0$ were computed based on location-specific baselines taken as the mean value of $R_0$ from the beginning of the simulations, 5 days before the first reported case in each canton, until March 8. Changepoint models are employed to identify dates of change in mobility patterns and in $R_0$. \marginnote[-5\baselineskip]{All data and code except for individual hospitalization data from the canton of Vaud have been deposited on Zenodo (\url{doi.org/10.5281/zenodo.3862075}). The change point analysis, model fit and additional results are omitted from this thesis and may be found in the supplementary information of \parencite{Lemaitre:AssessingImpactNonpharmaceutical:2020}.}


\section{Results}
\begin{figure*}\centering
  \includegraphics[width=\textwidth]{fig_covid-switzerland-npi/FIGURE_2.png}
  \caption[Estimates of changes in the basic reproduction number $R_0$][-1\baselineskip]{Estimates of changes in the basic reproduction number $R_0$. Median (dashed line), IQR (dark gray), and the 95\% QR (light gray) of the estimated time series of $R_0$ are shown for each canton. Vertical dotted lines indicate the issuing of NPIs as described in fig.~\ref{fig:covid-ch-data}. Transparency at the end of the time series indicates increasing uncertainty (style inspired by the reports of CMMID.}
  \label{fig:covid-ch-r0}
\end{figure*}
Over the study period, $R_0$ trends follow a common trajectory nationally and across cantons, starting with a high plateau ($R_0 >2$) in the early stage of the epidemic followed by a rapid reduction starting at the beginning of March, and reaching a low and stable value ($R_0 <1$) from the end of March onwards (fig.~\ref{fig:covid-ch-r0}). 

At the beginning of the epidemic, $R_0$ is estimated at 2.8 (95\% confidence interval [CI] 2.06–3.83) at the national level, with cantonal-level values ranging from 2.5 to 3.1 (postprint \textsc{si} tab. 5). The onset of the reduction was estimated to be between March 4 (Basel-Stadt and Vaud) and March 11 (Geneva and Valais) at the cantonal level and on March 7 at the national level (postprint \textsc{si} fig. 11). Overall strong support is found for the reduction in $R_0$ starting before school closures on March 13 (probability 0.99 at the national level, postprint \textsc{si} fig. 11). Once started, a strong decrease in $R_0$ is estimated at the national (reduction of 0.16/day) and cantonal levels (between 0.14/day in Jura to 0.18/day in Basel-Landschaft) (fig.~\ref{fig:covid-ch-r0}). No strong support was found for changes of the slope during the decrease phase at either at the national or cantonal levels except for Bern, Basel-Stadt, and Vaud, for which additional changes in slopes were inferred towards the stabilization of $R_0$ at low values (postprint \textsc{si} tab. 7).\marginnote[0\baselineskip]{Additional results may be found in the supplementary information of \fullcite{Lemaitre:AssessingImpactNonpharmaceutical:2020}.} $R_0$ in Switzerland has been estimated to drop below 1 on March 19 (95\% CI March 16–22) with individual cantons meeting this threshold between March 16 (Basel-Stadt) and March 20 (Neuchâtel) (postprint \textsc{si} fig. 10). The probability that $R_0$ had already fallen below one was low when schools closed on March 13 (national 0.006, cantonal from $0$ in Geneva to 0.23 in Basel-Landschaft), and high by the time gatherings of five people or more were banned on March 20 (national 0.92, cantonal from 0.52 in Neuchâtel to 0.99 in Ticino) (postprint \textsc{si} fig. 9). The estimated plateau value of $R_0$ after the reduction, that is, from March 29 to April 10, was 0.4 (95\% quantile range [QR] 0.3–0.6) at the national level, with median values at the cantonal level ranging from 0.2–0.7 (postprint \textsc{si} tab. 5). At the national level, $R_0$ was reduced by 86\% (95\% QR 79–90\%), with median reductions ranging from 53\% (Jura) to 92\% (Basel-Stadt) at the cantonal level. A gradual reduction in $R_0$ leading to values below one around the third week of March is consistent with the observed reduction of confirmed case incidence in early April, when taking into consideration the delays due to the incubation period, with a median of 5.2 days\cite[-6\baselineskip]{Lauer:IncubationPeriodCoronavirus:2020}, and between symptom onset and reporting\cite[-2\baselineskip]{Bi:EpidemiologyTransmissionCOVID19:2020}. Similarly, the inflection in the number of current hospitalizations and ICU usage in early April also supports $R_0$ dropping below one in mid-March. 

\begin{figure*}\centering
  \includegraphics{fig_covid-switzerland-npi/FIGURE_3.png}
  \caption[Changes in mobility patterns and $R_0$][-1\baselineskip]{Changes in mobility patterns and $R_0$.Changes in mobility with respect to baseline are shown by activity type in terms of the daily values (transparent lines) and 7-day rolling mean (full lines), against the median estimate of $R_0$ (black line). Vertical dotted lines indicate the issuing of NPIs as described in fig.~\ref{fig:covid-ch-data}.}
  \label{fig:covid-ch-mobility}
\end{figure*}

Activity-related mobility patterns changed markedly in all cantons since the beginning of the epidemic (fig.~\ref{fig:covid-ch-mobility}). Mobility related to work, retail and recreation, and transit stations dropped by 50\% to 75\% at the national level, with cantonal-level reductions ranging from 30\% to 80\% depending on activities. Residential-related mobility increased across cantons between 20\% and 30\%. Strong support is found for mobility changes starting simultaneously for all activity types within each canton. Changes in mobility are estimated to have started between March 6 to 14 for all cantons (postprint \textsc{si} fig. 12), thus finding strong support for changes starting before school closure on March 13 (national-level mean probability across activities 0.70, cantonal range 0.55–0.99). 

Based on the changepoint models, reductions in $R_0$ likely started (probability 0.76) before observed reductions in mobility at the national level and across cantons (fig.~\ref{fig:covid-ch-mobility}). Changes in $R_0$ were highly correlated with changes in mobility, the strongest associations being with mobility related to work, transit stations, retail and recreation, and residential (cross-correlations >0.9 in all cantons and nationally, fig.~\ref{fig:covid-ch-timing}). 
\begin{figure*}\centering
  \includegraphics[width=\textwidth]{fig_covid-switzerland-npi/FIGURE_4.png}
  \caption[Timing between changes in $R_0$ and mobility][-2\baselineskip]{Timing between changes in $R_0$ and mobility. Left: the probability that the first changepoint in $R_0$ occurred before the first changepoint in mobility-related activity. Right: Maximum cross-correlations between time series of changes in $R_0$ and changes in mobility (bars 95\% CI). Lags refer to the delay between changes in mobility-related activity and changes in $R_0$ (positive lag k indicates that current changes in mobility have maximal cross-correlation with changes in $R_0$ $k$ days in the past).}
  \label{fig:covid-ch-timing}
\end{figure*}
In the majority of cases, the correlation between mobility and $R_0$ was strongest with no lag between the two. However, changes in mobility to workplaces lagged behind changes in $R_0$ in Basel-Stadt and Basel-Landschaft. Correlations between changes in $R_0$ and grocery and pharmacy mobility were less marked (national level 0.65), with changes in mobility occurring after changes in $R_0$ (negative lags in fig.~\ref{fig:covid-ch-timing}). In most cantons, a strong increase in park mobility after March 25 resulted in a positive correlation with changes in $R_0$, but with negative lags (change in activity after change in $R_0$, fig.~\ref{fig:covid-ch-timing}). 
The linear associations between the level of reduction in mobility and maximum reduction in $R_0$ across cantons were found to be not significant, except for a small effect size for reduced park mobility (regression coefficient of 0.15, 95\% CI 0.02–0.25) (postprint \textsc{si} fig. 8). 

The effective reproduction number $R_{eff}$, a common measure which is an aggregate measure of transmission capturing aspects of both infectious contacts and of population susceptibility was estimated. Across cantons, $R_{eff}$ was extremely close to $R_0$, indicating that a small fraction of the population is expected to have a natural immunity to SARS-CoV-2. 
As of April 24, an estimated 3.9\% (95\% QR 3.6–4.3\%) of the population nationally had been infected, with median estimates ranging from 1.9\% (Bern) to 16\% (Ticino) (fig.~\ref{fig:covid-ch-map}). 
Modeled estimates of the proportion infected of people infected in the canton of Geneva are in agreement with preliminary results from ongoing serological studies, which have estimated the seroprevalence to be 9.7\% (95\% CrI 6.1–13.1\%) in the third week of April\cite[-3\baselineskip]{Stringhini:RepeatedSeroprevalenceAntiSARSCoV2:2020}, compared with modeled estimates of 8.9\% (95\% QR 7.8–10.1\%) after accounting for the time from infection to seroconversion\cite{Wolfel:VirologicalAssessmentHospitalized:2020}(postprint \textsc{si} fig. 7).

\begin{figure}\centering
  \includegraphics[width=\textwidth]{fig_covid-switzerland-npi/FIGURE_5_mod.png}
  \caption[Modelled proportion of people infected with SARS-CoV-2 in Switzerland][2\baselineskip]{Modelled proportion of people infected with SARS-CoV-2 in Switzerland up to April 24. Estimates were produced for 11 of 26 cantons for which enough data were available (unmodelled cantons are shown in gray with points indicating the national-level estimated incidence proportion of 3\%). Values are reported in the postprint \textsc{si} tab. 6.}
  \label{fig:covid-ch-map}
\end{figure}

\section{Discussion}
Our results suggest a strong reduction of $R_0$ across Switzerland since the start of the epidemic. The reduction in $R_0$ started around March 7, thus about 1 week before the implementation of lockdown-type NPIs. Analysis of activity-related mobility data also showed strong support for changes in mobility starting before the implementation of most NPIs. Estimated reductions of viral transmission were strongly correlated and mostly synchronous with observed changes in mobility patterns, although the initiation of changes in transmission preceded measurable changes in activity-related mobility. 

The methods used to infer the time series of $R_0$ do not rely on assumptions on the shape of how it changed in time, nor on the dates at which changes started. Alternative methods that rely on fixed dates (such as that of the Imperial College \textsc{covid}-19 Response Team\cite[-6\baselineskip]{Flaxman:Report13Estimating:2020}) might be biased as changes in transmission are not synchronous with policy changes. Distribution based methods such as provided by R package EpiEstim\cite[-4\baselineskip]{Wallinga:DifferentEpidemicCurves:2004,Cori:NewFrameworkSoftware:2013} are flexible but subject to bias when misused\cite{Lipsitch:CommentPanLiu:2020}. In addition, the present approach enables the estimation of $R_0$, which is a direct quantification of transmission potential, as opposed to the effective reproduction number $R_{eff}$, which also accounts for the effect of susceptible depletion as done in the above-mentioned statistical approaches. This enabled us to estimate the proportion of reduction in transmission attributable to behavioral changes, which is therefore more suited to study the impact of NPIs. Aside from these methodological differences, the present estimates are in line with other estimates in Switzerland: Althaus et al.\cite{Althaus:RealtimeModelingProjections:2020} estimated a reduction of 89 \% (83–94\%) from a baseline of 2.78 (2.51–3.11), Scire et al.\cite{Scire:ReproductiveNumberCOVID19:2020} estimated a reduction of 76 \% (70–82\%) from a baseline of 1.88 (1.80–1.98) and Imperial College estimated a reduction of 60\% (50–80\%) from a baseline of 3.5 (2.8–4.3)\cite{Flaxman:Report13Estimating:2020}. 

The presented results provide strong support for a reduction of transmission starting about 1 week before school closure, the first national-level NPI targeting daily activities, which was ordered on March 13. Moreover, initiation of transmission reduction was found to precede changes in mobility patterns as detectable from the Google dataset. A possible explanation for this initial decrease in transmission could be linked to the strong increase in public interest in \textsc{covid}-19 in February as measured by Google searches for \textsc{covid}-19-related keywords (fig.~\ref{fig:googlemob}).
\begin{marginfigure}[1\baselineskip]
%\centering
\includegraphics{fig_covid-switzerland-npi/fig_supp/google_trends.png}
\margincaption[Google trends for \textsc{covid}-19 and changes in $R_0$ in Switzerland]{\footnotesize Google trends for \textsc{covid}-19 and changes in $R_0$ in Switzerland. Trends corresponds to the amount of searches for the keyword ``coronavirus'' (red line) between February 15 and April 30 and are given as a percent of the maximum number of searches in the period, the time evolution of $R_0$.}\label{fig:googlemob}
\end{marginfigure}
 In fact, a second sharp rise in Google searches is estimated to have started on March 7 (95\% CrI March 3–9), which overlaps with the estimated start of the national level decrease in $R_0$ on March 7 (probability that changepoints coincide 0.76). The Federal Office for Public Health issued an information campaign on \textsc{covid}-19 on February 28, which was updated on March 2 to stress basic hygiene rules\cite{OFSP:NouvellesReglesHygiene:2020}. This may have resulted in voluntary social distancing as well as increased hygiene early on without noticeable changes in mobility patterns. This type of proactive change in behavior would be in line with early changes in mobility patterns, which were estimated to precede school closures and subsequent measures. The present results suggest that the value of $R_0$ was likely already below one on March 20, when the federal government banned gatherings of more than five people and recommended voluntary home isolation for the whole population. This result should however be taken within context, as the announcement was anticipated on social networks earlier that week, and so was probably already impacting social distancing behavior. 
Therefore caution is recommended in any causal interpretation of these results on the role of this last NPI on driving $R_0$ below one.

Despite the strong association between the changes in mobility and reductions in $R_0$ within each canton, the lack of cross-cantonal associations between the level of reduction in mobility types and the level of reduction in $R_0$ suggests context-specific pathways between \textsc{covid}-19 transmission and mobility intensity. These warrants caution in attempting to apply general relations between mobility and transmission reduction. Investigation of general associations will require more in-depth studies controlling for other factors such as population density, economic activities and social mixing patterns, and inter-cantonal mobility patterns, in addition to the incorporation of potential environmental drivers of transmission such as temperature and relative humidity\cite{Neher:PotentialImpactSeasonal:2020, Kissler:ProjectingTransmissionDynamics:2020}. 
  
  Several limitations to this work are noted. First, due to the relatively recent introduction of SARS-CoV-2 in Switzerland compared with the length of hospital and ICU stays, the time distribution of hospital in- and out-patients is biased towards shorter duration (see \textsc{Appendix}), which is addressed by accounting with right-censoring using survival models. In addition, because of the limited data available in some places, it was only possible to fit the model for 11 of the 26 cantons. Modeling results presented in this work are subject to hypothesis on yet uncertain parameters of \textsc{covid}-19, including the infection fatality rate and the proportion of severe infections requiring hospitalization. An important uncertainty is the fraction of asymptomatic infections and their relative contribution to disease transmission. It is assumed that all infected individuals contribute equally to transmission, which means the estimation of the proportion of people infected would under-estimate true cumulative incidence if there were a large fraction of asymptomatic infections with a relatively low contribution to transmission. Evidence from South Korea, however, suggests that only a small fraction (2\%) of confirmed \textsc{covid}-19 infections are asymptomatic, and none of the household members of these asymptomatic carriers were infected\cite[-4\baselineskip]{Park:EarlyReleaseCoronavirus:2020}. Moreover, model results are in agreement with preliminary results from ongoing serological studies in Switzerland\cite[-2\baselineskip]{Stringhini:RepeatedSeroprevalenceAntiSARSCoV2:2020}.  The presented estimates of time-varying basic reproduction numbers assume that the generation interval for \textsc{covid}-19 in Switzerland remained unchanged, thus potentially ignoring the joint role of $R_0$, the infectious period, and contact rates in determining the disease’s intrinsic growth rate\cite{Yan:SeparateRolesLatent:2008}. If the generation interval increased with the reduction of social contact then these estimates are conservative overestimates of the “true” value of $R_0$, which is encouraging from a public health perspective. Inferred disease dynamics and estimated time-varying $R_0$ also depend on the values of the incubation period, which is set to the estimates currently available in the literature. 
  In the present modeling framework, the initial conditions were estimated along with changes in $R_0$, which could, however, be influenced by the role of imported cases in driving disease dynamics, especially in cantons bordering regions with strong \textsc{covid}-19 transmission in early February (Eastern France for Basel-Stadt and Basel-Landschaft and Northern Italy for Ticino). 
  Since importations were not modeled, this could yield an overestimation of the initial value of $R_0$, which warrants caution in interpreting specific values of $R_0$ in these cantons. This potential overestimation would, however, not affect the strong inferred reduction in $R_0$. Another limitation of this study is that it was not possible to disentangle the individual contribution of each NPI on $R_0$ in this analysis owing to the early onset of changes in $R_0$ and in mobility patterns, as well as the very close spacing between the different types of NPI. This information would, however, be extremely valuable in supporting decisions on NPI strategies against \textsc{covid}-19. Efforts to constitute a global database of NPIs will provide the opportunity to extend this type of analysis to other settings and produce evidence for the effect of different types of NPIs\cite{HITCOVIDTeam:HealthInterventionsTracking:2020}. 
  
  As the Swiss government plans to gradually lift restrictions, close monitoring of changes in $R_0$ is critical, given that the reductions in transmission appear to be almost entirely driven by changes in behavior, not through herd immunity. Near real-time estimates of $R_0$ may serve as a critical tool for public health and political decision makers in the months to come, and efforts should be made to refine models like ours using new data, including those from population-based serological studies, mobility data, and more detailed individual-level data on \textsc{covid}-19 cases across the spectrum of severity.\marginnote[-4\baselineskip]{While relying on hospitalization and death allowed for an early robust identification of the basic reproduction number, it would be necessary to add a reporting process and case data to update this estimate through 2020-2021. Methods based on observed data are easier to maintain, and reliable continuous updates of the reproduction number in Switzerland are available on the Swiss National \textsc{covid}-19 Task Force website \url{sciencetaskforce.ch/en/current-situation/}. It is provided by the ETHZ, with the method described in \fullcite{Huisman:EstimationWorldwideMonitoring:2021}. The modeled incidence from the present \textsc{Chapter} has been later used as ``truth'' to compare waste-water data with reported cases, see: \fullcite{Fernandez-Cassi:WastewaterMonitoringOutperforms:2021}.}

