\chapter{A modelling-oriented primer on cholera}
Cholera is an acute intestinal infection causing severe diarrhea that may lead to dehydration, and sometimes death. While the global burden of cholera is difficult to estimate as the majority of cases are not reported, it ia estimated that 3 million cases and 95'000 deaths occurs every year in endemic areas (around 50 countries)\cite{Ali:UpdatedGlobalBurden:2015}, with millions more at risk. Despite its household name, cholera is a neglected tropical diseases and our understanding of many important aspects of cholera clinical course and transmission is limited.
A political will to eliminate this ancient disease has recently arisen. The World Health Organization (WHO) initiated the Global Task Force for Cholera Control (GTFCC), who provides a concrete path towards the elimination of cholera by 2030\footnote{Elimination being defined as a 90\% reduction of cholera deaths/year.}. The general consensus is that to reach this goal in endemic countries, there must be substantial long-term improvements in education, sanitation, hygiene, medical treatment and prevention.  Moreover, in the event of a cholera outbreak, timely interventions such as vaccination and awareness campaigns are crucial to limit the spread of the disease. %With limited resources, public health officials face a number of challenging decisions, and a data driven decision support to guide the rational deployment of cholera control strategies is needed. Furthermore, on the road toward elimination, the need to setup context-specific tailored approaches appears. 

%The poorest of the poor

Cholera is an infectious disease that, if left untreated, may lead to death in a few hours. 
%most infected individuals are asymptomatic, i.e., they do not present symptoms, while other experience mild or severe symptoms.  %Since their mobility is not hindered, they become a vector of the infection. If not properly treated, cholera can kill children and adults within hours. 
\section{History and epidemiology}
\begin{marginfigure}[3\baselineskip]
%\centering
\includegraphics[width=\textwidth]{fig/snow-cholera-map}
\caption[John Snow map of cluster cholera cases in London, 1854]{Original map by John Snow. Stacked rectangles diplay cholera cases of the 1854 Broad Street outbreak. The work of John Snow conviced the authorities to close the water pump (circled in red), leading to a decrease in Mortality.}\label{johnsnow}
\end{marginfigure}
Humanity and cholera share a long history, with supposed mentions as early the 5th centurey \textsc{bce}. The disease become more widely know in the modern era. From 1817 to 1923, six successive pandemics -- all originated in the delta of the Ganges river -- occured. Cholera spread around the world owning to the the nascent mobility, leaving 10s of millions of death in many countries.
  In 1854, during the third cholera pandemic (1846-60), a cholera outbreak in Broad Street (Soho, London) was studied by physician John Snow (see\textcite{Snow:ModeCommunicationCholera:1855}) and lead to an impressive early work in investigative epidemiology and public health. Analyzing the contamination pattern among resident, Snow postulated that cholera spread through water contaminated by an infectious agent, instead of foul air\footnote{The accepted mode of contamination for cholera and many other disease was through miasma, \textit{bad air} contaminated by organic matter. Chasing odor  justified urban amangement along streets and river banks in Paris and London. But also in Lausanne, where rivers Flon and Louve where covered in 1832 in response to a cholera outbreak. An fortunate consequence this cholera in Switzerland is the cholera dish from Valais.}.  Simultaneously, Italian microbiologist Filippo Pacini isolated the bacterium in Florence, further disproving miasma theory. Thirty years after Pacini, German scientist Robert Koch independently rediscovered the cholera pathogens after investigation in Egypt and India and its causative relationship with the disease. A hundred year later, Indian researcher Sambhu Nath De discovered the cholera toxin in 1959.

During the seventh pandemic from 1961 onwards, cholera spread in several waves, through asia in the 1960s, and reached Africa and the Middle East in the 1970s and the Americas in 1991\cite{Mutreja:EvidenceSeveralWaves:2011}. Improvement in sanitation and hygiene spared higher income country from disease, and cholera became a burden of the Global South. Series of outbreaks (\eg Zimbabwe 2008, Haïti 2010) and continuous epidemic in edemic cout

The span of this thesis was marked by a cholera outbreak in Yemen (2016--2021), an humanitarian crisis with 2.5M suspected cases and nearly 4'000 deaths, but also an outbreak in Zimbabwe and flare up in Algeria in addition to seasonal and  endemic cholera in the Asia and sub-saharian Africa.

While the number of cholera cases doubled from 2018 to 2019 to nearly a million, reported cholera death decreased to less than 2'000 in 2019, with Africa reporting lowest number since the 2000s. Haiti reported its last confirmed cholera cases in january 2019, bringing hope to cholera elimination from its last foothold in the Americas. Effort towards improvement of sanitary conditions and reactive vaccinations (24M doses of cholera vaccines were distribued in 2019) hope to bring these number down\cite{WHO:Cholera2019:2020}. 
In the following we highlight some relevant aspects about cholera transmission, while leaving some biological and medical features of cholera outside the scope of this thesis.


\section{Pathogen} 
\begin{marginfigure}[3\baselineskip]
\centering
\includegraphics{fig/vibrio}
\caption[Vibrio cholerae bacteria]{Scanning electron microscope image of \textit{Vibrio cholerae}.\small{(Public domain image by Ronald Taylor, Tom Kirn, Louisa Howard)}}
\label{rain}
\end{marginfigure}

\paragraph{Pathogen} Cholera is an infection caused by a waterborne bacteria: the \emph{Vibrio cholerae}. While many serogoups of \emph{V. cholerae} can secrete the cholera toxin reponsible for massive watery diarrhea, only serogoups O139 and O1 are responsible for disease epidemics. O1 is causing most recent epidemics, and is divided in two biotypes: Classical O1 and El Tor, which are both divided into three serotypes: Ogawa, Inaba and the rare Hikojima\cite{Kaper:Cholera:1995}. Cholera classification has its importance as it affects many epidemiological characteristic \eg, El Tor survives longer in water and has an higher asymtomatic/symptomatic ratio\cite{WHO:CholeraVaccinesWHO:2017}. The current cholera pandemic is mainly caused by El Tor, while the Classical derivative caused previous fifth and sixth pandemic and is now cantoned to the Gange delta\cite{Nair:CholeraDueAltered:2006}. 

 \paragraph{Environmental reservoir}  \textit{V. cholerae} exists as natural habitant in some aquatic ecosystem, in particular in brackish waters and estuaries. There is no marine host, but complex ecological association processes take place in the aquatic medium, and natural genetic transformation is enabled by chitin, the polymer of the crustacean exoskeleton\cite{Reidl:VibrioCholeraeCholera:2002,Meibom:ChitinInducesNatural:2005}. There is no concensus on how long \textit{V. cholerae} remains infectious in water and under which conditions it is able to reproduce\cite{Mavian:ToxigenicVibrioCholerae:2020}. In cholera epidemics, it is difficult to isolate the role of the natural cholera reservoir from freshly introduced \textit{Vibrios} from the feces of an infected person.
 
\paragraph{Climatic Drivers} Cholera express a marked seasonality that takes different shapes in different countries. A complex and unclear association between precipitation and cholera infections has been put forward in many research works. After hurricane Matthew and heavy rainfall in October 2016 a cholera epidemic started from few cases in Haiti\cite{Rinaldo:Reassessment20102011:2012, Gaudart:SpatioTemporalDynamicsCholera:2013}. Similarly, cholera in many African countries follows a seasonal trend, with the epidemiological curve raising during the rainy season\cite{Baracchini:SeasonalityCholeraDynamics:2017}.  Rainfalls might play a major role in water contamination, for instance through the washout of open-air defecation and raw sewage circulation in the environment. More information on cholera and rainfall is provided in the next chapter. Temperature, water acidity, sunlight and other environemental factors have also been shown to affect the survival and reproduction of \textit{V. cholerae} in water bodies. Hence macro climate phenomena such the El Niño Southern Oscillation have been associated with changes in transmission, even if no causal link could be established\cite{Pascual:CholeraDynamicsNinoSouthern:2000}.
\begin{marginfigure}
\centering
\includegraphics[width=\textwidth]{fig/cholera-rainfall.png}
\caption[Daily cholera cases and rainfall in Haiti]{Daily cholera cases (red) and daily rainfall (blue) in Haiti from September 15, 2010 to October
16, 2011. We observe a correlation between heavy rainfall event and case resurgence. Adapted from \fullcite{Gaudart:SpatioTemporalDynamicsCholera:2013}.}
\label{rain}
\end{marginfigure}

\section{Cholera in the human} 
\paragraph{Disease} An human host becomes infected through the ingestion of a critical dose of \emph{V. cholerae}\cite{Kaper:Cholera:1995,Nelson:CholeraTransmissionHost:2009}. \textit{V. cholerae} colonize the small intestine for an incubation period lasting 12 hours to 5 days\cite{Azman:IncubationPeriodCholera:2013} before symptoms. Then, a wide range of outcomes are possible. Most of the time the infection is inapparent, resulting in asymptomatic individuals. On the other end of the spectrum severe infection (or cholera gravis), caracterized by vomiting and profuse rice water diarrhea, occurs in 1\% to 15\% of the cases. Many stages of mild infections lie between asymptomatics and severe infections.  Most symptoms are indistinguishable unless tested in a laboratory from those of numerous other infections causing diarrhea\cite{King:InapparentInfectionsCholera:2008, Kaper:Cholera:1995, Nelson:CholeraTransmissionHost:2009,vandeLinde:ObservationsSpreadCholera:1965,Mccormack:CommunityStudyInapparent:1969}.  The severity of illness correlates with the number of \textit{V. cholerae} ingested\cite{Brouwer:DoseresponseRelationshipsEnvironmentally:2017}, and depends on the cholera strain and personal characteristics, immunity, pregnancy, blood type, ...\cite{WHO:CholeraVaccinesWHO:2017,Azman:IncubationPeriodCholera:2013}.%Table 1 for a review of individual data studies.
Severely infected individual migth loose up to 20 litres  of diarrhea a day, treatment is crucial: untreadted, natural mortality may reach up to 60\%, but a proper therapy lowers mortality below 1\%\cite{Luquero:MortalityRatesCholera:2016}.

\paragraph{Shedding and transmission} The intensity of the bacterial shedding varies with the intensity of the infection. It is estimated to range from $10^3$ (asymptomatic) to $10^{9}$ (heavy symptoms) vibrios per gram of stool for asymptomatic infected and severely infected individual respectively\cite{Nelson:CholeraTransmissionHost:2009}. Similarly, the duration of the shedding period typically ranges from a day up to two weeks\cite{Nelson:CholeraTransmissionHost:2009, Kaper:Cholera:1995}.
Transmission occurs along the fecal-oral route (consumption of contaminated water or food, contact with fomites), but contamination is also possible from aquatic reservoirs, through water or seafood. Freshly shed \textit{Vibros} may be in an hyperinfectious state, which could be of great importance in driving epidemic transmission\cite{Butler:CholeraStoolBacteria:2006}. The principle mechanism of transmission is the intake of water contaminated by the untreated diarrheal discharge of other infectious individuals.
This suggests that asymptomatic individuals may be of importance for cholera transmission. The susceptibility depends on many factors such as gastric acidity and age, with under 5 much more likely to become infected\cite{Sack:Cholera:2004}.

\paragraph{Reporting} Without equipement, it is impossible to distinguish cholera from another pathogen in a patient with acute watery diarrhea. The has been an effort to standardize the clinical definition of a suspected cholera case, which varies between countries and during outbreaks. A suspected case combine acute watery diarrhea and severe dehydration, the latter condition being dropped in case of outreak. This diagnosis can be validated using rapid diagnosis tests (RTDs), with a pretty high sensibility but low sensitivity. Precise information is obtained through culture, the current gold standard. Test results are not always available, especially during an outbreak.  As a consequence, over-reporting during an outbreak situation is likely, as is under reporting as transmission settings might be isolated or plagued with conflicts, natural disaster. WHO guidelines recommend that, when a patient enters a treatment center, his name, address, sex, age (over or below 5) and symptoms are recorded\cite{WHO:FirstStepsManaging:2010}. % https://apps.who.int/iris/bitstream/handle/10665/334241/WER9537-eng-fre.pdf?ua=1 end, and https://www.cdc.gov/cholera/diagnosis.html

\paragraph{Immunity} Infected individuals that recover from the infection are immunized against the same \textit{V. cholerae} serogroup. The duration of acquired immunity is difficult to estimate, and depends on many factors. Acquired immunity has been reported to range from few months to several years, possibly depending on the virulence of the infection\cite{Levine:DurationInfectionDerivedImmunity:1981,Kaper:Cholera:1995,Woodward:CholeraReinfectionMan:1971,Glass:SeroepidemiologicalStudiesEI:1985,Clemens:BiotypeDeterminantNatural:1991,Leung:ProtectionAffordedPrevious:2021}.

\paragraph{Human mobility and hydrological transport} Spatial spread of cholera outbreaks may occurs through two networks. \textit{V. Cholerae} may be transported through the river network. A example is the spread of the 2010 Haiti epidemic along the Artibonite river\cite{Piarroux:UnderstandingCholeraEpidemic:2011}. Human mobility also plays a major role in the spreading of the infections possibly due to the large number of asymptomatic that transport and disperse cholera across a country or even worldwide.  Cholera was brought into Haiti by infected United Nations peacekeepers\cite{Piarroux:UnderstandingCholeraEpidemic:2011}. %However, mobility, especially in humanitarian crisis situations often associated with cholera, remains difficult to predict\cite{Lu:PredictabilityPopulationDisplacement:2012,Riley:LargeScaleSpatialTransmissionModels:2007,Bengtsson:ImprovedResponseDisasters:2011,Rebaudet:DrySeasonHaiti:2013}.

\subsection{Intervention Strategies} 
Cholera interventions tackle may be preventive or concern the treatment of infected individual. Treatment plays an important role in reducing the reproduction number of the epidemic and consists consists of:
\begin{description}
\item[Oral (or Intravenous) Rehydratation Therapy] The main treatement for cholera consists of replacing fluids as fast as they are lost. Despite its simplicity, it is very effective in reducing mortality. Fluids with the same electrolyte composition must be administred\cite{Kuhn:GlucoseNotRiceBased:2014}.  Rehydratation is usually done in treatment centers, but may take place at the patient home. This differentiation might determine if stools contribute to the infection cycle or are properly disposed.
\item[Antibiotics] reduce the severity and the duration of the infection. WHO recommends their use only for the most severe cases as antibiotic resistance of \emph{V. cholerae} is raising worldwide\cite{Sack:GettingSeriousCholera:2006}.
\end{description}

Prevention measures may be carried out before and during the outbreak. They are divided into surveillance, vaccination and water, sanitation and hygiene (WaSH).

\paragraph{Surveillance} Prevention starts from surveillance. During an outbreak, it consists of the timely reporting of new cases. In many countries where outbreaks occurs annually during the rainy season, the observation of past epidemics provides insight on the severity and timing of the infection that can be used for preparation\cite{Baracchini:SeasonalityCholeraDynamics:2017}. Environmental surveillance, monitoring for \textit{V. Cholerae} in the water, is also possible. However, never has \textit{V. Cholerae} been found in the environment before an epidemic.


\paragraph{Vaccination} is a safe and effective way to protect individuals from cholera, and to reduce the propagation of the epidemic. It can be used in a preventive or reactive way. Several vaccine exists for cholera, with different characteristics. As of today, two main oral cholera vaccines (OCVs) are used in vaccination campaigns around the world: WC-rBS and BivWC\footnote{An other vaccine, Vaxchora, was recently approved by the FDA, mostly for travelers.}. The main characteristics of these two vaccines are shown in Table~\ref{tab:vacc}\cite{WHO:CholeraVaccinesWHO:2017,WHO:BackgroundPaperWholeCell:2017,Azman:PopulationLevelEffectCholera:2016,Luquero:FirstOutbreakResponse:2013}. Vaccine can either be administered in a targeted fashion or to whole populations in mass vaccination campaigns. Despite effort to build a worldwide vaccine stockpile, demand for cholera vaccine vastly exceeds supply\cite{Parker:AdaptingGlobalShortage:2017a,Seidlein:PreventingCholeraOutbreaks:2018}.
\begin{table*}[h]
\centering\small
\label{tab:prior}
\begin{tabular}{lp{50mm}p{50mm}}
\toprule
Generic Name &  BiWC & WC-rBS\\ 
\midrule
Commercial name   &  mORCVAX, Shanchol,  Euvichol, Cholvax & Dukoral  \\
Target strain O1 &   yes (classical, El Tor, Ogawa, Inaba)& yes (classical, El Tor, Ogawa, Inaba), also  target a cholera toxin  \\
Target strain O139   &  yes &      no     \\
Doses   &  2 doses, 2 weeks apart & 2 doses (3 for children) 1--6 weeks apart  \\
%Vaccine Efficacy & 58\% &  \\
Field Effectiveness  & between 37\% and 87\% for two years & 78\% protection 1--6 months after vaccination\\
Age   &  $>$ 1 year & $>$ 2 year      \\
Usage & Mass vaccination, Global OCV stockpile, 25M doses administered & Mainly for travelers ($>$ 1M doses administered)\\
Protection length & 3 years (1 dose: short term protection) & 2 years\\
Constraints & -- & needs buffer solution\\
Price per dose & 1.85\$ & 5.25\$ \\ 
Usage & Since 1998 in nearly all recent outbreaks & Between 1997 and 2009 in Uganda, Tanzania, Indonesia,~... \\
\bottomrule
\end{tabular}
\caption{Characteristic of currently available vaccines. The proposed value for field effectiveness, along with vaccine efficacy (not shown here) is really difficult to evaluate, cannot be written in a simple way without omitting crucial information. A good review of research on cholera vaccine is the WHO Position paper on cholera vaccines~\fullcite{WHO:CholeraVaccinesWHO:2017}. See also~\parencite{WHO:BackgroundPaperIntegration:2009,Luquero:UseVibrioCholerae:2014,Qadri:EfficacySingledoseRegimen:2018,Bi:ProtectionCholeraKilled:2017,Azman:ImpactOneDoseTwoDose:2015,Tohme:OralCholeraVaccine:2015}.}
\label{tab:vacc}
\end{table*}

\paragraph{WaSH} Water, Sanitation and Hygiene (WaSH) is a broad term that includes many intervention strategies that are key to the long term elimination of cholera. The improved sanitary conditions have been the main factor that led to cholera elimination  in first world countries. WaSH is divided into short and long term measures. Short term strategies involve sterilization, decontamination, hand washing, education sessions and water purification and filtering (chlorination, ...)\cite{Rebaudet:NationalAlertresponseStrategy:2018,Fewtrell:WaterSanitationHygiene:2005}. Long-term sanitation strategies involve the construction of infrastructures for fecal sludge management, sewage systems, toilets and access to safe water sources. From a modeling point of view, WaSH reduces exposure (water purification, sari filtration) and shedding (sewage and fecal sludge management). By its nature, WaSH improvement is difficult to quantify in modeling framework.

\section{Cholera Modeling}
Cholera modeling has received renewed a lot of attention during the 2010 Haiti outbreak, and serves different purposes. The ultimate goal being  the forecasting and estimating the propagation of uncertainties from the observable past to the future. Alongside, modeling helps us to understand the different processes of the cholera life cycle. As a substitute for physical experiments, different mechanistic pathways can be compared by benchmarking them against the reproduction of reported cases. 

Models may have only one spatial dimension or may consider the spatial spread of the epidemic across several nodes. Spatially-explicit mathematical models provide key insights into the course of an ongoing epidemic, where transmission is heterogeneous in space and time.

%%%%%%%%%%%%%%%%%%%%%%%%%%%%%%%%%%%%%%%%%%%%%%%%%%%%%%%%%%%%%%%%%%%%%%%%%%%%%%%%%%%%%%%%%%%% ok
%\subsection{Spatially-explicit cholera model}
A spatially-explicit model has been developed at ECHO in the past 10 years\parencite{Bertuzzo:SpacetimeEvolutionCholera:2008}. Studies on the dynamics of several cholera epidemics, occurred in South Africa (2000)\parencite{Mari:ModellingCholeraEpidemics:2012}, Senegal (2005), Haiti (2010-2016)\parencite{Bertuzzo:PredictionSpatialEvolution:2011}, Democratic Republic of the Congo  (2004-2011), stand as a significant proof of concept.  
We present here a complete formulation of the model, including patterns and effectiveness of vaccinations, human and hydrological mobility\cite{Bertuzzo:ProbabilityExtinctionHaiti:2016,Pasetto:RealtimeProjectionsCholera:2017}



%%%%%%%%%%%%%%%%%%%%%%%%%%%%%%%%%%%%%%%%%%%%%%%%%%%%%%%%%%%%%%%%%%%%%%%%%%%%%%%%%%%%%%%%%%%% ...
%\subsection{Other cholera models}

Recently, cholera modeling has received a lot of attention. Mechanistic cholera models embed uncertainties into their parameter distribution, and differ in the way they account for the epidemiological processes\cite{Kirpich:ControllingCholeraOuest:2017,Tuite:CholeraEpidemicHaiti:2011,Chao:VaccinationStrategiesEpidemic:2011,Kirpich:CholeraTransmissionOuest:2015}. For example, Einsenberg \textit{et al.} accounts for rainfall with a multiplicative effect on the force of infection\cite{Eisenberg:ExaminingRainfallCholera:2013,Eisenberg:IdentifiabilityEstimationMultiple:2013,Eisenberg:CholeraModelPatchy:2013}. Most models are spatially implicit, however there have been a number of attempts to describe the spatial spread of the epidemic. For example, Andrew and Basu used an approach with isolated nodes and independent transmission parameters in each node\cite{Andrews:TransmissionDynamicsControl:2011}.  Mechanistic models can be compared using formal model selection criteria like the Akaike Information Criteria (AIC)\cite{Akaike:NewLookStatistical:1974}, thus allowing for the contrasting of different formulation of the underlying physical processes\cite{Baracchini:SeasonalityCholeraDynamics:2017,King:InapparentInfectionsCholera:2008,Akman:ExaminationModelsCholera:2016, Rinaldo:Reassessment20102011:2012}. A detailed example of model comparison between Eisenberg's and the above SIRB model is shown in \textbf{section 6.1}. Most of recent modeling efforts focus on phenomenological (or statistical) models with different degrees of deterministic processes\cite{Azman:UrbanCholeraTransmission:2012,Finger:PotentialImpactCasearea:2018,Camacho:CholeraEpidemicYemen:2018,Lessler:MappingBurdenCholera:2018,Koelle:DisentanglingExtrinsicIntrinsic:2004}.


%ffective targeted interventions 
%could eliminate 50\% of the region’s cholera by covering 35·3 million people (95% CrI 26·3 million to 62·0 million),
%which is less than 4\% of the total population\cite{lessler_mapping_2018} + hotspot vs optimal strategiy
%hotspot\cite{azman_micro-hotspots_2018}

%The Dry Season in Haiti: a Window of Opportunity to Eliminate Cholera\cite{rebaudet_dry_2013}

%\paragraph{Interventions design} The planning of the interventions described above is difficult to establish, as many factors enter into consideration including logistical and political constraints. Expert opinion is a valuable resource, however its use during outbreaks is not necessarily feasible and a consensus on strategy choice does not always emerges\cite{Cyranoski:CholeraVaccinePlan:2011}. Moreover, the global vaccine shortage\cite{Parker:AdaptingGlobalShortage:2017a,Seidlein:PreventingCholeraOutbreaks:2018} calls for an optimal use of current resources. Mass vaccination campaigns should be conducted only when strictly necessary. Case-area targeted interventions (CATIs) are an effective way of mitigating an outbreak while saving scarce resources. However, the optimal allocation in time and space of such interventions is strongly context-dependent which hinders the definition of general guidelines\cite{Eubank:ModellingDiseaseOutbreaks:2004,Finger:PotentialImpactCasearea:2018,Seidlein:PreventingCholeraOutbreaks:2018,Azman:MicrohotspotsRiskUrban:2018,Lessler:MappingBurdenCholera:2018,Rebaudet:DrySeasonHaiti:2013}. 

%This thesis will explore the possibility to apply optimal control for  both short- and long-term interventions across all scales. 